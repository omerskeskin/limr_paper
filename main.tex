\documentclass[a4paper,UKenglish,cleveref, autoref, numberwithinsect, anonymous, thm-restate]{lipics-v2021}
%This is a template for producing LIPIcs articles. 
%See lipics-v2021-authors-guidelines.pdf for further information.
%for A4 paper format use option "a4paper", for US-letter use option "letterpaper"
%for british hyphenation rules use option "UKenglish", for american hyphenation rules use option "USenglish"
%for section-numbered lemmas etc., use "numberwithinsect"
%for enabling cleveref support, use "cleveref"
%for enabling autoref support, use "autoref"
%for anonymousing the authors (e.g. for double-blind review), add "anonymous"
%for enabling thm-restate support, use "thm-restate"
%for enabling a two-column layout for the author/affilation part (only applicable for > 6 authors), use "authorcolumns"
%for producing a PDF according the PDF/A standard, add "pdfa"

%\graphicspath{{./graphics/}}%helpful if your graphic files are in another directory
\usepackage{prooftree}
\usepackage{mathpartir}
\usepackage{xcolor}
\usepackage{listings}
\usepackage{tikz-cd}
\usepackage{mpst_macros}
\usepackage{stmaryrd}
\usepackage{amsmath}
\usepackage{hyperref}
\usepackage{todonotes}
\usepackage{url}
\usepackage{wasysym}
\def\delequal{\mathrel{\ensurestackMath{\stackon[1pt]{=}{\scriptstyle\Delta}}}}


\newcommand*\justify{%
  \fontdimen2\font=0.4em% interword space
  \fontdimen3\font=0.2em% interword stretch
  \fontdimen4\font=0.1em% interword shrink
  \fontdimen7\font=0.1em% extra space
  \hyphenchar\font=`\-% allowing hyphenation
}


\newcommand{\newpar}[1]{
\bigskip
\noindent\textbf{#1}\ }

\newcommand{\ct}[1]{\rem{CT: #1}}
\newcommand{\cb}[1]{\rem{CB: #1}}

\newcommand{\smtcoq}{SMTCoq\xspace}
\newcommand{\coq}{Coq\xspace}
\newcommand{\zchaff}{ZChaff\xspace}
\newcommand{\verit}{veriT\xspace}
\newcommand{\cvcfour}{CVC4\xspace}
\newcommand{\lfsc}{LFSC\xspace}
\newcommand{\ocaml}{OCaml\xspace}

\newcommand{\rbvl}{\texttt{RAWBITVECTOR\_LIST}\xspace}
\newcommand{\bvl}{\texttt{BITVECTOR\_LIST}\xspace}
\newcommand{\bv}{\texttt{BITVECTOR}\xspace}
\newcommand{\rbv}{\texttt{RAWBITVECTOR}\xspace}
\newcommand{\rtbv}{\texttt{RAW2BITVECTOR}\xspace}


% serif sections
\definecolor{colorsec}{HTML}{345A8A}
\definecolor{colorsubsec}{HTML}{4F81BD}
\definecolor{colorsubsubsec}{HTML}{5388C8}
% \addtokomafont{disposition}{\rmfamily}
% \usepackage{relsize}
% \addtokomafont{disposition}{\color{colorsec}}
% \addtokomafont{section}{\LARGE}
% \addtokomafont{subsection}{\Large\color{colorsubsec}}
% \addtokomafont{subsubsection}{\color{colorsubsubsec}}
% \addtokomafont{paragraph}{\color{colorsubsubsec}}

\hypersetup{
  colorlinks = true,
  citecolor=purple,
  linkcolor=colorsubsec,
}


\definecolor{bg}{rgb}{0.95,0.95,0.95}

\makeatletter

\usepackage[framemethod=tikz]{mdframed}
\usepackage[most]{tcolorbox}

% |===| Lustre listings.
  \lstset{
    % frame=tb,framerule=0pt,
    aboveskip=0pt,
    belowskip=0pt,
    % framexleftmargin=10pt,
    % framextopmargin=2pt,
    % framexbottommargin=2pt, 
    showstringspaces=false,
    columns=fullflexible,keepspaces=true,
    basicstyle={\ttfamily\small\upshape},
    % backgroundcolor=\color{bg},
    numbers=none,
    numberstyle=\tiny\color{gray},
    breaklines=true,
    breakatwhitespace=true,
    mathescape=true,
    tabsize=3,
    commentstyle=\itshape\color{black!50},
    keywordstyle=\color{green!50!black}\bfseries,
    stringstyle=\color{purple},
  }


\def\beginlstdelim#1#2#3#4%
{%
  \def\endlstdelim{#2\egroup}%
  {\ttfamily#3#1}\bgroup#4\aftergroup\endlstdelim%
}



\lstdefinelanguage{smt}{
  language=lisp,
  alsoletter=0123456789>=,
  keywords={define-fun,declare-fun,declare-const,set-option,echo,set-info,exit,pop,push,assert},
  classoffset=1,
  morekeywords={Int,Real,Bool},keywordstyle=\color{blue!60!black}\bfseries,
  classoffset=2,
  morekeywords={not,and,or,=>},keywordstyle=\bfseries,
  classoffset=3,
  morekeywords={check-sat,check-sat-assume},keywordstyle=\color{red}\bfseries,
  classoffset=4,
  morekeywords={true,false,0,1,2,3,4,5,6,7,8,9,10},keywordstyle=\color{violet}\bfseries,
  classoffset=0,
  moredelim=**[is][\beginlstdelim{define-fun\ }{\ }{\color{green!50!black}\bfseries}{\color{blue!80!black!50!white}\bfseries}]{define-fun\ }{\ },
  moredelim=**[is][\beginlstdelim{declare-const\ }{\ }{\color{green!50!black}\bfseries}{\color{blue!80!black!50!white}\bfseries}]{declare-const\ }{\ },
  moredelim=**[is][\beginlstdelim{declare-fun\ }{\ }{\color{green!50!black}\bfseries}{\color{blue!80!black!50!white}\bfseries}]{declare-fun\ }{\ },
  mathescape=false
}


\lstdefinelanguage{Coq}{
  alsoletter={<}{:}0123456789,
  morekeywords={Variable,Variant,Inductive,CoInductive,Fixpoint,CoFixpoint,%
    Definition,Program, Lemma,Theorem,Corollary,Axiom,Local,Save,Grammar,Syntax,Intro,%
    Trivial,Qed,Intros,Symmetry,Simpl,Rewrite,Apply,Elim,Assumption,%
    Left,Cut,Case,Auto,Unfold,Exact,Right,Hypothesis,Pattern,Destruct,%
    Constructor,Defined,Fix,Record,Proof,Induction,Hints,Exists,%
    Parameter,Split,Red,Reflexivity,Transitivity,if,then,else,Opaque,Module,%
    Transparent,Inversion,Absurd,Generalize,Mutual,Cases,of,Analyze,%
    AutoRewrite,Functional,Scheme,params,Refine,using,Discriminate,Try,%
    Require,Load,Import,Scope,Open,Section,End,Ltac,fun,forall,exists,Canonical,Structure,Eval,Notation,as,return,Goal,Class,Module%
  },%
  classoffset=1,
  morekeywords={Type,Prop,bool,nat,Set,let,in,match,with,end,as,<:,Z,farray,bitvector},keywordstyle=\color{blue!60!black}\bfseries,
  classoffset=2,
  morekeywords={Error:,Warning:},keywordstyle=\color{red}\bfseries,
  classoffset=3,
  morekeywords={0,1,2,3,4,5,6,7,8,9,10,11,12,13,14,15,16,16,18,19,20},keywordstyle=\color{violet},
  classoffset=0,
  sensitive, %
  moredelim=**[is][\beginlstdelim{CoInductive\ }{\ }{\color{green!50!black}\bfseries}{\color{blue!80!black!50!white}\bfseries}]{CoInductive\ }{\ },
  moredelim=**[is][\beginlstdelim{Inductive\ }{\ }{\color{green!50!black}\bfseries}{\color{blue!80!black!50!white}\bfseries}]{Inductive\ }{\ },
  moredelim=**[is][\beginlstdelim{Definition\ }{\ }{\color{green!50!black}\bfseries}{\color{blue!80!black!50!white}\bfseries}]{Definition\ }{\ },
  moredelim=**[is][\beginlstdelim{Lemma\ }{\ }{\color{green!50!black}\bfseries}{\color{blue!80!black!50!white}\bfseries}]{Lemma\ }{\ },
  moredelim=**[is][\beginlstdelim{Axiom\ }{\ }{\color{green!50!black}\bfseries}{\color{blue!80!black!50!white}\bfseries}]{Axiom\ }{\ },
  moredelim=**[is][\beginlstdelim{Theorem\ }{\ }{\color{green!50!black}\bfseries}{\color{blue!80!black!50!white}\bfseries}]{Theorem\ }{\ },
  moredelim=**[is][\beginlstdelim{Class\ }{\ }{\color{green!50!black}\bfseries}{\color{blue!60!black}\bfseries}]{Class\ }{\ },
  moredelim=**[is][\beginlstdelim{Module\ }{\ }{\color{green!50!black}\bfseries}{\color{blue!60!black}\bfseries}]{Module\ }{\ },
  moredelim=**[is][\beginlstdelim{Record\ }{\ }{\color{green!50!black}\bfseries}{\color{blue!60!black}\bfseries}]{Record\ }{\ },
  morecomment=[n]{(*}{*)},%
  morestring=[d]",%
  literate={=>}{{$\Rightarrow$}}1
  {->}{{$\,\to\,$}}1
  {<-}{{$\leftarrow$}}1
  {>->}{{$\rightarrowtail$}}2
  {<->}{{$\leftrightarrow$}}1
  {forall}{{\color{blue!60!black}\bfseries$\forall$}}1
  {exists}{{\color{blue!60!black}\bfseries$\exists$}}1
  {<>}{{$\neq$}}1
  {<=}{{$\leq$}}1
  {>=}{{$\geq$}}1
  {:=}{{$\triangleq$}}1
  {\/\\}{{$\wedge$}}1
  {|-}{{$\vdash$}}1
  {\\\/}{{$\vee$}}1
  % {~}{{$\sim$}}1
  {'}{'}1
  {⟦}{{$\llbracket$}}1
  {⟧}{{$\rrbracket$}}1
  {-->}{{$\longrightarrow$}}1
  % {nat}{{\color{blue!60!black}\bfseries$\mathbb{N}$}}1
  % {bool}{{\color{blue!60!black}\bfseries$\mathbb{B}$}}1
  % {Qed.}{{\color{green!50!black}\bfseries$\blacksquare$}}1
  % {Defined.}{{\color{green!50!black}\bfseries$\square$}}1
  % {Proof.}{{\color{green!50!black}\bfseries$\because$}}1
}


% Syntax for LFSC listings
\lstdefinelanguage{lfsc}{
  language=lisp,
    alsoletter={!}{\%}{@}{\\},
  keywords={check,define,declare,program},
  classoffset=1,
  morekeywords={int,mpz,th_holds,holds,term,sort,type,match,fail,default},keywordstyle=\color{blue!60!black}\bfseries,
  classoffset=2,
  keywords={\%,@,!,\\},keywordstyle=\color{violet}\bfseries,
  classoffset=0,
  % mathescape=false,
  moredelim=**[is][\beginlstdelim{define\ }{\ }{\color{green!50!black}\bfseries}{\color{green!50!black!50!white}\bfseries}]{define\ }{\ },
  moredelim=**[is][\beginlstdelim{declare\ }{\ }{\color{green!50!black}\bfseries}{\color{blue!80!black!50!white}\bfseries}]{declare\ }{\ },
  moredelim=**[is][\beginlstdelim{program\ }{\ }{\color{green!50!black}\bfseries}{\color{blue!80!black!50!white}\bfseries}]{program\ }{\ },
  escapechar=\&
}

\lstdefinelanguage{smtcoq}{
  % language=lisp,
  alsoletter=\#0123456789\=,
  classoffset=0,
  keywords={or,and,not,impl,true,false,\=,->},keywordstyle=\color{black}\bfseries,
  classoffset=1,
  morekeywords={0,1,2,3,4,5,6,7,8,9,10,11,12,13,14,15,16,16,18,19,20},keywordstyle=\color{violet}\bfseries,
  classoffset=0,
  sensitive=true,
  % moredelim=**[is][\beginlstdelim{\#}{\ }{}{\color{blue!60!black}\bfseries}]{\#}{\ },
  % moredelim=**[is][\beginlstdelim{\#}{:}{\color{blue!60!black}\bfseries}{}]{\#}{:},
  moredelim=**[is][\beginlstdelim{:(}{\ }{}{\color{green!50!black}\bfseries}]{:(}{\ },
}

\lstdefinelanguage{ocaml}{
  language=[Objective]caml,
  identifierstyle=\ocidstyle
}

\newcommand*\ocidstyle{%
        \expandafter\id@style\the\lst@token\relax
}
\def\id@style#1#2\relax{%
        \ifcat#1\relax\else
                \ifnum`#1=\uccode`#1%
                        \color{blue!60!black}
                \fi
        \fi
}


\newenvironment{tcb}[2][\tiny]{%
  \tcblisting{enhanced jigsaw,breakable,lines before break=3,
    listing only,colback=bg,colframe=bg,enlarge
    top by=0mm,top=0pt,bottom=0pt,left=2pt,right=2pt,enhanced,
    before={\vspace{10pt}},
    after={\par\vspace{5pt}\noindent},
    listing options={language=#2,basicstyle={\ttfamily#1\upshape}}%
    }}{\endtcblisting}


\newenvironment{tcbfl}[2][\linewidth]{%
  \tcblisting{enhanced jigsaw,breakable,lines before break=3,%
    listing only,colback=bg,colframe=bg,enlarge
    top by=0mm,top=0pt,bottom=0pt,left=2pt,right=2pt,enhanced,%
    width=#1,%
    % before={\vspace{\baselineskip}},
    % after={\par\vspace{\baselineskip}\noindent},
    listing options={language=#2}%
    }}{\endtcblisting}


\definecolor{bgcolor}{HTML}{E0E0E0}

\lstset{language=Coq, breaklines=true}
\sloppy
\newcommand{\code}[1]{\colorbox{bg}{\lstinline!#1!}}

\newcommand{\lstin}[1]{\lstinline|#1|}
\captionsetup{labelfont=bf,font={footnotesize}}

\bibliographystyle{plainurl}% the mandatory bibstyle

\title{Formally Verified Liveness with Synchronous Multiparty Session Types in Rocq} %TODO Please add

\titlerunning{Dummy short title} %TODO optional, please use if title is longer than one line

\author{John Q. Public}{Dummy University Computing Laboratory, [optional: Address], Country \and My second affiliation, Country \and \url{http://www.myhomepage.edu} }{johnqpublic@dummyuni.org}{https://orcid.org/0000-0002-1825-0097}{(Optional) author-specific funding acknowledgements}%TODO mandatory, please use full name; only 1 author per \author macro; first two parameters are mandatory, other parameters can be empty. Please provide at least the name of the affiliation and the country. The full address is optional

\author{Joan R. Public\footnote{Optional footnote, e.g. to mark corresponding author}}{Department of Informatics, Dummy College, [optional: Address], Country}{joanrpublic@dummycollege.org}{[orcid]}{[funding]}

\authorrunning{J.\,Q. Public and J.\,R. Public} %TODO mandatory. First: Use abbreviated first/middle names. Second (only in severe cases): Use first author plus 'et al.'

\Copyright{John Q. Public and Joan R. Public} %TODO mandatory, please use full first names. LIPIcs license is "CC-BY";  http://creativecommons.org/licenses/by/3.0/

\ccsdesc[100]{\textcolor{red}{Replace ccsdesc macro with valid one}} %TODO mandatory: Please choose ACM 2012 classifications from https://dl.acm.org/ccs/ccs_flat.cfm 

\keywords{Dummy keyword} %TODO mandatory; please add comma-separated list of keywords

\category{} %optional, e.g. invited paper

\relatedversion{} %optional, e.g. full version hosted on arXiv, HAL, or other respository/website
%\relatedversiondetails[linktext={opt. text shown instead of the URL}, cite=DBLP:books/mk/GrayR93]{Classification (e.g. Full Version, Extended Version, Previous Version}{URL to related version} %linktext and cite are optional

%\supplement{}%optional, e.g. related research data, source code, ... hosted on a repository like zenodo, figshare, GitHub, ...
%\supplementdetails[linktext={opt. text shown instead of the URL}, cite=DBLP:books/mk/GrayR93, subcategory={Description, Subcategory}, swhid={Software Heritage Identifier}]{General Classification (e.g. Software, Dataset, Model, ...)}{URL to related version} %linktext, cite, and subcategory are optional

%\funding{(Optional) general funding statement \dots}%optional, to capture a funding statement, which applies to all authors. Please enter author specific funding statements as fifth argument of the \author macro.

\acknowledgements{I want to thank \dots}%optional

%\nolinenumbers %uncomment to disable line numbering

%\hideLIPIcs  %uncomment to remove references to LIPIcs series (logo, DOI, ...), e.g. when preparing a pre-final version to be uploaded to arXiv or another public repository

%Editor-only macros:: begin (do not touch as author)%%%%%%%%%%%%%%%%%%%%%%%%%%%%%%%%%%
\EventEditors{John Q. Open and Joan R. Access}
\EventNoEds{2}
\EventLongTitle{42nd Conference on Very Important Topics (CVIT 2016)}
\EventShortTitle{CVIT 2016}
\EventAcronym{CVIT}
\EventYear{2016}
\EventDate{December 24--27, 2016}
\EventLocation{Little Whinging, United Kingdom}
\EventLogo{}
\SeriesVolume{42}
\ArticleNo{23}
%%%%%%%%%%%%%%%%%%%%%%%%%%%%%%%%%%%%%%%%%%%%%%%%%%%%%%

\begin{document}

\maketitle

%TODO mandatory: add short abstract of the document
\begin{abstract}
Multiparty session types (MPST) offer a framework for the description of communication-based
protocols involving multiple participants. In the \textit{top-down} approach to MPST, 
the communication pattern of the session is described using a \textit{global type}. Then 
the global type is \textit{projected} on to a \textit{local type} for each participant,
and the individual processes making up the session are type-checked against these projections.
Typed sessions possess certain desirable properties such as \textit{safety}, \textit{deadlock-freedom} and 
\textit{liveness} (also called \textit{lock-freedom}).

In this work, we present the first mechanised proof of liveness
for synchronous multiparty session types in the Rocq Proof Assistant. 
Building on recent work, we represent global and local types
as coinductive trees using the paco library. We use a coinductively defined \textit{subtyping} relation 
on local types together with another coinductively defined \textit{plain-merge} projection
relation relating local and global types .
We then \textit{associate} collections of local types, or \textit{local type contexts}, with 
global types using this projection and subtyping relations, and prove an \textit{operational correspondence}
between a local type context and its associated global type. We then utilize this
association relation to prove the safety and liveness of associated local type contexts
and, consequently, the multiparty sessions typed by these contexts.  

Besides clarifying the often informal proofs of liveness found in the MPST literature,
our Rocq mechanisation also enables the certification of lock-freedom properties
of communication protocols. Our contribution amounts to around 12K lines of Rocq code.
\end{abstract}

\newcommand{\rocqlink}[1]{\href{#1}{\includegraphics[width=11pt]{img/icon-rocq-blue.png}}}

\section{Introduction}
\label{sec:introduction}
Multiparty session types \cite{honda2008} provide a type discipline for the correct-by-construction specification of message-passing protocols.
Desirable protocol properties guaranteed by session types include \textit{communication safety} 
(the labels and types of senders' payloads cohere with the capabilities of the receivers),
\textit{deadlock-freedom} (also called \textit{progress} or \textit{non-stuck property} \cite{srpaper}) 
(it is possible for the session to progress so long as it has at least one active participant), and 
\textit{liveness} (also called \textit{lock-freedom} \cite{fairnesslock} or \textit{starvation-freedom} \cite{castro2026synthetic})   
(if a process is waiting to send and receive then a communication involving it eventually happens).

There exists two common methodologies for multiparty session types. 
In the \textit{bottom-up} approach, the individual processes making up the session are typed using 
a collection of \textit{participants} and \textit{local types}, that is, a \textit{local type context}, and the properties of the session is examined by model-checking
this local type context. Contrastingly, in the \textit{top-down} approach sessions are typed by a \textit{global type}
that is related to the processes using endpoint \textit{projections} and \textit{subtyping}.
The structure of the global type ensures that the desired properties are satisfied by the session.
These two approaches have their advantages and disadvantages: the bottom-up approach is generally 
able to type more sessions, while type-checking and 
type-inferring in the top-down approach tend to be more efficient than model-checking the bottom-up system \cite{projsurvey}.  

In this work, we present the Rocq \cite{coq} formalisation of a 
synchronous MPST that that ensures the aforementioned properties
for typed sessions. Our type system uses an \textit{association} relation ($\assoc$) \cite{LessIsMoreRevisited, Pischke2026}
defined using (coinductive plain) projection \cite{tirore2023sound} and subtyping, in order to relate 
local type contexts and global types. This association relation ensures \textit{operational correspondence}
between the labelled transition system (LTS) semantics we define for local type contexts and global types. We then type ($\vdash_\mathcal{M}$) sessions using local type contexts that are associated with
global types, which ensure that the local type context, and hence the session, is well-behaved in some sense.
Whenever an associated local type context $\Gamma$ types a session $\M$, our type system guarantees 
safety (\cref{theo-type-safe}), deadlock-freedom \cref{theo-progress} and liveness \cref{theo-sess-live}.
To our knowledge, this work presents the first mechanisation of liveness for multiparty session types in
a proof assistant.



% https://tikzcd.yichuanshen.de/#N4Igdg9gJgpgziAXAbVABwnAlgFyxMJZABgBpiBdUkANwEMAbAVxiRAHEQBfU9TXfIRRkAjFVqMWbACrdeIDNjwEiZAEzj6zVohAAFOXyWCiI0mOpapugDo3gACjQB9LKWmuAlAAIAPt6xvOywwbwBJOy4AXjt2OgBbeLpDBX5lIWQ1c01JHRBYhKSAchTFARUUMw1LXLYAWVK0kxQs6oltepKeI3KMrMoajt12LvEYKABzeCJQADMAJwh4pAB2ahwIJAAWQesQOjg4CABjFIWl7fXNxAA2XbycAE80OHgEbpBz5cQyEA2kMztPZoRYAKzOi2+gP+iCyQIezxBJwhFx+VyQAGZ7mwDkdTh8vgD0YgMQTIUg4TCAKxk1FYv7XKnY3RPF5vFHfX4wtbwth2eZYCYACxwdHmiwA7hzMcSdrzbDYBcLReKIFLad8mQykHd5flFYKRWLJdwKFwgA
\begin{figure}
\begin{tikzcd}
\G \arrow[d, "\projf", dotted] \arrow[rd, "\assoc"] \arrow[rr, "\rightarrow",dotted] &   & \G' \arrow[d, "\assoc"]        \\
\T \arrow[d, "\vdash_{\mathsf{P}}", dotted] \arrow[r]                                  & {\{(p_i,T_i) \; |\; i \in I\}=\Gamma} \arrow[d, "\vdash_\M"] \arrow[r, "\rightarrow"] & \Gamma' \arrow[d, "\vdash_M"] \\
\mathsf{P} \arrow[r]                                                       & {\prod_{i}^{} \pp_i \triangleleft \mathsf{P}_i = \M} \arrow[r, "\rightarrow"]                                                     & \M'                          
\end{tikzcd}
\caption{Design overview. The dotted lines correspond to relations inherited from \cite{srpaper}
while the solid lines denote relations that are new, or substantially rewritten, in this paper.}
\end{figure}

Our Rocq implementation builds upon the recent formalisation of subject reduction for MPST by Ekici et. al. \cite{srpaper},
which itself is based on \cite{SynchronousSubtyping}.
The methodology in \cite{srpaper} takes an equirecursive approach where an inductive syntactic global or local type is identified 
with the coinductive tree obtained by fully unfolding the recursion. 
It then defines a coinductive projection relation between global and local type trees,
the LTS semantics for global type trees, and typing rules for the session calculus
outlined in \cite{SynchronousSubtyping}. 
We extensively use these definitions and the lemmas concerning them, but we still depart from and extend 
\cite{srpaper} in numerous ways by introducing local typing contexts, their correspondence with global types and 
a new typing relation. Our addition to the code amounts to around 12K lines of Rocq code.

As with \cite{srpaper}, our implementation heavily uses the parameterized coinduction technique 
of the paco \cite{paco} library. Namely, our liveness property is defined using 
possibly infinite \textit{execution traces} which we represent as coinductive streams.
The relevant predicates on these traces, such as fairness, are then defined as mixed inductive-coinductive 
predicates using linear temporal logic (LTL)\cite{pnueli1977temporal}.
This approach, together with the proof techniques provided by paco,
results in compositional and clear proofs.

\textbf{Outline.} In \cref{sec-procs} we define our session calculus and its LTS semantics. 
In \cref{sec:types} we recapitulate the definitions of local and global type trees, and the subtyping and projection relations on them, from \cite{srpaper}. 
In \cref{sec:lts} we give LTS
semantics to local type contexts and global types, and detail the association relation between them.
In \cref{sec-props} we define safety and liveness for local type contexts, and prove that they hold
for contexts associated with a global type tree. In \cref{sec-proc-props} we give the typing rules for 
our session calculus, and prove the desired properties of these typable sessions.
\section{The Session Calculus}
\label{sec-procs}
We introduce the simple synchronous session calculus that our type system will be used on. 
\newcommand{\PT}{\mathsf P}
\newcommand{\Xv}{\ensuremath{{\textbf{X}}}}
\newcommand{\ST}{{\mathsf S}}
\newcommand{\QT}{{\mathsf Q}}
\newcommand{\RT}{{\mathsf R}}

\subsection{Processes and Sessions}
\begin{definition}[Expressions and Processes]
  \label[definition]{def:processes}
  We define processes as follows: 
  \begin{align*}
    \PT ::= \prt{p}!\ell(\kf{e}).\PT \SEP \sum_{i \in I}
\prt{p}?\ell_i(x_i).\PT_i \SEP \cond{\kf{e}}{\PT}{\PT} \SEP \mu \Xv.\PT \SEP \Xv \SEP \textbf{0}
    \end{align*}
    where \kf{e} is an expression that can be a variable, a value such as \texttt{true}, $0$ or $-3$, 
    or a term built from expressions by applying the operators \texttt{succ}, \texttt{neg}, $\neg$, 
    non-deterministic choice $\sendsign$ and $>$. 
\end{definition}
$\prt{p}!\ell(\kf{e}).\PT$ is a process that sends the value of expression $\kf{e}$ 
with label $\ell$ to participant $\pp$, and continues with process $\PT$.
$\sum_{i \in I}
\prt{p}?\ell_i(x_i).P_i$ is a process that may receive a value from any 
$\ell_i \in I$, binding the result to $x_i$ and continuing with $\PT_i$, depending on which $\ell_i$
the value was received from. $\Xv$ is a recursion variable, $\mu \Xv.\PT$ is a recursive process,
$\cond{\kf{e}}{\PT}{\PT}$ is a conditional and $\textbf{0}$ is a terminated process.

Processes can be composed in parallel into sessions.
\begin{definition}[Multiparty Sessions]\label[definition]{def:sessions}
  Multiparty sessions are defined as follows.
    \[
    \M \;::=\;%
    {\prt{p}} \triangleleft \PT \quad \SEP \quad(\M  \mid \M)
    \quad \SEP \quad \mathcal{O}
  \]
\end{definition}
${\prt{p}} \triangleleft \PT$ denotes that participant $\pp$ is running the process $\PT$,
$\mid$ indicates parallel compositon. We write $\displaystyle  \prod_{i \in I}^{} \pp_i \triangleleft \PT_i$
to denote the session formed by $\pp_i$ running $\PT_i$ in parallel for all $i \in I$. 
$\mathcal{O}$
is an empty session with no participants, that is, the unit of parallel composition.
\begin{remark}
Note that $\mathcal{O}$ is different than
$\pp \triangleleft \textbf{0}$ as $\pp$ is a participant in the latter but not the former. 
This differs from previous work, e.g. in \cite{SynchronousSubtyping} the unit of 
parallel composition is 
$\pp \triangleleft \textbf{0}$ while in \cite{srpaper} there is no unit. 
The unitless appproach of \cite{srpaper} results in a lot of repetition in the code, 
for an example see their definition of \lstin{unfoldP} which contains two of every constructor:
one for when the session is composed of exactly two processes, and one for when it's composed of three or more.
Therefore we chose to add an unit element to parallel composition. However, we didn't make 
that unit $\pp \triangleleft \textbf{0}$ in order to reuse some of the lemmas from \cite{srpaper}
that use the fact that structural congruence preserves participants.
\end{remark} 
In Rocq processes and sessions are expressed in the following way
\begin{tcb}{Coq}
Inductive process : Type := 
  | p_send : part -> label -> expr -> process -> process
  | p_recv : part -> list(option process) -> process 
  | p_ite : expr -> process -> process -> process
  | p_rec : process -> process
  | p_var : nat -> process
  | p_inact : process.

Inductive session: Type :=
  | s_ind : part   -> process -> session
  | s_par : session -> session -> session
  | s_zero : session.
Notation "p '<--' P"   :=  (s_ind p P) (at level 50, no associativity).
Notation "s1 '|||' s2" :=  (s_par s1 s2) (at level 50, no associativity).
\end{tcb}
\subsection{Structural Congruence and Operational Semantics}
We define a structural congruence relation $\equiv$ on sessions which expresses the commutativity, associativity and 
unit of the parallel composition operator.

\begin{table}[h]
{\footnotesize
\[
\begin{array}{@{}l@{}}
  \inferrule[\rulename{sc-sym}]
  {}
  {\pp \triangleleft \PT  \mid \pq \triangleleft \QT  \equiv \pq \triangleleft \QT \mid \pp \triangleleft \PT} 
  \qquad
  \inferrule[\rulename{sc-assoc}]
  {}
  {(\pp \triangleleft \PT  \mid \pq \triangleleft \QT) \mid \pr \triangleleft \RT  
  \equiv \pp \triangleleft \PT  \mid (\pq \triangleleft \QT \mid \pr \triangleleft \RT)} 
  \\\\
  \inferrule[\rulename{sc-o}]
  {}
  {\pp \triangleleft \PT  \mid \mathcal{O}  
  \equiv \pp \triangleleft \PT } 
\end{array}
\]}
\caption{Structural Congruence over Sessions}
\label{tbl:scong}
\end{table}
We now give the operational semantics for sessions by the means of a labelled transition system.
We will be giving two types of semantics: one which contains silent $\tau$ transitions, and 
another, \textit{reactive} semantics \cite{fairnesslock} which doesn't contain explicit $\tau$ reductions while still considering $\beta$ reductions
up to silent actions. We will mostly be using the reactive semantics throughout this paper,
for the advantages of this approach see \cref{remark-reactive-justif}.
\subsubsection{Semantics With Silent Transitions}\label{subsec-tau}
We have two kinds of transitions, \textit{silent} ($\tau$) and 
\textit{observable} $(\beta)$. Correspondingly, we have two kinds of \textit{transition labels}, 
$\tau$ and $(\pp,\pq)\ell$ where $\pp, \pq$ are participants and $\ell$ is a message label.  
We omit the semantics of expressions, they are 
standard and can be found in \cite[Table 1]{SynchronousSubtyping}. We write $e \downarrow v$ when 
expression $e$ evaluates to value $v$. 
\newcommand{\taured}{\ensuremath{\xrightarrow{\tau}}}

\begin{table}[h]
{\footnotesize
\[
\begin{array}{@{}l@{}}
    \inferrule[\rulename{R-comm}]
  {j\in I \quad e \downarrow v}
  {\pp \triangleleft \sum^{}_{i\in I} \tin\pq{\ell_i}{x_i}.\PT_i \ \mid \  \pq \triangleleft \tout{\pp}{\ell_j}{\kf{e}}.\QT \ \mid \ \N \ \ 
  \xrightarrow{(\pp,\pq)\ell_j} \ \ 
    \pp \triangleleft \PT_j[v/x_j] \ \mid \ \pq \triangleleft \QT \ \mid \ \N}
    \\[10mm]
    \inferrule[\rulename{R-rec}]
  {}
  {\pp \triangleleft \mu \Xv.\PT \mid \N \ \ 
  \taured \ \
    \pp \triangleleft \PT[\mu \Xv.\PT / \Xv] \ \mid \ \N}
    \quad  
    \inferrule[\rulename{R-condt}]
  {e \downarrow \text{true}}
  {\pp \triangleleft \text{ if } e \text{ then } \PT \text{ else } \QT \ \mid \ \N \ \ 
  \taured \ \
    \pp \triangleleft \PT \ \mid \ \N}    
      \\[5mm]
        \inferrule[\rulename{R-condf}]
  {e \downarrow \text{false}}
  {\pp \triangleleft \text{ if } e \text{ then } \PT \text{ else } \QT \ \mid \ \N \ \ 
  \taured \ \
    \pp \triangleleft \QT \ \mid \ \N}  
      \quad
        \inferrule[\rulename{R-struct}]
  {\N_1' \equiv \N_1 \quad \N_1 \xrightarrow{\lambda} \N_2 \quad \N_2 \equiv \N_2'}
  {\N'_1 \xrightarrow{\lambda} \N'_2}  
\end{array}
\]}
\caption{Operational Semantics of Sessions}
\label{tbl:srr}
\end{table}
In \cref{tbl:srr}, \rulename{R-comm} describes a synchronous communication from $\pp$ to $\pq$ via 
message label $\ell_j$.
\rulename{R-rec} unfolds recursion, \rulename{R-condt} and \rulename{R-condf} express how to evaluate conditionals,
and \rulename{R-struct} shows that the reduction respects the structural pre-congruence.
We write $\M \rightarrow \N$ if $\M \xrightarrow{\lambda} \N$ for some transition label $\lambda$.
We write $\rightarrow^*$ to denote the reflexive transitive closure of $\rightarrow$. We also write 
$\M \Rrightarrow \N$ when $\M \equiv \N$ or $\M \rightarrow^* \N$ where all the transitions
involved in the multistep reduction are $\tau$ transitions.
\subsection{Reactive Semantics}
In reactive semantics $\tau$ transitions are captured by an \textit{unfolding} relation $(\Rrightarrow)$,
and $\beta$ reductions are defined up to this unfolding.
\begin{table}[h]
{\footnotesize
\[
\begin{array}{@{}l@{}}
    \inferrule[\rulename{Unf-struct}]
  {\M \equiv \N}
  {\M \Rrightarrow \N}
  \quad
  \inferrule[\rulename{Unf-rec}]
  {}
  {\pp \triangleleft \mu \Xv.\PT \mid \N \ \ 
  \Rrightarrow
    \pp \triangleleft \PT[\mu \Xv.\PT / \Xv] \ \mid \ \N}  
  \quad
    \inferrule[\rulename{Unf-condt}]
  {e \downarrow \text{true}}
  {\pp \triangleleft \text{ if } e \text{ then } \PT \text{ else } \QT \ \mid \ \N \ \ 
  \Rrightarrow \ \
    \pp \triangleleft \PT \ \mid \ \N}    
      \\[5mm]
        \inferrule[\rulename{Unf-condf}]
  {e \downarrow \text{false}}
  {\pp \triangleleft \text{ if } e \text{ then } \PT \text{ else } \QT \ \mid \ \N \ \ 
  \Rrightarrow \ \
    \pp \triangleleft \QT \ \mid \ \N}  
  \quad
    \inferrule[\rulename{Unf-trans}]
  {\M \Rrightarrow \M' \quad \M' \Rrightarrow \N }
  {\M \Rrightarrow \N}  
\end{array}
\]}
\caption{Unfolding of Sessions}
\label{tbl:unf}
\end{table}

$\M \Rrightarrow \N$ means that $\M$ can transition to $\N$ through some internal actions, or $\tau$ transitions in the semantics of 
\cref{subsec-tau}. We say that $\M$ \textit{unfolds} to $\N$. 
In Rocq it's captured by the predicate \lstin{unfoldP : session -> session -> Prop}.
\begin{table}[h]
{\footnotesize
\[
\begin{array}{@{}l@{}}
    \inferrule[\rulename{R-comm}]
  {j\in I \quad e \downarrow v}
  {\pp \triangleleft \sum^{}_{i\in I} \tin\pq{\ell_i}{x_i}.\PT_i \ \mid \  \pq \triangleleft \tout{\pp}{\ell_j}{\kf{e}}.\QT \ \mid \ \N \ \ 
  \xrightarrow{(\pp,\pq)\ell_j} \ \ 
    \pp \triangleleft \PT_j[v/x_j] \ \mid \ \pq \triangleleft \QT \ \mid \ \N}
    \\[10mm]
    \inferrule[\rulename{R-unfold}]
  {\M \Rrightarrow \M' \quad \M' \lts{\lambda} \N' \quad \N' \Rrightarrow \N}
  {\M \lts{\lambda} \N}
    \\[10mm]
\end{array}
\]}
\caption{Reactive Semantics of Sessions}
\label{tbl:srr-react}
\end{table}

\rulename{R-comm} captures communications between processes, and \rulename{R-unfold} lets us consider reductions up to
unfoldings. In Rocq, \lstin{betaP_lbl M lambda M'} denotes 
$\M \lts{\lambda} \M'$. We write $\M \lts{} \M'$ if $\M \lts{\lambda} \M'$
for some $\lambda$, which is written \lstin{betaP M M'} in Rocq.
We write $\lts{}^*$ to denote the reflexive transitive closure of
$\lts{}$, which is called \lstin{betaRtc} in Rocq. 
\section{The Type System}
We briefly recap the core definitions of local and global type trees, 
subtyping and projection from
\cite{SynchronousSubtyping}. We take an equirecursive approach and 
work directly on the possibly infinite local and global type trees 
obtained by unfolding the recursion
in guarded syntactic types, details of this approach
can be found in \cite{srpaper} and hence are omitted here.  
\label{sec:types}
\subsection{Local Type Trees}
We start by defining the sorts that will be used to type expressions, 
and local types that will be used to type single processes.
\begin{definition}[Sorts and Local Type Trees]
We define three atomic sorts: $\tint$, $\tbool$ and $\tnat$.
Local type trees are then defined coinductively with the following syntax: 
\begin{minipage}{0.45\textwidth}
\begin{align*}
    \T ::= \quad &\tend \\
     \SEP &\procinset{\prt{p}}{\ell_i(\S_i)}{\T_i}{i \in I} \\
     \SEP &\procoutset{\prt{p}}{\ell_i(\S_i)}{\T_i}{i \in I}
    \end{align*}  
\end{minipage}
\begin{minipage}{0.45\textwidth}
\begin{tcb}{Rocq}
Inductive sort: Type :=
   | sbool: sort | sint : sort | snat : sort.
CoInductive ltt: Type :=
  | ltt_end : ltt
  | ltt_recv: part -> list (option(sort*ltt)) -> ltt
  | ltt_send: part -> list (option(sort*ltt)) -> ltt.
\end{tcb}  
\end{minipage}
\end{definition}
In the above definition, $\tend$ represents a role that has finished communicating.
$\ltsend{\prt{p}}{\ell_i(\S_i).\T}_{i \in I}$ denotes a role that may, from any $i \in I$, 
receive a value of sort $S_i$ with message label $\ell_i$ and continue with $\tlt_i$.
Similarly, $\ltrec{\prt{p}}{\ell_i(\S_i).\T_i}_{i \in I}$ represents a role that may choose
to send a value of sort $S_i$ with message label $\ell_i$ and continue with $\T_i$ for any $i \in I$.

In Rocq we represent the continuations using a \lstin{list} of \lstin{option} types. 
In a continuation \lstin{gcs : list (option(sort*ltt))}, index \lstin{k} (using zero-indexing) being equal to 
\lstin{Some (s_k, T_k)} means that $\ell_k (S_k).\T_k$ is available in the continuation.
Similarly index \lstin{k} being equal to \lstin{None} or being out of bounds of the list means 
that the message label $\ell_k$ is not present in the continuation. The function \lstin{onth} \rocqlink{todo} 
formalises this convention
in Rocq.
\begin{remark}
  Note that Rocq allows us to create types such as \lstin{ltt_send q []} which 
  don't correspond to well-formed local types as the continuation is empty.
  In our implementation we define a predicate \lstin{wfltt : ltt -> Prop} capturing that 
  all the continuations in the local type tree are non-empty. Henceforth we assume that all local types
  we mention satisfy this property.
\end{remark}
\subsection{Subtyping}
We define the subsorting relation on sorts and the subtyping relation on local type trees.
\begin{definition}[Subsorting and Subtyping] \label[definition]{def:subtyping}
  Subsorting $\subso$ is the least reflexive binary relation that satisfies $\tnat \subso \tint$. 
  Subtyping $\subtp$ is the largest relation between local type trees coinductively defined by the following rules:
  \[
  \begin{array}[t]{@{}c@{}}
    \cinferrule[]{
    }{
    \tend \subtp \tend
    }
    \enspace \rulesubend
    \quad
  \cinferrule[]{
    \forall i \in I: \qquad S'_i \subso S_i \qquad \T_i \subtp \T'_i
  }
  {
    \procinset{p}{\ell_i(S_i)}{\T_i}{i \in I \cup J} \subtp 
    \procinset{p}{\ell_i(S'_i)}{\T'_i}{i \in I}
  }
  \enspace \rulesubin
  \\\\
  \cinferrule[]{
    \forall i \in I: \qquad S_i \subso S'_i \qquad \T_i \subtp \T'_i
  }
  {
    \procoutset{p}{\ell_i(S_i)}{\T_i}{i \in I}
    \subtp
    \procoutset{p}{\ell_i(S'_i)}{\T'_i}{i \in I \cup J}
    }
  \enspace \rulesubout
  \end{array}
  \]
\end{definition}
Intutively, $\T_1 \subtp \T_2$ means that a role of type $\T_1$ can be supplied anywhere 
a role of type $\T_2$ is needed.
$\rulesubin$ captures the fact that we can supply a role that is able to receive 
more labels than specified, and $\rulesubout$ captures that we can supply a role
that has fewer labels available to send. Note the contraviance of the sorts in $\rulesubin$,
if the supertype demands the ability to receive an $\tnat$ then the subtype can 
receive $\tnat$ or $\tint$.

In Rocq, the subtyping relation \lstin{subtypeC : ltt -> ltt -> Prop} is expressed as a greatest fixpoint using the 
\lstin{Paco} library \cite{paco}, 
for details of we refer to \cite{SynchronousSubtyping}.
\subsection{Global Type Trees}
We now define global types which give a bird's eye view of the whole protocol.
As before, we work directly on infinite trees and omit the details which can be found in 
\cite{srpaper}.
$\tend$ denotes a protocol that has ended, $\GvtPair{\prt{p}}{\prt{q}}{\ell_i(\S_i).\mathbb{G}_i}_{i\in I}$
denotes a protocol where for any $i \in I$, participant $\pp$ may send a value of sort 
$S_i$ to another participant $\pq$ via message label $\ell_i$, after which the protocol continues as $\mathbb{G}_i$.
\begin{definition}[Global type trees]
  \label[definiton]{def:global_type_trees}
  We define global type trees coinductively as follows: 
  \begin{minipage}{0.47\textwidth}
  \begin{align*}
\G &::= \quad \tend 
     \SEP \GvtPair{\prt{p}}{\prt{q}}{\ell_i(\S_i).\G_i}_{i\in I}
    \end{align*}  
  \end{minipage}
    \begin{minipage}{0.50\textwidth}
    \begin{tcb}{Rocq}
    CoInductive gtt: Type :=
    | gtt_end    : gtt 
    | gtt_send   : part -> part -> list (option (sort*gtt)) -> gtt.
    \end{tcb}  
    \end{minipage}    
\end{definition}
We further define the function $\funprt(\G)$ that denotes the 
participants of the global type $\G$ as the least solution 
\footnote{Here we adopt a simplified presentation as $\funprt(\G)$ 
is actually defined by extending it from an inductively defined function on 
syntactic types, we refer to \cite{srpaper} for details.} to the following equations:
\begin{align*}
  &\funprt(\tend)=\emptyset \quad
  &\funprt(\GvtPair{\prt{p}}{\prt{q}}{\ell_i(\S_i).\G_i}_{i\in I})=
      \{\prt{p},\prt{q}\} \cup \bigcup _{i \in I} {\funprt(\G_i)}
\end{align*}
We extend the function $\funprt$ onto trees by defining $\funprt(\G)=\funprt(\mathbb{G})$
where the global type $\mathbb{G}$ corresponds to the global type tree $\G$. Technical details 
of this definition such as well-definedness can be found in \cite{srpaper,SynchronousSubtyping}. 

In Rocq $\funprt$ is captured with the 
predicate \lstin{isgPartsC : part -> gtt -> Prop}, where \lstin{isgPartsC p G} 
denotes $\prt{p} \in \funprt(\G)$.  


\subsection{Projection}
We now define coinductive projections with plain merging 
(see \cite{projsurvey} for a survey of other notions of merge). 
\begin{definition}[Projection] \label[definition]{def:projections}
    The projection of a global type tree onto a participant ${\prt{r}}$ is the largest relation $\upharpoonright_{\prt{r}}$ between global type trees and
    local type trees such that, whenever $\G\proj{r}\T$:
    \begin{itemize}
      \item ${\prt{r}} \notin \participant{\G}$ implies $\T = \tend$; \hfill {\ruleprojend}
      \item $\G = \GvtPair{\prt{p}}{\prt{r}}{\ell_i(S_i).\G_i}_{i \in I}$ implies 
      $\T = \procinset{\prt{p}}{\ell_i(\S_i)}{\T_i}{i \in I}$ and $\forall i \in I, \G\proj{r}\T_i $ \hfill {\ruleprojin}
      \item $\G = \GvtPair{\prt{r}}{\prt{q}}{\ell_i(S_i).\G_i}_{i \in I}$ implies 
      $\T = \procoutset{\prt{q}}{\ell_i(\S_i)}{\T_i}{i \in I}$ and $\forall i \in I, \G\proj{r}\T_i $ \hfill {\ruleprojout}
      \item $\G = \GvtPair{\prt{p}}{\prt{q}}{\ell_i(S_i).\G_i}_{i \in I} \text{and } {\prt{r}} \notin \{{\prt{p}},{\prt{q}}\}$ implies 
      that $\forall i \in I, \G_i\proj{r}\T $ \hfill {\ruleprojcont}
    \end{itemize}
\end{definition}

Informally, the projection of a global type tree $\G$ onto a participant $\prt{r}$
extracts a role for participant  $\prt{r}$ from the protocol whose bird's-eye view 
is given by $\G$. \ruleprojend expresses that if $\prt{r}$ is not a participant of $\G$ then
$\prt{r}$ does nothing in the protocol. \ruleprojin and \ruleprojout handle the cases where $\prt{r}$
is involved in a communication in the root of $\G$. $\ruleprojcont$ says that, if $\prt{r}$ is not
involved in the root communication of $\G$ and all continuations of $\G$ project on to the same type,
then $\G$ also projects on to that type.
In Rocq, projection is defined as a \lstin{Paco} greatest fixpoint as the relation
\lstin{projectionC : gtt -> part -> ltt -> Prop}.

We further have the following fact about projections that lets us regard it as a partial
function:
\begin{lemma}[\cite{srpaper}] \label[lemma]{lem-proj-func} If \lstin{projectionC G p T} and 
\lstin{projectionC G p T'} then \lstin{T = T'}.  
\end{lemma}
We write $\projfn{\G}{\pr}=\T$ when $\G\proj{\pr}\T$. Furthermore we will be frequently
be making assertions about subtypes of projections of a global type e.g.
$\T \subtp \projfn{\G}{\pr}$. In our Rocq implementation we define the predicate 
\lstin{issubProj : ltt -> gtt -> part -> Prop}
as a shorthand for this.
\subsection{Balancedness, Global Tree Contexts and Grafting}
We introduce an important constraint on the types of global type trees we will consider, balancedness.

\begin{definition}[Balanced Global Type Trees]\label[definition]{def-balance}
A global tree $\G$ is balanced if for any subtree $\G'$ of $\G$, there exists $k$ such that
for all $\pp \in \funprt(\G')$, $\pp$ occurs on every path from the root of $\G'$ of length at 
least $k$.
\end{definition}
We omit the technical details of this definition and the Rocq implementation, they can be found in 
\cite{SynchronousSubtyping} and \cite{srpaper}.

Balancedness is a regularity condition that imposes a notion of \textit{liveness} on the protocol
described by the global type tree. Indeed, our liveness results in \cref{sec-proc-props} 
hold only for balanced global types. Another reason for formulating balancedness is that it allows 
us to use the "grafting" technique, turning proofs by coinduction on infinite trees to proofs by induction
on finite global type tree contexts.

\begin{definition}[Global Type Tree Contexts and Grafting]\label{def:global-ctx}
  Global type tree contexts are defined inductively with the following syntax:

  \begin{minipage}{0.4\textwidth}
  \begin{align*}
    \Gcx &::= \quad \GvtPair{p}{q}{\ell_i(S_i).\Gcx_i}_{i \in I} 
   \SEP \hole_i
  \end{align*}  
  \end{minipage}
  \begin{minipage}{0.5\textwidth}
  \begin{tcb}{Rocq}
Inductive gtth: Type :=
  | gtth_hol    : fin -> gtth
  | gtth_send   : part -> part -> list (option (sort * gtth)) -> gtth.
  \end{tcb} 
  \end{minipage}
   Given a global type tree context $\Gcx$ whose holes are in the indexing set $I$
  and a set of global types $\{\G_i\}_{i \in I}$,  the grafting $\Gcx[\G_i]_{i \in I}$ denotes 
  the global type tree obtained by substituting $\hole_i$ with $\G_i$ in $\Gcx$.

  In Rocq the indexed set $\{\G_i\}_{i \in I}$ is represented using a \lstin{list (option gtt)}.
  Grafting is expressed with the  inductive relation 
  \lstin{typ_gtth : list (option gtt) -> gtth -> gtt -> Prop}.
  \lstin{typ_gtth gs gcx gt} means that the grafting of the set of global type trees 
  \lstin{gs} onto the context \lstin{gcx} results in the tree \lstin{gt}.
  We additionally define $\funprt$ and \lstin{ishParts} on global type tree contexts analogously 
  to $\funprt$ and \lstin{isgPartsC} on trees. 
\end{definition}
A global type tree context can be thought of as the finite prefix of a global type 
tree, where holes $\hole_i$ indicate the cutoff points. Global type tree contexts
are related to global type trees with the \textit{grafting} 
operation that fills in the holes with type trees.
The following lemma relates global type tree contexts to balanced
global type trees. In particular, it allows us to turn proofs by coinduction on infinite trees 
to proofs by induction on the grafting context.
\begin{lemma}[Proper Grafting Lemma, \cite{srpaper}]\label[lemma]{lem-grafting}
If \lstin{G} is a balanced global type tree and \lstin{isgPartsC p G}, then there is 
a global type tree context \lstin{Gctx} and an option list of global type trees 
\lstin{gs} such that \lstin{typ_gtth gs Gctx G}, \lstin{\~ ishParts p Gctx} and every \lstin{Some} element of \lstin{gs}
is of shape \lstin{gtt_end}, \lstin{gtt_send p q} or \lstin{gtt_send q p}. 
\end{lemma}
If \lstin{typ_gtth gs Gctx G}, \lstin{\~ ishParts p Gctx} and every \lstin{Some} element of \lstin{gs}
is of shape \lstin{gtt_end}, \lstin{gtt_send p q} or \lstin{gtt_send q p}, then we call 
the pair \lstin{gs} and \lstin{Gctx} as the $\pp$-grafting of \lstin{G}, expressed in Rocq as
\lstin{typ_p_gtth gs Gctx p G}. When we don't care about the contents of \lstin{gs} we may just say
that \lstin{G} is \lstin{p}-grafted by \lstin{Gctx}.
\begin{remark}
  From now on, all the global type trees we will be referring to are assumed to be balanced.
  When talking about the Rocq implementation, any \lstin{G : gtt} we mention is assumed
  to satisfy the predicate \lstin{wfgC G}, expressing that \lstin{G} corresponds to some global type
  and that \lstin{G} is balanced. 
  Furthermore, we will often require that a global type is projectable onto all its participants. 
  This is captured by
  the predicate \lstin{projectableA G = forall p, exists T, projectionC G p T}. As with \lstin{wfgC}, 
  we will be assuming that all types we mention are projectable. 
\end{remark}
\section{Semantics of Types}
\label{sec:lts}
In this section we introduce local type contexts, and define Labelled Transition System 
semantics on these constructs.
\subsection{Local Type Contexts and Reductions} 
We start by defining typing contexts as finite mappings of participants to local type trees.
\begin{minipage}{0.45\textwidth}
\begin{definition}[Typing Contexts]\label[definition]{def-type-ctx}
  \[
  \Gamma \;::=\;%
  \emptyset \SEP \Gamma, {\prt{p}}:\T
\]
\end{definition}
\end{minipage}
\begin{minipage}{0.5\textwidth}
  \begin{tcb}{Rocq}
Module M := MMaps.RBT.Make(Nat).
Module MF := MMaps.Facts.Properties Nat M.
Definition tctx: Type := M.t ltt.
\end{tcb}
\end{minipage}

Intuitively, p : {\T} means that participant p is associated with a process that has the type tree T. 
We write $\dom{\Gamma}$ to denote the set of participants occuring in $\Gamma$. We write $\Gamma(\pp)$ for the type of $\pp$ in $\Gamma$. 
We define the composition $\Gamma_1,\Gamma_2$ iff $\dom{\Gamma_1} \cap \dom{\Gamma_2}=\emptyset$.

In the Rocq implementation we implement local typing contexts as finite maps of 
participants, which are represented as natural numbers, and local type trees.
We use the red-black tree based finite map implementation of the MMaps library \cite{mmaps}.

\begin{remark}
From now on, we assume the all the types in the local type contexts always have non-empty continuations.
In Rocq terms, if \lstin{T} is in context \lstin{gamma} then \lstin{wfltt T} holds. This is expressed by the predicate
\lstin{tctx_wf: tctx -> Prop}.
\end{remark}

We now give LTS semantics to local typing contexts, for which we first define the transition labels.
\begin{definition}[Transition labels] A transition label $\alpha$ has the following form:
  \begin{align*}
    \alpha ::&= \lblrec{p}{q}{\ell(S)}  && 
    \text{(${\prt{p}}$ receives a value of sort ${S}$ from ${\prt{q}}$ with message label ${\ell}$)}\\
    & \SEP \lblsend{p}{q}{\ell(S)} && \text{(${\prt{p}}$ sends a value of sort $S$ to ${\prt{q}}$ with message label $\ell$ )}\\
    & \SEP \lblsync{p}{q}{\ell} && \text{(A synchronised communication from $\pp$ to $\pq$ occurs via label $\ell$)}\\
  \end{align*}
\end{definition}
Next we define labelled transitions for local type contexts.
\begin{definition}[Typing context reductions] \label[definition]{def-ctx-red}
The typing context transition $\lts{\alpha}$ is defined inductively by the following rules:
  \[
\begin{array}[t]{@{}c@{}}
\inferrule[]{
     k \in I }{
      \lbltrans
    {\prt{p} : \procinset{\prt{q}}{\ell_i(\S_i)}{\T_i}{i \in I}} 
    {\lblrec{p}{q}{\ell_k(S_k)}}
    {\prt{p} : \T_k}
    }
    \; \rulename{ $\Gamma$-\andsign}
    \;
  \inferrule[]{
     k \in I }{
      \lbltrans
      {\prt{p} : \procoutset{\prt{q}}{\ell_i(\S_i)}{\T_i}{i \in I}} 
      {\lblsend{p}{q}{\ell_k(S_k)}}
      {\prt{p} : \T_k}
    }
    \; \rulename{ $\Gamma$-$\oplus$}
  \\\\
    \inferrule[]{
  \lbltrans{\Gamma}{\alpha}{\Gamma'}}
  {
    \lbltrans{\Gamma, \prt{p} : \T}
     {\alpha} {\Gamma', {\prt{p}} : \T} 
  }
  \enspace \rulename{$\Gamma$-,}
  
  \qquad
    \inferrule[]{
    \lbltrans{\Gamma_1}{\lblsend{\prt{p}}{\prt{q}}{\ell(S)}}{\Gamma'_1}
    \qquad  
    \lbltrans{\Gamma_2}{\lblrec{\prt{q}}{\prt{p}}{\ell(S')}}{\Gamma'_2}
    \qquad S \subso \S'
    }{
    \lbltrans{\Gamma_1, \Gamma_2 }
    {\lblsync{p}{q}{\ell}}
    {\Gamma'_1, \Gamma'_2}
    }
    \enspace \rulename{$\Gamma$-$\oplus$\andsign}
\end{array}
\]
We write $\Gamma \lts{\alpha}$ if there exists $\Gamma'$ such that $\Gamma\lts{a}\Gamma'$.  
We define a reduction $\Gamma\lts{}\Gamma'$ that holds iff \; $\lbltrans{\Gamma}{\lblsync{\prt{p}}{\prt{q}}{\ell}}{\Gamma'}$ for some ${\prt{p}}$, ${\prt{q}}$, $\ell$. We write $\Gamma\lts{}$ iff \; $\Gamma\lts{}\Gamma'$ for some $\Gamma'$. 
We write $\lts{}^*$ for the reflexive transitive closure of $\lts{}$.
\end{definition}
$\ruleredsend$ and $\ruleredrec$, express a single participant sending or receiving.
$\ruleredsync$ expresses a synchronised communication where one participant sends while another receives,
and they both progress with their continuation. $\ruleredvar$ shows how to extend a context.  
In Rocq typing context reductions are defined with the predicate \lstin{tctxR}.

\begin{minipage}{0.45\textwidth}
  \begin{tcb}{Rocq}
Notation opt_lbl := nat.
Inductive label: Type :=
  | lrecv: part -> part -> option sort -> opt_lbl -> label
  | lsend: part -> part -> option sort -> opt_lbl -> label
  | lcomm: part -> part -> opt_lbl -> label.
\end{tcb}
\end{minipage}
\begin{minipage}{0.45\textwidth}
\begin{tcb}{Rocq}
Inductive tctxR: tctx -> label -> tctx -> Prop :=
  | Rsend: ...
  | Rrecv: ...  
  | Rcomm: ...
  | RvarI: ...
  | Rstruct: forall g1 g1' g2 g2' l, tctxR g1' l g2' ->
    M.Equal g1 g1' -> M.Equal g2 g2' -> tctxR g1 l g2.
\end{tcb}
\end{minipage}

The first four constructors in the definition of \lstin{tctxR} corresponds to the 
rules in \cref{def-ctx-red}, and \lstin{Rstruct} expresses the indistinguishability
of local contexts under the \lstin{M.Equal} predicate from the MMaps library. 

We illustrate typing context reductions with an example.
\begin{example}\label{exam:reductions}
  Let $\Gamma = \{{\prt{p}}:\T_{\prt{p}}, \; {\prt{q}} : \T_{\prt{q}},\; {\prt{r}}: \T_{\prt{r}}\}$ where
  $\T_{\prt{p}} = \procoutmult{\prt{q}}{\ell_0(\tint).\T_{\prt{p}},\ell_1(\tint).\tend}$
  $\T_{\prt{q}} = \procinmult{\prt{p}}{\ell_0(\tint).\T_{\prt{q}},  \ell_1(\tint).\procoutsingle{\prt{r}}{\ell_2(\tint)}{\tend}}$
  and 
  $\T_{\prt{r}} = \procinmult{\prt{q}}{\ell_2(\tint).\tend}$.
We have the reductions $\Gamma\lts{\lblsend{p}{q}{\ell_0(\tint)}} \Gamma$ and
$\Gamma \lts{\lblrec{\prt{q}}{\prt{p}}{\ell_0(\tint)}} \Gamma$, which synchronise to give the reduction
and $\Gamma \lts{\lblsync{\prt{p}}{\prt{q}}{\ell_0}}  \Gamma$. Similarly via synchronised communication of $\pp$ and $\pq$
via message label $\ell_1$ we get $\Gamma \lts{\lblsync{\prt{p}}{\prt{q}}{\ell_1}} \Gamma'$
where $\Gamma'$ is defined as  $\{{\prt{p}} : \tend, \; {\prt{q}}: 
\procoutsingle{\prt{r}}{\ell_2(\tint)}{\tend}, {\prt{r}} : \T_{\prt{r}}\}$.
We further have that $\Gamma' \lts{\lblsync{q}{r}{\ell_2}} \Gamma_\mathtt{end}$ where
$\Gamma_{\mathtt{end}}$ is defined as 
$\{{\prt{p}}:\tend, \; {\prt{q}} : \tend,\; {\prt{r}}: \tend \}$.

In Rocq, $\Gamma$ is defined the following way \rocqlink{todo}:
\begin{tcb}{Rocq}
Definition prt_p:=0.
Definition prt_q:=1.
Definition prt_r:=2.
CoFixpoint T_p := ltt_send prt_q [Some (sint,T_p); Some (sint,ltt_end); None].
CoFixpoint T_q := ltt_recv prt_p [Some (sint,T_q); Some (sint, ltt_send prt_r [None;None;Some (sint,ltt_end)]); None].
Definition T_r := ltt_recv prt_q [None;None; Some (sint,ltt_end)].
Definition gamma := M.add prt_p T_p (M.add prt_q T_q (M.add prt_r T_r M.empty)).
\end{tcb}
Now $\Gamma \lts{\lblsync{\prt{p}}{\prt{q}}{\ell_0}}  \Gamma$  can be expressed as 
\lstin{tctxR gamma (lsend prt_p prt_q (Some sint) 0) gamma}.
\end{example}

\subsection{Global Type Reductions}
As with local typing contexts, we can also define reductions for global types.
\begin{definition}[Global type reductions]
  The global type transition $\lts{\alpha}$ is defined coinductively as follows.
  \[
\begin{array}[t]{@{}c@{}}
\cinferrule[]{
     k \in I }{
      \lbltrans
    {\GvtPair{p}{q}{\ell_i(S_i).\G_i}_{i \in I}} 
    {\lblsync{p}{q}{\ell_k}}
    {\G_k}
    }
    \enspace \rulename{GR-$ \sendsign \andsign$}
    \\\\
    \cinferrule[]{
     \forall i \in I \enspace \lbltrans{\G_i}{\alpha}{\G'_i} \qquad
     \subject{\alpha} \cap \{\prt{p},\prt{q}\} = \emptyset
     \qquad \forall i \in I \enspace \{\prt{p},\prt{q}\} \subseteq \participant{\G_i}
     }{
      \lbltrans
    {{\GvtPair{p}{q}{\ell_i(S_i).\G_i}_{i \in I}}} 
    {\alpha}
    {{\GvtPair{p}{q}{\ell_i(S_i).\G'_i}_{i \in I}}}
    }
    \enspace \rulename{GR-Ctx}
\end{array}
\]
\end{definition}
\rulegredsendrec says that a global type tree with root $\pp \rightarrow \pq$ can
transition to any of its children corresponding to the message label choosen by $\pp$. 
\rulegredctx says that if the subjects of $\alpha$ are disjoint from the root and all its children
can transition via $\alpha$, then the whole tree can also transition via $\alpha$, with the root remaining 
the same and just the subtrees of its children transitioning.
In Rocq global type reductions are expressed using the coinductively defined predicate \lstin{gttstepC}. 
For example, $\lbltrans{\G}{\lblsync{p}{q}{\ell_k}}{\G'}$ translates to \lstin{gttstepC G G' p q k}. 
We refer to \cite{srpaper} for details.

\subsection{Association Between Local Type Contexts and Global Types}
We have defined local type contexts which specifies protocols bottom-up 
by directly describing the roles of every participant,
and global types, which give a top-down view of the whole protocol, and the transition relations on them.
We now relate these local and global definitions by defining \textit{association} between local type
context and global types.
\begin{definition}[Association \rocqlink{todo}]\label[definition]{def-assoc}
A local typing context $\Gamma$ is associated with a global type tree $\G$, written $\Gamma \assoc \G$,
if the following hold:
\begin{itemize}
  \item For all $\pp \in \funprt(\G)$, $\pp \in \dom{\Gamma}$ and $\Gamma(\pp) \subtp \G \projf{\pp}$.
  \item For all $\pp \notin \funprt(\G)$, either $\pp \notin \dom{\Gamma}$ or $\Gamma(\pp)=\tend$. 
\end{itemize}
In Rocq this is defined with the following:
\begin{tcb}{Rocq}
Definition assoc (g: tctx) (gt:gtt) := 
    forall p, (isgPartsC p gt -> exists Tp, M.find p g=Some Tp /\  
        issubProj Tp gt p) /\
         (~ isgPartsC p gt -> forall Tpx, M.find p g = Some Tpx -> Tpx=ltt_end).
\end{tcb}
\end{definition} 
Informally, $\Gamma \assoc \G$ says that the local type trees in $\Gamma$ 
obey the specification described by the global type tree $\G$. 
\begin{example}\label{exam:assoc}
  In Example \ref{exam:reductions},
  we have that $\Gamma \assoc \G$ where
  $\G := \GvtPair{\prt{p}}{\prt{q}}{\ell_0(\tint).\G,\ell_1(\tint).\GvtPair{\prt{q}}{\prt{r}}{\ell_2(\tint).\tend}}$.
  In fact, we have $\Gamma(\ps) = \projfn{\G}{\ps}$ for $\ps \in \{\prt{p},\pq,\pr\}$. 
  Similarly, we have $\Gamma' \assoc \G'$ where
    $\G' := \GvtPair{\prt{q}}{\prt{r}}{\ell_2(\tint).\tend}$
\end{example}

It is desirable to have the association
be preserved under local type context and global type reductions, that is, 
when one of the associated constructs "takes a step" so should the other. We formalise this property with 
soundness and completeness theorems.

\begin{theorem}[Soundness of Association \rocqlink{todo}]\label{theo-soundness} 
  If \lstin{assoc gamma G} and \lstin{gttstepC G G' p q ell}, then there is a local type context \lstin{gamma'},
  a global type tree \lstin{G''} and a message label \lstin{ell'} such that 
  \lstin{gttStepC G G'' p q ell'}, \lstin{assoc gamma' G''} and 
  \lstin{tctxR gamma (lcomm p q ell') gamma'}.
\end{theorem}
\begin{theorem}[Completeness of Association \rocqlink{todo}] \label{theo-completeness}
  If \lstin{assoc gamma G} and \lstin{tctxR gamma (lcomm p q ell) gamma'}, 
  then there exists a global type tree \lstin{G'} such that 
  \lstin{assoc gamma' G'} and \lstin{gttstepC G G' p q ell}. 
\end{theorem}
\begin{remark}
  Note that in the statement of soundness we allow the message label for the local type context reduction 
  to be different to the message label for the global type reduction. 
  This is because our use of subtyping in association causes the entries in the local type context
  to be less expressive than the types obtained by projecting the global type. For example consider
    $\Gamma = \pp : \ltsend{q}{\ell_0(\tint).\tend}, \; \pq : \ltrec{p}{\ell_0(\tint).\tend, \ell_1(\tint).\tend}$
    and $\G= \GvtPair{p}{q}{\ell_0(\tint).\tend, \ell_1(\tint).\tend}$.
  We have $\Gamma \assoc \G$ and $\G \lts{\lblsync{p}{q}{\ell_1}}$.
  However $\Gamma \lts{\lblsync{p}{q}{\ell_1}}$ is not a valid transition.
\end{remark}

\section{Properties of Local Type Contexts} \label{sec-props}
We now use the LTS semantics to define some desirable properties on type contexts and their 
reduction sequences. Namely, we formulate
safety, fairness and liveness properties based on the definitions in \cite{LessIsMoreRevisited}. 
\subsection{Safety}
We start by defining the \textit{safety} property that plays an important role 
in bottom-up session type systems \cite{LessIsMore}:
\begin{definition}[Safe Type Contexts] \label[definition]{def:safety}
  We define $\safe$ coinductively as the largest set of type contexts such that whenever we have $\Gamma \in \safe$:
  \begin{align}
    \label{saferule:sendrec} \tag*{\rulesafesync} & \quad \lbltrans{\Gamma}{\lblsend{p}{q}{\ell(S)}}{} \; and \; \lbltrans{\Gamma}{\lblrec{q}{p}{\ell'(S')}} \; implies 
    \; \lbltrans{\Gamma}{\lblsync{p}{q}{\ell}}\\
    \label{saferule:reduce} \tag*{\rulesafereduce} & \quad  \lbltrans{\Gamma}{}{\Gamma'} \;implies\; \Gamma' \in \safe
  \end{align}
  We write $\safe(\Gamma)$ if \;$\Gamma \in \safe$.
\end{definition}
\begin{tcolorbox}[colback=blue!10]
Safety says that if $\pp$ and $\pq$ communicate with each other and $\pp$ requests to 
send a value using message label $\ell$, then $\pq$ should be able to receive that message label.
Furthermore, this property should be preserved under any typing context reductions. 
\end{tcolorbox}

Being a coinductive property, to show that $\safe(\Gamma)$ it suffices to give a set
$\varphi$ such that $\Gamma \in \varphi$ and $\varphi$ satisfies \rulesafesync and \rulesafereduce.
This amounts to showing that every element of $\Gamma'$ of the set of reducts of $\Gamma$,
defined
$\varphi := \{\Gamma' \; | \; \Gamma \lts{}^* \Gamma' \}$, satisfies \rulesafesync. 
We illustrate this with some examples:
\begin{example}\label{exam:safe}
   Let $\Gamma=\prt{p}:\procoutsingle{q}{\ell_0(\tint)}{\tend},\prt{q}:\procinsingle{p}{\ell_0(\tnat)}{\tend} $. 
   $\Gamma$ is not safe as as we have $\lbltrans{\Gamma}{\lblsend{p}{q}{\ell_0}}{}$ and 
   $\lbltrans{\Gamma}{\lblrec{q}{p}{\ell_0}}{}$ 
   but we don't have $\lbltrans{\Gamma}{\lblsync{p}{q}{\ell_0}}$ as $\tint \nleqslant \tnat $.
   
  Consider $\Gamma$ from Example \ref{exam:reductions}.  All the reducts satisfy \rulesafesync, hence $\Gamma$ is safe.
\end{example}

In Rocq, we define $\safe$ coinductively with Paco:
\begin{tcb}{Rocq}
Definition weak_safety (c: tctx ) :=
  forall p q s s'  k k', tctxRE (lsend p q (Some s) k) c -> tctxRE (lrecv q p (Some s') k') c -> tctxRE (lcomm p q k) c.
Inductive safe (R: tctx -> Prop): tctx -> Prop :=
  | safety_red :  forall c, weak_safety c -> (forall p q c' k, tctxR c (lcomm p q k) c' -> R c') ->  safe R c.
Definition safeC c := paco1 safe bot1 c.
\end{tcb}
\lstin{weak_safety} corresponds \rulesafesync where \lstin{tctxRE l c} is shorthand for
\lstin{exists c', tctxR c l c'}. 
In the inductive \lstin{safe}, the constructor \lstin{safety_red} corresponds to 
\rulesafereduce. Then \lstin{safeC} is defined as the greatest fixed point of \lstin{safe}.

We have that local type contexts with associated global types are always safe.
\begin{theorem}[Safety by Association \rocqlink{todo}]
  If \lstin{assoc gamma g} then \lstin{safeC gamma}.
\end{theorem}
\subsection{Fairness and Liveness}
\label{subsec-ltl}
We now focus our attention to fairness and liveness. 
We first restate the definition of fairness and liveness for 
local type context paths from \cite{LessIsMoreRevisited}.
\begin{definition}[Fair, Live Paths] \label[definition]{def-fair-live}
A local type context reduction path (also called executions or runs) is a possibly infinite sequence of transitions
$\Gamma_0 \lts{\lambda_0} \Gamma_1 \lts{\lambda_1} ..$ such that $\lambda_i$ is a synchronous transition label,
that is, of the form $\lblsync{p}{q}{\ell}$, for all $i$.

We say that a local type context reduction path $\Gamma_0 \xrightarrow{\lambda_0} \Gamma_1 \xrightarrow{\lambda_2} ..$ is fair if,
for all $n \in N: 
\Gamma_n \lts{\lblsync{\prt{p}}{\prt{q}}{\ell}}$ implies 
$\exists k,\ell'$ such that $N \ni k \ge n$ and 
$\lambda_k = \lblsync{\prt{p}}{\prt{q}}{\ell'}$, and therefore
$\Gamma_k \lts{\lblsync{\prt{p}}{\prt{q}}{\ell'}} \Gamma_{k+1}$.
We say that a path $(\Gamma_n)_{n \in N}$ is live iff, $\forall n \in N$:
\begin{enumerate}
  \item $\forall n \in N: \Gamma_n \lts{\lblsend{\prt{p}}{\prt{q}}{\ell(S)}}$ implies $\exists k,\ell'$ such that $N \ni k \ge n$ and $\Gamma_k \lts{\lblsync{\prt{p}}{\prt{q}}{\ell'}} \Gamma_{k+1}$
  \item $\forall n \in N: \Gamma_n \lts{\lblrec{\prt{q}}{\prt{p}}{\ell(S)}}$ implies $\exists k,\ell'$ such that $N \ni k \ge n$ and $\Gamma_k \lts{\lblsync{\prt{p}}{\prt{q}}{\ell'}} \Gamma_{k+1}$
\end{enumerate}
\end{definition}
\begin{definition}[Live Local Type Context]\label[definition]{def-live-ctx}
  A local type context $\Gamma$ is live if whenever $\Gamma \rdc^{*} \Gamma'$,
  every fair path starting from $\Gamma'$ 
  is also live.
\end{definition}
\begin{tcolorbox}[colback=blue!10]
In general, fairness assumptions are used so that only the reduction sequences that are
"well-behaved" in some sense are considered when formulating other properties \cite{fairness}. 
We define fairness such that, in a fair path, whenever a synchronous transition $\lblsync{p}{q}{\ell}$
is enabled, a communication between $\pp$ and $\pq$ is eventually executed.  
Then live paths are defined to be paths such that whenever $\pp$ attempts to send to $\pq$ \textit{or} 
$\pq$ attempts to receive from $\pp$, eventually a $\pp$ to $\pq$ communication takes place.
Informally, this means that every communication request is eventually answered.
Live typing contexts are then defined to be the $\Gamma$ such that whenever 
$\Gamma$ can evolve (in possibly multiple steps) into $\Gamma'$, 
all fair paths that start from  $\Gamma'$ are also live.  
\end{tcolorbox}


\begin{example}\label[examples]{exam:live}
  Consider the contexts $\Gamma, \Gamma'$ and  $\Gamma_\tend$ from 
  Example \ref{exam:reductions}. One possible reduction path is 
  $\Gamma \lts{\lblsync{\prt{p}}{\prt{q}}{\ell_0}} \Gamma \lts{\lblsync{\prt{p}}{\prt{q}}{\ell_0}} \dots$. 
  Denote this path as $(\Gamma_n)_{n \in \mathbb{N}}$, where $\Gamma_n = \Gamma$ for all $n \in \mathbb{N}$. 
We have $\forall n, \Gamma_n \lts{\lblsync{\prt{p}}{\prt{q}}{\ell_0}}$ and 
  $\Gamma_n \lts{\lblsync{\prt{p}}{\prt{q}}{\ell_1}}$ as the only possible synchronised reductions from $\Gamma_n$.
  Accordingly, we also have $\forall n, \Gamma_n \lts{\lblsync{\prt{p}}{\prt{q}}{\ell_0}} \Gamma_{n+1}$ in the path so this path is fair.
  However, this path is not live as we have $\Gamma_1 \lts{\lblrec{\prt{r}}{\prt{q}}{\ell_2(\tint)}}$ but there is no $n, 
  \ell'$ with $\Gamma_n \lts{\lblsync{\prt{q}}{\prt{r}}{\ell'}} \Gamma_{n+1}$ in the path.
  Consequently, $\Gamma$ is not a live type context.
  %FIX

  Now consider the reduction path $\Gamma \lts{\lblsync{\prt{p}}{\prt{q}}{\ell_0}} \Gamma 
  \lts{\lblsync{p}{q}{\ell_0}}  \Gamma' \lts{\lblsync{\prt{q}}{\prt{r}}{\ell_2}} 
  \Gamma_\tend$. 
  This path is fair and live as it contains the $\lblsync{\prt{q}}{\prt{r}}{}$ transition 
  from the counterexample above.   
\end{example}
\cref{def-fair-live} , while intuitive, is not really convenient for a Rocq formalisation due to 
the existential statements it contains. It would be ideal if these properties could
be expressed as a least or greatest fixed point, which could then be formalised via Rocq's 
inductive or (via Paco) coinductive  types. 
To achieve this, we recast fairness and liveness for local type context paths 
in Linear Temporal Logic (LTL) 
\cite{pnueli1977temporal}. 
The LTL operators \textit{eventually} ($\lozenge$) and \textit{always} ($\square$) can be characterised as least and greatest fixed points using their expansion laws \cite[Chapter 5.14]{baier}.
Hence they can be implemented in Rocq as the inductive type \lstin{eventually} \rocqlink{todo} and the coinductive type 
\lstin{alwaysCG} \rocqlink{todo}. We can further represent reduction paths
as \textit{cosequences}, or \textit{streams}. Then the Rocq definition of \cref{def-fair-live} amounts to the following \rocqlink{todo}:

\begin{minipage}{0.4\textwidth}
\begin{tcb}{Rocq}
CoInductive coseq (A: Type): Type :=
  | conil : coseq A
  | cocons: A -> coseq A -> coseq A.
Notation local_path := (coseq (tctx*option label)).
\end{tcb}
\end{minipage}
\begin{minipage}{0.5\textwidth}
\begin{tcb}{Rocq}
Definition fair_path_local_inner (pt: local_path): Prop :=
  forall p q n, to_path_prop (tctxRE (lcomm p q n)) False pt ->  eventually (headComm p q) pt.
Definition fair_path := alwaysCG fair_path_local_inner.
Definition live_path_inner (pt: local_path) : Prop := forall p q s n, 
(to_path_prop (tctxRE (lsend p q (Some s) n)) False pt -> eventually (headComm p q) pt) /\
(to_path_prop (tctxRE (lrecv p q (Some s) n)) False pt -> eventually (headComm q p) pt).
Definition live_path := alwaysCG live_path_inner.
\end{tcb}  
\end{minipage}

With these definitions we can now prove that local type contexts associated with a global type are live,
which is the most involved of the results mechanised in this work.
\begin{remark}
  We once again emphasise that all global types mentioned are assumed to be balanced (\cref{def-balance}).
  Indeed association with non-balanced global types doesn't guarantee liveness. 
  As an example, consider $\Gamma$ from Example \ref{exam:reductions}, which is associated with 
  $\G$ from Example \ref{exam:assoc}. Yet we have shown in Example \ref{exam:live} that $\Gamma$
  is not a live type context. This is not surprising as $\G$ is not balanced.  
\end{remark}
\begin{theorem}[Liveness by Association \rocqlink{todo}]\label{theo-ctx-live}
If \lstin{assoc gamma g} then \lstin{gamma} is live.  
\end{theorem}
\begin{proof} (Simplified, Outline) Our proof proceeds in two steps. 
First, we prove that the typing context obtained by direct projections 
\footnote{Note that the actual Rocq proof defines an equivalent "enabledness" predicate on 
global types instead of working with direct projections. 
The outline given here is a slightly simplified presentation.} of \lstin{g},
that is, \lstin{gamma_proj = } $\{ \pp_i : \G \proj{\pp_i} \SEP \pp_i \in \participant{\G}\}$, is live. 
We then leverage
\cref{theo-completeness} to show that if \lstin{gamma_proj} is live, so is \lstin{gamma}. 

Suppose  \lstin{gamma_proj} $\lts{\lblsend{p}{q}{\ell(S)}}$ (the case for the receive is similar and omitted), 
and \lstin{xs} is a fair local type context reduction path beginning with \lstin{gamma_proj}.
To show that \lstin{xs} is live we need to show the 
existence of a  $\lblsync{p}{q}{\ell}$ transition in \lstin{xs}. 
We achieve this by taking the height of the \lstin{p}-grafting of the 
global type associated with the head of \lstin{xs} as our induction invariant.
We show (\rocqlink{todo}, \rocqlink{todo}, \rocqlink{todo}) that this invariant keeps decreasing until a $\lblsync{p}{q}{\ell}$ transition is enabled on the path,
at which point our fairness assumption forces that transition to fire \rocqlink{todo}. 

In the second step of the proof we extend association on to paths to get, for each local type context reduction path 
\lstin{xs} that begins with \lstin{gamma}, another local type context reduction path \lstin{ys} beginning with \lstin{gamma_proj}
such that the elements of \lstin{xs} are subtypes (subtyping on contexts defined pointwise) 
of the corresponding elements of \lstin{ys}. This is obtained from \cref{theo-completeness}, however the statement of
\cref{theo-completeness} is implemented as an \lstin{exists} statement
that lives in \lstin{Prop}, hence we need to use the \lstin{constructive_indefinite_description} axiom
to construct a \lstin{CoFixpoint} returning the desired cosequence \lstin{ys}. The proof then follows by the definition of
subtyping (\cref{def:subtyping}).
\end{proof}
\section{Properties of Sessions}\label{sec-proc-props}

We give typing rules for the session calculus introduced in \ref{sec-procs}, and prove subject 
reduction and progress for them. Then we define 
a liveness property for sessions, and show that processes typable by a local type context 
that's associated with a global type tree are guaranteed to satisfy this liveness property.
\subsection{Typing rules}
We give typing rules for our session calculus based on \cite{SynchronousSubtyping} and \cite{srpaper}. 

We distinguish between two kinds of typing judgements and type contexts.
\newcommand{\vdashm}{\vdash_\M}
\newcommand{\vdashp}{\vdash_\PT}
\begin{enumerate}
    \item A local type context $\Gamma$ associates participants with local type trees, as defined in
    c{def-type-ctx}. Local type contexts are used to type sessions (\cref{def:sessions}) i.e. a set of 
    pairs of participants and single processes composed in parallel. We express 
    such judgements as $\Gamma \vdashm \M$, or as \lstin{typ_sess M gamma} in Rocq.
    \item A process variable context $\Theta_\T$ associates process variables with local type trees,
    and an expression variable context $\Theta_\kf{e}$ 
    assigns sorts to expresion variables.
    Variable contexts are used to type single processes and expressions (\cref{def:processes}). 
    Such judgements are expressed 
    as $\Theta_\T, \Theta_\kf{e} \vdashp \PT : \T$, or in as \lstin{typ_proc theta_T theta_e P T}.
\end{enumerate}

\begin{table}[h]
{\footnotesize
\[
\begin{array}{@{}l@{}}
  \inferrule%[\rulename{steq}]
  {}
  {\Theta \vdashp n\colon \texttt{nat}\qquad \Theta \vdashp i\colon \texttt{int}
  \qquad \Theta \vdashp \mathtt{true}\colon \texttt{bool}\qquad \Theta \vdashp \mathtt{false}\colon \texttt{bool}\qquad \Theta,x\colon\ST \vdashp x\colon \ST}
      \\[3mm]
     \inferrule%[\rulename{steq}]
  {\Theta \vdashp e\colon \texttt{nat}}
  {\Theta \vdashp \mathtt{succ}\ e\colon \texttt{nat}}
      \quad
     \inferrule%[\rulename{steq}]
  {\Theta \vdashp e\colon \texttt{int}}
  {\Theta \vdashp \mathtt{neg}\ e\colon \texttt{int}}
  \qquad
     \inferrule%[\rulename{steq}]
  {\Theta \vdashp e\colon \texttt{bool}}
  {\Theta \vdashp \neg\ e\colon \texttt{bool}}
      \\[3mm]
     \inferrule%[\rulename{steq}]
  {\Theta \vdashp e_1\colon \ST \quad \Theta \vdashp e_2\colon \ST}
  {\Theta \vdashp  e_1\oplus e_2\colon \ST}
  \qquad
     \inferrule%[\rulename{steq}]
  {\Theta \vdashp e_1\colon \texttt{int} \quad \Theta \vdashp e_2\colon \texttt{int}}
  {\Theta \vdashp \ e_1 > e_2\colon \texttt{bool}}
  \qquad
     \inferrule%[\rulename{steq}]
  {\Theta \vdashp e\colon \ST \quad \ST \subso \ST'}
  {\Theta \vdashp e\colon \ST'}
\end{array}
\]}
\caption{Typing expressions}
\label{tbl:expr}
\end{table}
\begin{table}[h]
{\footnotesize
\[
\begin{array}{@{}l@{}}
  \inferrule[\rulename{t-end}]
  {}
  {\Theta \vdashp \inact \colon \tend}
\quad
  \inferrule[\rulename{t-var}]
  {}
  {\Theta,\Xv\colon\T \vdashp \Xv\colon\T}
 \quad
  \inferrule[\rulename{t-rec}]
  {\Theta,\Xv\colon\T \vdashp \PT\colon \T}
  {\Theta \vdashp \mu\Xv.\PT\colon\T}
 \quad
  \inferrule[\rulename{t-if}]
  {\Theta\vdashp e\colon \texttt{bool} \quad \Theta \vdashp \PT_1\colon \T  \quad \Theta \vdashp \PT_2\colon \T}
  {\Theta \vdashp \texttt{if}\ \kf{e} \ \mathtt{then}\ \PT_1  \ \mathtt{else} \ \PT_2\colon \T}
      \\[4mm]
  \inferrule[\rulename{t-sub}]
  {\Theta \vdashp \PT\colon \T \quad \T\leqslant \T'}
  {\Theta \vdashp \PT\colon \T'}
  \quad
   \inferrule[\rulename{t-in}]
  {\forall i\in I,\quad \Theta,x_i\colon\ST_i \vdashp \PT_i\colon \T_i}
  {\Theta \vdashp \sum_{i \in I} \prt{p} ?\ell_i(x_i).\PT_i\colon  
  \ltrec{\pp}{\ell_i(\ST_i).\T_i}_{i \in I}}
 \quad
   \inferrule[\rulename{t-out}]
  {\Theta \vdashp e\colon\ST \quad \Theta \vdashp \PT\colon \T}
  {\Theta \vdashp \prt{p}!\ell(\kf{e}).\PT \ \colon  \ {\pp}\sendsign\{{\ell(\ST).\T}\}}
\end{array}
\]}
\caption{Typing processes}
\label{tbl:proc}
\end{table}
\cref{tbl:expr} and \cref{tbl:proc} state the standard typing rules for expressions and processes.
We have a single rule for typing sessions:
\[
\inferrule[\rulename{t-sess}]{\forall i \in I: \qquad \vdashp \PT_i \colon \Gamma(\pp_i) \qquad 
\Gamma \assoc \G}
{\Gamma \vdashm \prod_i \pp_i \triangleleft \PT_i}
\]

\subsection{Subject Reduction, Progress and Session Fidelity}
The subject reduction, progress and non-stuck theorems from \cite{srpaper} \todo{give theorem no}
also hold in this setting, with minor changes in their statements and proofs. 
We won't discuss these proofs in detail.
\begin{lemma}
  If \lstin{typ_sess M gamma} and \lstin{unfoldP M M'} then \lstin{typ_sess M' gamma}.
\end{lemma}
\begin{proof}
  By induction on \lstin{unfoldP M M'}.
\end{proof}
\begin{theorem}[Subject Reduction]\label{theo-sub-red}
If \lstin{typ_sess M gamma} and \lstin{betaP_lbl M (lcomm p q ell) M'}, then 
there exists a typing context \lstin{gamma'} such that 
\lstin{tctxR gamma (lcomm p q ell) gamma'}
and \lstin{typ_sess M' gamma'}.
\end{theorem}
\begin{theorem}[Progress]
If \lstin{typ_sess M gamma}, one of the following hold :
\begin{enumerate}
  \item Either \lstin{unfoldP M M_inact} where every process making up \lstin{M_inact}
  is inactive, i.e. 
  \lstin{M_inact}$= \prod_{i=1}^{n}\pp_i \triangleleft \bf{0}$ for some $n$. 
  \item Or there is a \lstin{M'} such that \lstin{betaP M M'}.
\end{enumerate}
\end{theorem}
\begin{remark}
Note that in \cref{theo-sub-red} one transition between sessions corresponds to exactly one transition
between local type contexts with the same label. That is, every session transition is observed by the corresponding type.
This is the main reason for our choice of 
reactive semantics (\cref{def-sess-semantics}) as $\tau$ transitions are not observed by the type in
ordinary semantics. In other words, with $\tau$-semantics the typing relation is a \textit{weak simulation} \cite{weakbisim},
while it turns into a strong simulation with reactive semantics. For our Rocq implementation 
working with the strong simulation turns out be more convenient. 
\end{remark}
We can also prove the following correspondence result in the reverse direction to \cref{theo-sub-red}, analogus 
to \cref{theo-soundness}.
\begin{theorem}[Session Fidelity]\label{theo-sess-fid}
If \lstin{typ_sess M gamma} and \lstin{tctxR gamma (lcomm p q ell) gamma'}, there exists a
message label
\lstin{ell'} and a session \lstin{M'} such that \lstin{betaP_lbl M (lcomm p q ell') M'} 
and \lstin{typ_sess M' gamma'}.  
\end{theorem}
\begin{proof}
By inverting the local type context transition and the typing.
\end{proof}
\begin{remark}
Again we note that by \cref{theo-sess-fid} a single-step context reduction induces 
a single-step session reduction on the type. 
With the $\tau$-semantics the session reduction induced by 
the context reduction would be multistep.
\end{remark}
\subsection{Session Liveness}
We state the liveness property we are interested in proving,
and show that typable sessions have this property.
\begin{definition}[Session Liveness]\label[definition]{def-live-sess}
  Session $\M$ is live iff
  \begin{enumerate}
    \item ${\M}\longrightarrow^* {\M'} \Rrightarrow \pq \triangleleft \tout{\pp}{\ell_i}{x_i}.Q \;|\; \N$ implies 
    ${\M'}\longrightarrow^*{\M''} \Rrightarrow \pq \triangleleft Q \;|\; \N'$ for some $\M'', \N'$
    
    \item ${\M}\longrightarrow^* {\M'} \Rrightarrow \pq \triangleleft \texternal \tin{\pp}{\ell_i}{x_i}.Q_i \;|\; \N$ implies 
    ${\M'}\longrightarrow^*{\M''} \Rrightarrow \pq \triangleleft Q_i[v/x_i] \;|\; \N'$ for some $\M'', \N', i, v.$ 
  \end{enumerate}
\end{definition}
Session liveness, analogous to liveness for typing contexts (\cref{def-fair-live}), says that 
when $\M$ is live, if $\M$ reduces to a session $\M'$ containing a participant that's attempting to send 
or receive, then $\M'$ reduces to a session where that communication has happened. 
It's also called \textit{lock-freedom} in related work (\cite{fairnesslock,padovani}). 

We now prove that typed sessions are live. Our proof follows the following steps:
\begin{enumerate}
  \item Formulate an analogue of fairness for typable sessions,
  with the property that any finite session reduction path can be extended to a fair
  session reduction path.  
  \item Lift the typing relation to reduction paths, and show that fair session reduction paths
  are typed by fair local type context reduction paths.
  \item Prove that a certain transition eventually happens in the local context reduction 
  path, and that this means the desired transition is enabled in the session reduction path.   
\end{enumerate}
\begin{theorem}[Liveness by Typing]
  If $\Gamma \vdashm \M$ then $\M$ is live.
\end{theorem}
\begin{proof}
  We proceed by assuming that Item 1 of \cref{def-live-sess}
  doesn't hold and showing a contradiction. The case for when Item 2 doesn't hold is similar.

  Suppose  $\transcls{\M}{\M'} \Rrightarrow \pq \triangleleft \tout{\pp}{\ell_i}{x_i}.Q \;|\; \N$,
  and there is no $\M''$ such that 
  $\M'' \Rrightarrow \pq \triangleleft Q \;|\; \N'$. By subject reduction we have 
  $\Gamma' \vdashm \M'$ for some $\Gamma'$. By inverting $\rulename{T-out}$ and subtyping
  we get that $\Gamma'(\pq)=\ltsend{p}{\ell_i(S_i).\T_i}_{i \in I}$ for some $I$.
  Therefore by \cref{def-ctx-red} we have that 
  $\Gamma'
  \lts{\lblsend{q}{p}{\ell_i}}$ for some $i$.
  Furthermore by our assumption we don't ever have $\pp \triangleleft \texternal \tin{\pq}{\ell_i(x_i).\PT_i}$
  in a session $\M'$ can reduce to, hence for all $\M''$ s.t. $\M' \longrightarrow^* \M''$
  and $\Gamma'' \vdashm \M''$ we don't have $\Gamma''(\pp)=\ltrec{q}{...}$.
  Thus we have that there is no reduction path from $\Gamma'$ that contains a $\lblsync{p}{q}{\ell_k}$
  transition. Now suppose there exists a fair path from $\Gamma'$, then on this path 
  we perpetually have $\Gamma'(\pq) \lts{\lblsend{q}{p}{\ell_i}}$ 
  without ever transitioning via $\lts{\lblsync{p}{q}{\ell_k}}$, therefore this path is not live.
  Therefore $\Gamma$ is not a live type context.
  However, we have by \ref{theo-assoc-live} that $\Gamma$ is live, which is a contradiction.
  
  Now it suffices to show that there exists a fair path starting from $\Gamma'$. This 
  path always exists as we can just "schedule" the available synchronous transitions so that
  all of them are eventually executed e.g. in a round-robin-like way.  
\end{proof}
\section{Conclusion}
\subsection{Summary}
In this work, we have provided a typing system based on coinductive trees (\cref{sec:types}) 
for a simple synchoronous session calculus (\cref{sec-procs}).
We then defined local type contexts, provided a transition system for them,
and defined association between local type contexts and global types. 
In \cref{sec-props} we stated 
the safety and liveness properties for local type contexts based on this transition system,
and proved that these properties hold for associated types.
Then in \cref{sec:proc-props} we gave the typing rules for our session calculus,
formulated a liveness property for sessions, and showed that typeable sessions satisfy this property.
We have also mechanised many of the results in Rocq, with the soundness proof being the most involved
of them.  
\subsection{Future Work}
As far as the Rocq implementation is concerned, an obvious next step would be to actually finish 
the mechanisation. We believe that the definitions we provided are scalable and provide a good foundation
for future formalisation efforts. Beyond the verification of proofs, a full formalisation 
of the results of this paper would allow us to extract Rocq programs that are certifiably live,
which could have practical applications in error-sensitive settings. An example of this 
approach is the Zooid DSL \cite{zooid}.

One other possible future point of interest would be to extend our formalisation to use full merge instead of
plain merge. Time and again, proofs of properties such as subject reduction in settings
using full merge were shown to be flawed \cite{LessIsMore}, and formal verification of such proofs
would aid future research greatly. 

Another avenue for future efforts would be to investigate different flavours of fairness and 
liveness using the "association" framework we employed. There are dozens of fairness 
properties in the wider concurrency theory literature \cite{fairness}, with different fairness 
assumptions being useful for different liveness properties \cite{fairnesslock}. It could be interesting
to approach these notions of fairness from a MPST point of view and show how a specific typing discipline
relates to these properties. In particular, it would be nice to see variations on the balancedness (\cref{def-balance})
property we imposed on our type trees and see which of our results still hold, and what liveness
properties the new balancedness assumptions relate to.


\bibliography{references}

\end{document}
