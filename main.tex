
\documentclass[a4paper,UKenglish,cleveref, autoref, numberwithinsect, anonymous, thm-restate]{lipics-v2021}
%This is a template for producing LIPIcs articles. 
%See lipics-v2021-authors-guidelines.pdf for further information.
%for A4 paper format use option "a4paper", for US-letter use option "letterpaper"
%for british hyphenation rules use option "UKenglish", for american hyphenation rules use option "USenglish"
%for section-numbered lemmas etc., use "numberwithinsect"
%for enabling cleveref support, use "cleveref"
%for enabling autoref support, use "autoref"
%for anonymousing the authors (e.g. for double-blind review), add "anonymous"
%for enabling thm-restate support, use "thm-restate"
%for enabling a two-column layout for the author/affilation part (only applicable for > 6 authors), use "authorcolumns"
%for producing a PDF according the PDF/A standard, add "pdfa"

%\graphicspath{{./graphics/}}%helpful if your graphic files are in another directory
\usepackage{prooftree}
\usepackage{mathpartir}
\usepackage{mpst_macros}
\bibliographystyle{plainurl}% the mandatory bibstyle

\title{Dummy title} %TODO Please add

\titlerunning{Dummy short title} %TODO optional, please use if title is longer than one line

\author{John Q. Public}{Dummy University Computing Laboratory, [optional: Address], Country \and My second affiliation, Country \and \url{http://www.myhomepage.edu} }{johnqpublic@dummyuni.org}{https://orcid.org/0000-0002-1825-0097}{(Optional) author-specific funding acknowledgements}%TODO mandatory, please use full name; only 1 author per \author macro; first two parameters are mandatory, other parameters can be empty. Please provide at least the name of the affiliation and the country. The full address is optional

\author{Joan R. Public\footnote{Optional footnote, e.g. to mark corresponding author}}{Department of Informatics, Dummy College, [optional: Address], Country}{joanrpublic@dummycollege.org}{[orcid]}{[funding]}

\authorrunning{J.\,Q. Public and J.\,R. Public} %TODO mandatory. First: Use abbreviated first/middle names. Second (only in severe cases): Use first author plus 'et al.'

\Copyright{John Q. Public and Joan R. Public} %TODO mandatory, please use full first names. LIPIcs license is "CC-BY";  http://creativecommons.org/licenses/by/3.0/

\ccsdesc[100]{\textcolor{red}{Replace ccsdesc macro with valid one}} %TODO mandatory: Please choose ACM 2012 classifications from https://dl.acm.org/ccs/ccs_flat.cfm 

\keywords{Dummy keyword} %TODO mandatory; please add comma-separated list of keywords

\category{} %optional, e.g. invited paper

\relatedversion{} %optional, e.g. full version hosted on arXiv, HAL, or other respository/website
%\relatedversiondetails[linktext={opt. text shown instead of the URL}, cite=DBLP:books/mk/GrayR93]{Classification (e.g. Full Version, Extended Version, Previous Version}{URL to related version} %linktext and cite are optional

%\supplement{}%optional, e.g. related research data, source code, ... hosted on a repository like zenodo, figshare, GitHub, ...
%\supplementdetails[linktext={opt. text shown instead of the URL}, cite=DBLP:books/mk/GrayR93, subcategory={Description, Subcategory}, swhid={Software Heritage Identifier}]{General Classification (e.g. Software, Dataset, Model, ...)}{URL to related version} %linktext, cite, and subcategory are optional

%\funding{(Optional) general funding statement \dots}%optional, to capture a funding statement, which applies to all authors. Please enter author specific funding statements as fifth argument of the \author macro.

\acknowledgements{I want to thank \dots}%optional

%\nolinenumbers %uncomment to disable line numbering

%\hideLIPIcs  %uncomment to remove references to LIPIcs series (logo, DOI, ...), e.g. when preparing a pre-final version to be uploaded to arXiv or another public repository

%Editor-only macros:: begin (do not touch as author)%%%%%%%%%%%%%%%%%%%%%%%%%%%%%%%%%%
\EventEditors{John Q. Open and Joan R. Access}
\EventNoEds{2}
\EventLongTitle{42nd Conference on Very Important Topics (CVIT 2016)}
\EventShortTitle{CVIT 2016}
\EventAcronym{CVIT}
\EventYear{2016}
\EventDate{December 24--27, 2016}
\EventLocation{Little Whinging, United Kingdom}
\EventLogo{}
\SeriesVolume{42}
\ArticleNo{23}
%%%%%%%%%%%%%%%%%%%%%%%%%%%%%%%%%%%%%%%%%%%%%%%%%%%%%%

\begin{document}

\maketitle

%TODO mandatory: add short abstract of the document
\begin{abstract}
Lorem ipsum dolor sit amet, consectetur adipiscing elit. Praesent convallis orci arcu, eu mollis dolor. Aliquam eleifend suscipit lacinia. Maecenas quam mi, porta ut lacinia sed, convallis ac dui. Lorem ipsum dolor sit amet, consectetur adipiscing elit. Suspendisse potenti. 
\end{abstract}

\section{Introduction}
\label{sec:introduction}

\section{The Session Calculus}
\label{sec:ses-calc}
We introduce the simple synchronous session calculus that our type system will be used on. 
\newcommand{\PT}{\mathsf P}
\newcommand{\Xv}{\ensuremath{{\textbf{X}}}}
\newcommand{\ST}{{\mathsf S}}
\newcommand{\QT}{{\mathsf Q}}
\newcommand{\RT}{{\mathsf R}}

\subsection{Processes and Sessions}
\begin{definition}[Expressions and Processes]
  \label[definition]{def:processes}
  We define processes as follows: 
  \begin{align*}
    \PT ::= \prt{p}!\ell(\kf{e}).\PT \SEP \sum_{i \in I}
\prt{p}?\ell_i(x_i).\PT_i \SEP \cond{\kf{e}}{\PT}{\PT} \SEP \mu \Xv.\PT \SEP \Xv \SEP \textbf{0}
    \end{align*}
    where \kf{e} is an expression that can be a variable, a value such as \texttt{true}, $0$ or $-3$, 
    or a term built from expressions by applying the operators \texttt{succ}, \texttt{neg}, $\neg$, 
    non-deterministic choice $\sendsign$ and $>$. 
\end{definition}
$\prt{p}!\ell(\kf{e}).\PT$ is a process that sends the value of expression $\kf{e}$ 
with label $\ell$ to participant $\pp$, and continues with process $\PT$.
$\sum_{i \in I}
\prt{p}?\ell_i(x_i).P_i$ is a process that may receive a value from any 
$\ell_i \in I$, binding the result to $x_i$ and continuing with $\PT_i$, depending on which $\ell_i$
the value was received from. $\Xv$ is a recursion variable, $\mu \Xv.\PT$ is a recursive process,
$\cond{\kf{e}}{\PT}{\PT}$ is a conditional and $\textbf{0}$ is a terminated process.

Processes can be composed in parallel into sessions.
\begin{definition}[Multiparty Sessions]\label[definition]{def:sessions}
  Multiparty sessions are defined as follows.
    \[
    \M \;::=\;%
    {\prt{p}} \triangleleft \PT \quad \SEP \quad(\M  \mid \M)
    \quad \SEP \quad \mathcal{O}
  \]
\end{definition}
${\prt{p}} \triangleleft \PT$ denotes that participant $\pp$ is running the process $\PT$,
$\mid$ indicates parallel compositon. We write $\displaystyle  \prod_{i \in I}^{} \pp_i \triangleleft \PT_i$
to denote the session formed by $\pp_i$ running $\PT_i$ in parallel for all $i \in I$. 
$\mathcal{O}$
is an empty session with no participants, that is, the unit of parallel composition.
\begin{remark}
Note that $\mathcal{O}$ is different than
$\pp \triangleleft \textbf{0}$ as $\pp$ is a participant in the latter but not the former. 
This differs from previous work, e.g. in \cite{SynchronousSubtyping} the unit of 
parallel composition is 
$\pp \triangleleft \textbf{0}$. For a detailed discussion see \cref{rem:scong}.
\end{remark} 
\subsection{Structural Congruence and Operational Semantics}
We define a structural congruence relation $\equiv$ on sessions which expresses the commutativity, associativity and 
unit of the parallel composition operator.

\begin{table}[h]
{\footnotesize
\[
\begin{array}{@{}l@{}}
  \inferrule[\rulename{sc-sym}]
  {}
  {\pp \triangleleft \PT  \mid \pq \triangleleft \QT  \equiv \pq \triangleleft \QT \mid \pp \triangleleft \PT} 
  \qquad
  \inferrule[\rulename{sc-assoc}]
  {}
  {(\pp \triangleleft \PT  \mid \pq \triangleleft \QT) \mid \pr \triangleleft \RT  
  \equiv \pp \triangleleft \PT  \mid (\pq \triangleleft \QT \mid \pr \triangleleft \RT)} 
  \\\\
  \inferrule[\rulename{sc-o}]
  {}
  {\pp \triangleleft \PT  \mid \pq \triangleleft \mathcal{O}  
  \equiv \pp \triangleleft \PT } 
\end{array}
\]}
\caption{Structural Congruence over Sessions}
\label{tbl:scong}
\end{table}
We now give the operational semantics for sessions by the means of a labelled transition system.
We have two kinds of transitions, \textit{silent} ($\tau$) and 
\textit{observable} $(\beta)$. Correspondingly, we have two kinds of \textit{transition labels}, 
$\tau$ and $(\pp,\pq)\ell$ where $\pp, \pq$ are participants and $\ell$ is a message label.  
We omit the semantics of expressions, they are 
standard and can be found in \cite[Table 1]{SynchronousSubtyping}. We write $e \downarrow v$ when 
expression $e$ evaluates to value $v$. 
\newcommand{\taured}{\ensuremath{\xrightarrow{\tau}}}

\begin{table}[h]
{\footnotesize
\[
\begin{array}{@{}l@{}}
    \inferrule[\rulename{R-comm}]
  {j\in I \quad e \downarrow v}
  {\pp \triangleleft \sum^{}_{i\in I} \tin\pq{\ell_i}{x_i}.\PT_i \ \mid \  \pq \triangleleft \tout{\pp}{\ell_j}{\kf{e}}.\QT \ \mid \ \N \ \ 
  \xrightarrow{(\pp,\pq)\ell_j} \ \ 
    \pp \triangleleft \PT_j[v/x_j] \ \mid \ \pq \triangleleft \QT \ \mid \ \N}
    \\[10mm]
    \inferrule[\rulename{R-rec}]
  {}
  {\pp \triangleleft \mu \Xv.\PT \mid \N \ \ 
  \taured \ \
    \pp \triangleleft \PT[\mu \Xv.\PT / \Xv] \ \mid \ \N}  
      \\[5mm]
    \inferrule[\rulename{R-condt}]
  {e \downarrow \text{true}}
  {\pp \triangleleft \text{ if } e \text{ then } \PT \text{ else } \QT \ \mid \ \N \ \ 
  \taured \ \
    \pp \triangleleft \PT \ \mid \ \N}    
      \\[5mm]
        \inferrule[\rulename{R-condf}]
  {e \downarrow \text{false}}
  {\pp \triangleleft \text{ if } e \text{ then } \PT \text{ else } \QT \ \mid \ \N \ \ 
  \taured \ \
    \pp \triangleleft \QT \ \mid \ \N}  
      \\[5mm]
        \inferrule[\rulename{R-struct}]
  {\N_1' \equiv \N_1 \quad \N_1 \xrightarrow{\lambda} \N_2 \quad \N_2 \equiv \N_2'}
  {\N'_1 \xrightarrow{\lambda} \N'_2}  
\end{array}
\]}
\caption{Operational Semantics of Sessions}
\label{tbl:srr}
\end{table}
In \cref{tbl:srr}, \rulename{R-comm} describes a synchronous communication from $\pp$ to $\pq$ via 
message label $\ell_j$.
\rulename{R-rec} unfolds recursion, \rulename{R-condt} and \rulename{R-condf} express how to evaluate conditionals,
and \rulename{R-struct} shows that the reduction respects the structural pre-congruence.
We write $\M \xrightarrow{} \N$ if $\M \xrightarrow{\lambda} \N$ for some transition label $\lambda$.
We write $\rightarrow^*$ to denote the reflexive transitive closure of $\rightarrow$. We also write 
$\M \taured^* \N$ when all the transitions involved in the multistep reduction are $\tau$ transitions.

\begin{remark}[rem:unfold-motive]
Later in this paper, we also define types and LTS 
semantics on them, establish \textit{simulations} between sessions and their types, 
and use these simulations to prove properties about sessions. It turns out that $\tau$ transitions
are not observed by types, so it's convenient to elide $\tau$ transitions completely, establish . 
Hence we also define an \textit{unfolding} relationship ($\Rrightarrow$) on sessions.    
\end{remark}
\begin{table}[h]
{\footnotesize
\[
\begin{array}{@{}l@{}}
    \inferrule[\rulename{R-comm}]
  {j\in I \quad e \downarrow v}
  {\pp \triangleleft \sum^{}_{i\in I} \tin\pq{\ell_i}{x_i}.\PT_i \ \mid \  \pq \triangleleft \tout{\pp}{\ell_j}{\kf{e}}.\QT \ \mid \ \N \ \ 
  \xrightarrow{(\pp,\pq)\ell_j} \ \ 
    \pp \triangleleft \PT_j[v/x_j] \ \mid \ \pq \triangleleft \QT \ \mid \ \N}
    \\[10mm]
    \inferrule[\rulename{R-rec}]
  {}
  {\pp \triangleleft \mu \Xv.\PT \mid \N \ \ 
  \taured \ \
    \pp \triangleleft \PT[\mu \Xv.\PT / \Xv] \ \mid \ \N}  
      \\[5mm]
    \inferrule[\rulename{R-condt}]
  {e \downarrow \text{true}}
  {\pp \triangleleft \text{ if } e \text{ then } \PT \text{ else } \QT \ \mid \ \N \ \ 
  \taured \ \
    \pp \triangleleft \PT \ \mid \ \N}    
      \\[5mm]
        \inferrule[\rulename{R-condf}]
  {e \downarrow \text{false}}
  {\pp \triangleleft \text{ if } e \text{ then } \PT \text{ else } \QT \ \mid \ \N \ \ 
  \taured \ \
    \pp \triangleleft \QT \ \mid \ \N}  
      \\[5mm]
        \inferrule[\rulename{R-struct}]
  {\N_1' \equiv \N_1 \quad \N_1 \xrightarrow{\lambda} \N_2 \quad \N_2 \equiv \N_2'}
  {\N'_1 \xrightarrow{\lambda} \N'_2}  
\end{array}
\]}
\caption{The unfolding relation}
\label{tbl:unf}
\end{table}


\section{The Type System}
\label{sec:types}

\section{LTS Semantics for Types}
\label{sec:lts}

\section{Properties of Local Types}
\label{sec:type-props}

\section{Properties of Sessions}
\label{sec:session-props}

\bibliography{references}

\end{document}
