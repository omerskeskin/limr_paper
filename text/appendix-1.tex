
\section{Proofs}
\label{appx:proof}
\subsection{Semantics of typing contexts}
\begin{lemma}[Domain Preservation; \path{src/lcontext.v}, dom\_preservation\_6\_9]
  If \lstin{tctxR gamma l gamma'} then \lstin{M.Eqdom gamma gamma'}.
\end{lemma}
\begin{proof}
  By induction on \lstin{tctxR gamma l gamma'}.
\end{proof}
\begin{lemma}[Context Reduction Weakening; \path{src/lcontext.v}, tctxR\_weakening]
  \label[lemma]{lem-weakening}
  For contexts \lstin{tc,tc'} and \lstin{tc''} and label \lstin{l}, 
  if \lstin{tctxR tc l tc'} and \lstin{tc''} is disjoint from \lstin{tc} and 
  \lstin{tc'}, then \lstin{tctxR (disj_merge tc tc'' _) l (disj_merge tc' tc'' _)}
  (where the underscores indicate the proofs of disjointness).
\end{lemma}
\begin{proof}
  By induction on the context \lstin{tc''} using the induction principle \lstin{MF.map_induction} \cite{mmaps}.
  \lstin{MF.map_induction} asserts that a finite map \lstin{m} is either empty, or it's equal to 
  \lstin{M.add p t m'} for some key \lstin{p}, value \lstin{t} and finite map \lstin{m'}.
  The proof then proceeds by routine rewrites.
\end{proof}

\begin{lemma}[Participant Relevancy; \path{src/lcontext.v}, lem\_6\_10] \label[lemma]{lem-not-subj-same}
  For contexts \lstin{g} and \lstin{g'}, if \lstin{tctxR g l g'}, 
  then for all participants \lstin{r} such that \lstin{\~ ispSubjl r l} we have 
  \lstin{M.find r g = M.find r g'}.
\end{lemma} 
\begin{proof}
  By routine induction on the reduction \lstin{tctxR g l g'}.
\end{proof}

\begin{lemma}[Local Typing Context Reduction Inversion; \path{src/lcontext.v}, lem\_6\_11*]
  \label[lemma]{lem-ctx-red-inv}
  For contexts \lstin{g g'}, participants \lstin{p q}, labels \lstin{ell} and sorts \lstin{s},
  \begin{itemize}
    \item  
    if \lstin{tctxR g (lsend p q (Some s) ell) g'} then 
    exists \lstin{xs Tp'} such that \lstin{M.find p g = Some (ltt_send q xs)} and 
    \lstin{onth ell xs=Some (s, Tp')} and \lstin{M.find p g' = (Some Tp')}   
    \item if \lstin{tctxR g (lrecv p q (Some s) ell) g'} then 
    exists \lstin{xs Tp'} such that \lstin{M.find p g = Some (ltt_recv q xs)} and 
    \lstin{onth ell xs=Some (s, Tp')} and \lstin{M.find p g' = (Some Tp')}
    \item  if \lstin{tctxR g (lcomm p q) g'} then there are sorts \lstin{s s'}
    such that \lstin{subsort s s'} and 
    exists \lstin{xsp Tp'} such that \lstin{M.find p g = Some (ltt_send q xsp)} and 
    \lstin{onth ell xsp=Some (s, Tp')} and \lstin{M.find p g' = (Some Tp')} and 
    exists \lstin{xsq Tp'} such that \lstin{M.find q g = Some (ltt_recv q xsq)} and 
    \lstin{onth ell xsq=Some (s, Tq')} and \lstin{M.find q g' = (Some Tq')}
  \end{itemize}
\end{lemma}
\begin{proof}
  By routine induction on the local typing context reduction.
\end{proof}

\begin{lemma}[Determinism of context reduction; \path{src/lcontext.v}, lem\_6\_12\_reduction\_determinism]
If \lstin{tctxR gamma l gamma'} and \lstin{tctxR gamma l gamma''} then \lstin{M.Equal gamma' gamma''} 
\end{lemma}
\begin{proof}
  Case analysis on \lstin{l}, applying \cref{lem-ctx-red-inv} and \cref{lem-not-subj-same} when needed.
\end{proof}

\begin{lemma} [Inversion of Projection and Subtyping; \path{src/assoc.v}, subproj\_inv\_*] \label[lemma]{lem-subproj-inv}
  We have that
  \begin{enumerate}
  \item If \lstin{issubProj (ltt_send q xs) G p} then one of the following hold:
  \begin{enumerate}[label=(\roman*)]
    \item \lstin{G = gtt_send p q ys} for some \lstin{ys : list (option(sort*gtt))} and 
    for all \lstin{k}, if \lstin{onth k xs = Some (s_x,T_x)} then 
    there are \lstin{s_y, G_y} such that 
    \lstin{onth k ys = Some (s_y,G_y)},
    \lstin{subsort s_x s_y} and \lstin{issubProj T_x G_y p}.
    \item There exist \lstin{s, t : part} and \lstin{ys : list (option(sort*gtt))}
    such that \lstin{G=gtt_send s t ys}, \lstin{p <> s}, \lstin{p <> t} and 
    whenever \lstin{onth k ys= Some (s_y,G_y)}, we have 
    \lstin{issubProj (ltt_send q xs) G_y p}.
  \end{enumerate} 
  \item If \lstin{issubProj (ltt_recv q xs) G p} then one of the following hold:
  \begin{enumerate}[label=(\roman*)]
    \item \lstin{G = gtt_send q p ys} for some \lstin{ys : list (option(sort*gtt))} and 
    for all \lstin{k}, if \lstin{onth k ys = Some (s_y,G_y)} then 
    there are \lstin{s_x, T_x} such that 
    \lstin{onth k xs = Some (s_x, T_x)},
    \lstin{subsort s_y s_x} and \lstin{issubProj T_x G_y p}.
    \item There exist \lstin{s, t : part} and \lstin{ys : list (option(sort*gtt))}
    such that \lstin{G=gtt_send s t ys}, \lstin{p <> s}, \lstin{p <> t} and 
    whenever \lstin{onth k ys= Some (s_y,G_y)}, we have 
    \lstin{issubProj (ltt_recv q xs) G_y p}.
  \end{enumerate} 
  \item If \lstin{issubProj ltt_end G p} then \lstin{ \~ isgPartsC p G}.  
\end{enumerate}
\end{lemma}
\begin{proof}
  By inverting \lstin{subtypeC} and \lstin{projectionC}. Additionally, 
  when proving statements 
  containing \lstin{Forall, Forall2 } or \lstin{Forall2R} we induct on
  \lstin{xs} or \lstin{ys}. 
\end{proof}
\begin{lemma}[Simultaneous Inversion of Projections; \path{src/assoc.v}, lem\_6\_16\_simul\_subproj]\label[lemma]{lem-simul-subproj-inv}
  If \lstin{issubProj (ltt_send q xp) G p} and 
  \lstin{issubProj (ltt_recv p xq) G q} then whenever 
  \lstin{onth k xp = Some (s_p, T_p)}, we have 
  \lstin{onth k xq = Some (s_q, T_q)} where \lstin{subsort s_p s_q}.
\end{lemma}
\begin{proof}
  By induction on the global tree context \lstin{gcx} 
  given by grafting \lstin{G} with \lstin{p}.
\end{proof}
\subsection{Association}
\begin{lemma}[Inversion of association] \label[lemma]{lem-inv-assoc}
  Let \lstin{assoc gamma G}.
  \begin{enumerate}
  \item If \lstin{M.find p gamma= (ltt_send q xs)} then one of the following hold:
  \begin{enumerate}[label=(\roman*)]
    \item \lstin{G = gtt_send p q ys} for some \lstin{ys : list (option(sort*gtt))} and 
    for all \lstin{k}, if \lstin{onth k xs = Some (s_x,T_x)} then 
    there are \lstin{s_y, G_y} such that 
    \lstin{onth k ys = Some (s_y,G_y)},
    \lstin{subsort s_x s_y} and \lstin{issubProj T_x G_y p}.
    \item There exist \lstin{s, t : part} and \lstin{ys : list (option(sort*gtt))}
    such that \lstin{G=gtt_send s t ys}, \lstin{p <> s}, \lstin{p <> t} and 
    whenever \lstin{onth k ys= Some (s_y,G_y)}, we have 
    \lstin{issubProj (ltt_send q xs) G_y p}.
  \end{enumerate} 
  \item If \lstin{M.find p gamma= (ltt_recv q xs)} then one of the following hold:
  \begin{enumerate}[label=(\roman*)]
    \item \lstin{G = gtt_send q p ys} for some \lstin{ys : list (option(sort*gtt))} and 
    for all \lstin{k}, if \lstin{onth k ys = Some (s_y,G_y)} then 
    there are \lstin{s_x, T_x} such that 
    \lstin{onth k xs = Some (s_x, T_x)},
    \lstin{subsort s_y s_x} and \lstin{issubProj T_x G_y p}.
    \item There exist \lstin{s, t : part} and \lstin{ys : list (option(sort*gtt))}
    such that \lstin{G=gtt_send s t ys}, \lstin{p <> s}, \lstin{p <> t} and 
    whenever \lstin{onth k ys= Some (s_y,G_y)}, we have 
    \lstin{issubProj (ltt_recv q xs) G_y p}.
  \end{enumerate} 
\end{enumerate}
\end{lemma}
\begin{proof}
  Follows from the definition of association and \cref{lem-subproj-inv}.
\end{proof}
\begin{lemma}[Simultaneous Inversions of Association] \label[lemma]{sim-inv-assoc}
Let $\Gamma \assoc \G$. If $\Gamma(\pp)=\procoutset{q}{\ell_i(S_i)}{\T_i}{i \in I_p}$ and 
   $\Gamma(\pq)=\procinset{p}{\ell_i(S'_i)}{\T'_i}{i \in I_q}$, then either
   \begin{enumerate}
    \item \label{sim-inv-assoc1} $\G=\GvtPair{\pp}{\pq}{\ell_i(S''_i).\G_i}_{i \in I}$ where $I_p \subseteq I \subseteq I_q, \forall i \in I_p: S_i \subso S''_i$,
    $\forall i \in I: S''_i \subso S'_i$, $\forall i \in I_p, \T_i \subtp \projfn{G_i}{\pp}$ and
    $\forall i \in I: \T'_i \subtp \projfn{G_i}{\pq}$; or,
    \item \label{sim-inv-assoc2} $\G=\GvtPair{\prt{s}}{\prt{t}}{\ell_j(S''_j).\G_j}_{j \in J}$ where $\forall j \in J: \Gamma(\pp) \subtp \projfn{\G_j}{\pp},
    \Gamma(\pq) \subtp \projfn{\G_j}{\pq}$ and $\{\pp,\pq\}\cap \{\prt{s},\prt{t}\}=\emptyset$.
   \end{enumerate}
\end{lemma}
\begin{proof}
  By case analysis on \cref{lem-inv-assoc}.
\end{proof}
\subsection{Soundness of Association}
To prove soundness we prove the following stronger lemma:
\begin{lemma}[Soundness Lemma, \path{lemma/correspond.v}, assoc\_soundness'] \label[lemma]{lem-sound-lemma}
  Suppose for a local type context \lstin{gamma}, global type trees \lstin{G} and \lstin{G'}, 
  message labels \lstin{ell} and \lstin{ell'}, and participants \lstin{p} and \lstin{q}, the following hold:
  \begin{enumerate}[label=(\roman*)]
    \item \lstin{isgPartsC p G} \label{sound-i}
    \item \lstin{M.find p gamma = Some (ltt_send q xs)} \label{sound-ii}
    \item \lstin{assoc gamma G} \label{sound-iii}
    \item \lstin{gttstepC G G' p q ell} \label{sound-iv}
    \item \lstin{onth ell' xs <> None} \label{sound-v}
  \end{enumerate} 
  Then there exists a local type context \lstin{gamma'} and a global type tree \lstin{G''}
  such that \lstin{gttstepC G G'' p q ell'},  \lstin{assoc gamma' G''} and 
  \lstin{tctxR gamma (lcomm p q ell') gamma'}.
\end{lemma}
\begin{proof}
  We start by leveraging \cref{lem-grafting} to generalise the statement 
  over global type contexts. That is, we prove that if the statement is true for any \lstin{G}
  that can be grafted with \lstin{p}, then \cref{lem-grafting} means that the statement holds 
  for any \lstin{G} that has \lstin{p} among its participants.

  Now suppose \lstin{typ_gtth gs Gctx G} where \lstin{\~ ishParts p Gctx} and 
  every \lstin{Some} element of \lstin{gs} is of form either \lstin{gtt_end}, 
  \lstin{gtt_send p q} or \lstin{gtt_send q p}. We proceed by induction on the global tree context
  \lstin{Gctx}.
  \begin{itemize}
    \item Case \lstin{Gctx=gtth_hol n}: Then some element of \lstin{gs} fills the hole,
    hence by the $\pp$-grafting property of \lstin{gs} and \ref{sound-iv} \lstin{G=gtt_send p q gcs}
    for some \lstin{gcs : list(option(sort*gtt))}. 
    Then \lstin{M.find q gamma}
    must be of shape \lstin{ltt_recv p ys} for some \lstin{ys}. Hence \cref{lem-simul-subproj-inv} gives that
    for any index $i$, if \lstin{onth i xs = Some (s_1,T_p)} then \lstin{onth i ys = Some (s_2,T_q)}
    and \lstin{subsort s_1 s2}. By \ref{sound-v}, let \lstin{onth ell' xs = Some (s_1,T_p)},
    then we get \lstin{onth ell' ys = Some (s_2,T_q)} for some \lstin{s_2,T_q}.
    By inverting the projections we get that \lstin{onth ell' gcs=Some (s',G'')} for some \lstin{s', G''}
    and hence \lstin{gttstepC (gtt_send p q gcs) G'' p q ell'}.
    
    Now we construct a local type context
    \lstin{gamma'} such that \lstin{assoc gamma' G''} and \lstin{tctxR gamma (lcomm p q ell') gamma'}.
    Let \lstin{gamma' = M.add p Tp (M.add q Tq (M.remove p (M.remove q gamma)))} i.e. the typing context
    obtained by changing the entry for \lstin{p} into \lstin{T_p} and \lstin{q} into \lstin{T_q}.
    Then by applying \lstin{Rsend, Rrecv, Rcomm} and \lstin{tctxR_weakening} we get that 
    \lstin{tctxR gamma (lsend p q ell') gamma'}. Now all that remains is to show that \lstin{assoc gamma' G''}.

    \lstin{issubProj T_p G'' p} and \lstin{issubProj T_q G'' q} follows from the definitions of 
    \lstin{proj_in} and \lstin{proj_out}. For \lstin{r <> p, q}, 
    we have \lstin{M.find r gamma = M.find r gamma'}. We also have by the definition of plain merging
    that \lstin{projectionC G'' r T_r <-> projectionC G r T_r} for all \lstin{T_r}. 
    The transition \lstin{gttstepC (gtt_send p q gcs) G_ell' p q ell'} doesn't involve \lstin{r}, 
    hence if \lstin{\~ isgPartsC r G''} then  \lstin{\~ isgPartsC r G} as well, so 
    \lstin{M.find r gamma' = M.find r gamma = None} or \lstin{Some gtt_end} as needed.
    Similarly, if \lstin{ isgPartsC r G''} then  \lstin{isgPartsC r G} as well, so 
    {M.find r gamma} is not \lstin{None} and is a subtype of the projection of \lstin{G} onto \lstin{r},
    hence {M.find r gamma'} is a subtype of the projection of \lstin{G''} as well. 
    Hence \lstin{assoc gamma' G''}.

    \item Case \lstin{Gctx = gtth_send s t ghs}: First, by \lstin{p} not being a part of \lstin{Gctx}
    and \ref{sound-iv} we have that $\{\prt{s},\prt{t}\} \cap \{\prt{p},\prt{q}\} = \emptyset$.
    We also have that \lstin{G=gtt_send s t gcs} for some \lstin{gcs}. 
    Suppose \lstin{onth k gcs =Some (s_k, G_k)}. 
    Let \lstin{projectionC G_k s Tks}
    and \lstin{projectionC G_k t Tkt}, and
    \lstin{gamma_k = M.add s Tks (M.add t Tkt (M.remove s (M.remove t gamma)))}
    i.e. \lstin{gamma_k} is \lstin{gamma} except that the types of 
    \lstin{s} and \lstin{t} are updated to be their projections in \lstin{G_k}.
    We have that \lstin{assoc gamma_k G_k}, and by \ref{sound-iv} and \lstin{proj_cont}
    we have that \lstin{gttstepC G_k G'_k p q ell}. Then by applying the induction hypothesis
    we get that \lstin{gttstepC G_k G''_k p q ell'}, \lstin{assoc gamma'_k G''_k}
    and \lstin{tctxR gamma_k (lsend p q ell') gamma'_k}.

    Now let \lstin{gtt_send s t gcs''} be the global type tree obtained by 
    mapping every \lstin{onth k gcs =Some (s_k, G_k)} to 
    \lstin{onth k gcs =Some (s_k, G''_k)} where \lstin{G''_k} is generated as above.
    Now \lstin{proj_cont} gives that \lstin{gttstepC (gtt_send s t gcs) (gtt_send s t gcs'') p q ell'}.
    Now let \lstin{gamma'} be the context such that \lstin{tctxR gamma (lsend p q ell') gamma'}, which exists 
    by inverting the association, the global type reduction and \ref{sound-v}.
    It suffices to show that \lstin{assoc gamma' (gtt_send s t gcs'')}. As in the base case,
    for \lstin{r <> p,q} the association condition holds by \lstin{assoc gamma G},
    and for \lstin{p} and \lstin{q} the condition holds by case analysis.
  \end{itemize}
\end{proof}
\begin{theorem}[Soundness of Association, \path{lemma/correspond.v}, assoc\_soundness] \label{theo-soundness}
  If \lstin{assoc gamma G} and \lstin{gttStepC G G' p q ell}, then there is a local type context \lstin{gamma'},
  a global type tree \lstin{G''} and a message label \lstin{ell'} such that 
  \lstin{gttStepC G G'' p q ell'}, \lstin{assoc gamma' G''} and 
  \lstin{tctxR gamma (lcomm p q ell') gamma'}. 
\end{theorem}
\begin{proof}
  We invert the global type reduction and the association to get
  \lstin{isgPartsC p G},
  \lstin{M.find p gamma = Some (ltt_send q xs)}, and 
  for some \lstin{ell'}, \item \lstin{onth ell' xs <> None}. 
  Then the results follows from applying \cref{lem-sound-lemma} 
\end{proof}
\begin{theorem} [Completeness of Association] \label{theo-completeness}
    Given global type $\G$ and typing context $\Gamma$ such that $\Gamma \assoc \G$. 
    If $\lbltrans{\Gamma}{\lblsync{p}{q}{\ell_k}}{\Gamma'}$  then there exists 
    $\G'$ such that $\Gamma' \assoc \G'$ and $\lbltrans{\G}{\lblsync{p}{q}{\ell_k}}{\G'}$. 
\end{theorem}
\begin{proof}
  We give a sketch of the proof, the details can be found in \cite{LessIsMoreRevisited}.
  The proof is by induction on the typing context reduction $\lbltrans{\G}{\lblsync{p}{q}{\ell}}{\G'}$.
  Because of the shape of the transition label the only rule we care about is \ruleredsync.
  Then the proof proceeds by inverting $\lbltrans{\G}{\lblsync{p}{q}{\ell}}{\G'}$ by \cref{lem-ctx-red-inv}
  and applying case analysis on \cref{sim-inv-assoc}.
  \begin{itemize}
    \item For case \cref{sim-inv-assoc}.\ref{sim-inv-assoc1}, 
    we have $\G=\GvtPair{\pp}{\pq}{\ell_i(S''_i).\G_i}_{i \in I}$.
     Then applying \rulegredsendrec gives 
    $\lbltrans{\G}{\lblsync{p}{q}{\ell_k}}{\G_k}$. 
    Now $\Gamma' \assoc \G_k$ 
    follows from case analysis on participant $\pr \in \dom{\Gamma'}$, applying \cref{lem-inv-assoc}.
    and \cref{lem-not-subj-same} where needed. 
    \item For case \cref{sim-inv-assoc}.\ref{sim-inv-assoc2}, we have $\G=\GvtPair{\prt{s}}{\prt{t}}{\ell_j(S''_j).\G_j}_{j \in J}$ 
    where $\forall j \in J: \Gamma(\pp) \subtp \projfn{\G_j}{\pp},
    \Gamma(\pq) \subtp \projfn{\G_j}{\pq}$ and $\{\pp,\pq\}\cap \{\prt{s},\prt{t}\}=\emptyset$.
    The proof then proceeds by constructing $\Gamma_j$ and $\Gamma'_j$ such that 
    $\Gamma_j \assoc \G_j$ and $\lbltrans{\Gamma_j}{\lblsync{p}{q}{\ell_k}}{\Gamma'_j}$.
    Then the induction hypothesis is applied to create $\G'_j$ such that $\Gamma'_j \assoc \G'_j$.
    Now \rulegredctx gives 
    $\lbltrans{\GvtPair{s}{t}{\ell_j(S''_j).\G_j}_{j \in J}}
    {\lblsync{p}{q}{\ell_k}}
    {\GvtPair{s}{t}{\ell_j(S''_j).\G'_j}_{j \in J}}$. Then the induction hypotheses show that 
    $\Gamma' \assoc \GvtPair{s}{t}{\ell_j(S''_j).\G'_j}_{j \in J}$ as needed.
  \end{itemize}
\end{proof}

\subsection{Properties by Association}
\begin{theorem}[Safety by Association] \label{theo-assoc-safe}
  If $\Gamma \assoc \G$ then $\Gamma$ is safe.
\end{theorem}
\begin{proof}
  We give a sketch of the proof as it's nearly identical to the one in 
  \cite{LessIsMoreRevisited}, (Lem. 26).
  The proof is by considering the set 
  $\varphi =\{\Gamma' \; | \; \exists \G', \Gamma' \assoc \G' \;\text{and}\;
  \G \rightarrow^* \G'
  \}$ and showing that $\varphi$ satisfies the rules \rulesafesync and \rulesafereduce. 
  $\varphi$ satisfies \rulesafesync by \cref{lem-ctx-red-inv} and \cref{lem-inv-assoc},
  and \rulesafereduce by \cref{theo-completeness}.
\end{proof}

We now give a proof of liveness by association, for which we first define some new constructs.
\begin{definition}[Height of a context]
  $\hgt$ is a function on global tree contexts defined as follows:
  \begin{itemize}
    \item $\hgt(\hole)=0$
    \item $\hgt(\GvtPair{p}{q}{\ell_i(S_i).\Gcx_i}_{i \in I})$=$\max(\hgt(\Gcx_i))_{i \in I} +1$
  \end{itemize}
\end{definition}
The height of a global tree context is the maximum number of edges in the path from the root to 
any node of the tree, as commonly defined in literature e.g. \cite{clrs}.

When talking about different graftings of the same tree it's convenient to talk about 
\textit{tree prefixes}.
\begin{definition}[Prefixes of a Global Type Tree Context]  
  For two global type tree contexts $\Gcx_1$ and $\Gcx_2$ that graft the same global type tree,
  we say that $\Gcx_1$ is a proper prefix of $\Gcx_2$ if every leaf in $\Gcx_1$ is in the position of 
  an ancestor of a leaf in $\Gcx_2$.
  We say that $\Gcx_1$ is a prefix if we also allow leaves in $\Gcx_1$ to be in the position of
  leaves of $\Gcx_2$.
\end{definition}


We also extend our global type tree contexts to local type trees.
To avoid confusion with local type contexts we name them local type tree holes.
\begin{definition}[Local type tree holes]
  Local type tree holes are defined inductively with the following syntax   
  \begin{align*}
    \Tcx &::= \quad \ltsend{p}{\ell_i(S_i).\Gcx_i}_{i \in I} 
    \SEP \ltrec{p}{\ell_i(S_i).\Tcx_i}_{i \in I}
   \SEP \hole_i
  \end{align*}
  We also define the grafting of a local tree hole $\Tcx$ by a set of local type trees $\{\T_i\}$,
  analogously to the grafting operation on global tree contexts.
  We say that $\Tcx, \{\T_i\}$ is a $\pp\sendsign$-grafting of $\T$
  if $\Tcx[\T_i]=\T$ and every type in $\{T_i\}$ is of shape 
  $\ltsend{\pp}{...}$. $\pp\andsign$ is defined similarly.
\end{definition}
Local type tree holes enable us to formalise statements such as "$\pp$ necessarily interacts with 
$\pq$ after some amount of steps", which is what we want to prove in liveness. We don't need to 
pose additional balancedness assumptions on local type trees, the balancedness property of 
global type trees assures the graftability of the projected local types, as shown in the following lemma:
\begin{lemma}[Multigrafting Lemma]\label[lemma]{lem-multigraft}
  We have that
  \begin{enumerate}
    \item If $\Gamma \assoc \G$ and $\Gamma \lts{\lblsend{p}{q}{\ell_k}}$ then 
      \begin{enumerate}[label=(\roman*)]
        \item There is a $\pp$-grafting of $\G$, given by $\Gcx_p$ and $\{\G^p_i\}_{i \in I_p}$. 
        \item There is a $\pq$-grafting of $\G$, given by $\Gcx_q$ and $\{\G^q_i\}_{i \in I_q}$.
        \item There is a $\pp\andsign$-grafting of $\Gamma(\pq)$, 
        given by $\Tcx$ and $\{T^q_i\}_{i \in I}$. Furthermore, if $\Tcx=\hole$ then $\Gcx_p=\Gcx_q$,
        and if $\Tcx \neq \hole$ then $\Gcx_q$ is a proper prefix of $\Gcx_p$. 
      \end{enumerate}
    
      \item If $\Gamma \assoc \G$ and $\Gamma \lts{\lblrec{p}{q}{\ell_k}}$ then 
      \begin{enumerate}[label=(\roman*)]
        \item There is a $\pp$-grafting of $\G$, given by $\Gcx_p$ and $\{\G^p_i\}_{i \in I_p}$. 
        \item There is a $\pq$-grafting of $\G$, given by $\Gcx_q$ and $\{\G^q_i\}_{i \in I_q}$.
        \item There is a $\pp\sendsign$-grafting of $\Gamma(\pq)$, 
        given by $\Tcx$ and $\{T^q_i\}_{i \in I}$. Furthermore, if $\Tcx=\hole$ then $\Gcx_p=\Gcx_q$,
        and if $\Tcx \neq \hole$ then $\Gcx_q$ is a proper prefix of $\Gcx_p$. 
      \end{enumerate}
  \end{enumerate}
  
\end{lemma}
\begin{proof}
  We prove only case (1), case (2) is proven similarly.
  Applying \cref{lem-ctx-red-inv} gives that $\Gamma(\pp)=\ltsend{\pq}{...}$. Then inverting 
  $\Gamma \assoc \G$ gives that $\pp \in \funprt(\G)$. Then \cref{lem-grafting}
  gives a global tree context $\Gcx_p$ and $\{\G^p_i\}_{i \in I_p}$ as the $\pp$-grafting of $\G$.
  Furthermore, note that all $G^p_i$ must be of shape $\pp \rightarrow \pq$, as otherwise 
  $\Gamma(p) \subtp \projfn{\G}{\pp}$ wouldn't hold. This gives $\pq \in \funprt(\G)$ and hence 
  there is a $\pq$-grafting $\{\G^q_i\}_{i \in I_q}$ of $\G$. This proves item (i) and (ii).
  
  For item (iii), we first have that $\Gcx_q$ is a prefix of $\Gcx_p$: assume otherwise,
  then we would have that $\Gcx_q$ contains a child of one of the holes in $\Gcx_p$, but we have that
  all the holes in $\Gcx_p$ are filled with $\pp \rightarrow \pq$ types, which would mean 
  that $\pq$ is a participant in $\Gcx_q$, which contradicts \cref{lem-grafting}.
   
  Now (iii) can be proved by bottom-up induction on $\Gcx_p$. Our leaf nodes are $\hole$ which 
  are all filled with $\pp \rightarrow \pq$ types, whose projections to $\pq$ are of the form 
  $\pp\andsign\{...\}$ and hence they can be $\pp\andsign$-grafted with $\Tcx=\hole$.
  For the internal nodes, if the node is $\pq \rdc \pr$ then by \ruleprojout 
  its subtree's projection onto $\pq$ is $\pr \andsign\{\ell_i(S_i).(\projfn{\G_i}{\pq})\}$ 
  where $\T_i$ are the children of the node, hence the projection can be grafted by 
  $\pr \andsign \{\ell_i(S_i).\Tcx_i\}$ where $\Tcx_i$ is the $\pp\andsign$-grafting of child $i$
  that exists by the induction hypothesis. The case when the node is $\pr \sendsign \pq$ is similar.
  If the node doesn't involve $\pq$ then the projection of its subtree on $\pq$ is, 
  by \ruleprojcont 
  and plain merge, equal to the projection of any of its children, which are $\pp\andsign$-graftable
  by the induction hypothesis.
  
  This proves the first statement in (iii). It also proves the second, as we have shown that 
  if the grafting of the tree is not just $\hole$ then there is a node involving $\pq$ in $\Gcx_p$,
  which means that $\Gcx_q$ is a prefix of $\Gcx_p$.
\end{proof}
\begin{theorem}\label{theo-assoc-live}
  If $\Gamma \assoc \G$ then $\Gamma$ is live.
\end{theorem}
\begin{proof}
  We proceed by contradiction. Suppose $\Gamma \assoc \G$ and $\Gamma$ is not live.
  Then there should be a fair path $\Gamma = \Gamma_0 \rdc \Gamma_1 \rdc ..$ and a 
  $k$ such that $\lbltrans{\Gamma_k}{\alpha}$ where 
  $\alpha$ is either (i) $\lblsend{p}{q}{\ell(S)}$ or (ii) $\lblrec{q}{p}{\ell(S)}$
  and for all $k_1 \geq k$, we don't have 
  $\lbltrans{\Gamma_{k_1}}{\lblsync{p}{q}{\ell'}}{\Gamma_{k_1+1}}$.
  We write down the proof only for the send case (i), the proof for (ii) is similar.

  First, by completeness we have that $\Gamma_k \assoc \G_k$ for some $\G_k$.
  Now we have by \cref{lem-ctx-red-inv} and association that $\pp \in \participant{\G_k}$. 
  Therefore, as we did in \cref{theo-soundness}, we can proceed by 
  generalising the statement over $\pp$-graftings of $\G_k$. 
  
  Applying \cref{lem-multigraft} gives $\Gcx_p$ and $\{\G^p_i\}_{i \in I_p}$
  that forms a $\pp$-grafting of $\G_k$,
  $\Gcx_q$ and $\{\G^q_i\}_{i \in I_q}$
  that forms a $\pq$-grafting of $\G_k$,
  and $\Tcx$ and $\{T_j\}_{j \in J}$ that is 
  a $\pp\andsign$-grafting of $\Gamma_k(\pq)$. Now we apply (strong) induction on the height of $\Gcx_p$:
  \begin{enumerate}
    \item $\hgt(\Gcx_p)=0$: Then $\Gcx_p=\hole$, so $\G_k=\GvtPair{p}{q}{...}$.
    Then applying \cref{lem-simul-subproj-inv} gives the existence of 
    $\lbltrans{\Gamma_k}{\lblsend{p}{q}{\ell'(S)}}{}$
    and $\lbltrans{\Gamma_k}{\lblrec{q}{p}{\ell'(S')}}{}$ for some $\ell'$, which gives 
    $\lbltrans{\Gamma_k}{\lblsend{p}{q}{\ell'}}{}$. Hence by fairness we eventually have 
    $\Gamma_{k_1} \lts{\lblsync{p}{q}{\ell'}} \Gamma_{k_1+1}$ for some $k_1 \geq k$, 
    which contradicts our assumption.
    
    \item $\hgt(\Gcx_p)=n+1$: 
    Our induction hypothesis (IH1) is the following: 
    
    Let $\Gcx'$ be the $\pp$-grafting of $\G'$. If  $\hgt(\Gcx') \leq n$, 
      and $\Gamma' \assoc \G'$ for some $\Gamma'$,
      $\Gamma'=\Gamma'_0 \rdc \Gamma'_1..$ is a fair path, 
      and  
      $\Gamma'_{k'_1} \lts{\lblsend{r}{s}{\ell_m(S_m)}}$ for some 
      $r,s,k'_1, m$, then we have 
      $\Gamma'_{k'_2} \lts{\lblsync{r}{s}{\ell_m'}} \Gamma'_{k'_2+1}$ for some $m'$ and $k'_2 > k'1$.  
    
    We once again do induction, this time on the structure of the local tree hole $\Tcx$,
    which is the $\pp\andsign$-grafting of $\Gamma_k(\pq)$.
    
    In the base case, if $\Tcx=\hole$ then $\Gamma_k(\pq)=\ltrec{p}{...}$, so as in the previous case
    applying \cref{lem-simul-subproj-inv} gives 
    $\lbltrans{\Gamma_k}{\lblsync{p}{q}{\ell'}}{}$.
    Then by fairness we eventually have $\Gamma_{k_1} \lts{\lblsync{p}{q}{\ell'}} \Gamma_{k_1+1}$ 
    for some $k_1 \geq k$, which is a contradiction.   
  
    If $\Tcx \neq \hole$, assume without loss of generality that 
    $\Tcx=\ltsend{r}{\ell_i(S_i).\Tcx_i}_{i \in I}$ for some $\pr \neq \pp$ 
    (The proof for the receive case is similar and omitted). 
    Our inductive hypothesis (IH2) is the following:
    
    If for some $k_1$, $\Gamma_{k_1}(\pq)$ is $\pp\andsign$-grafted by $\Tcx_i$, then 
    $\Gamma_{k_2} \lts{\lblsync{p}{q}{\ell'}} \Gamma_{k_2+1}$ for some $k_2 \geq k_1$.

    We have 
    $\Gamma_k \lts{\lblsend{q}{r}{\ell}}$. 
    By \cref{lem-multigraft} (iii) we have that 
    $\Gcx_q$ is a proper prefix of $\Gcx_p$, and hence has a smaller height. 
    Therefore we can apply the induction hypothesis (IH1)
    and conclude that $\Gamma_{k_1} \lts{\lblsync{q}{r}{\ell''}} \Gamma_{k_1+1}$ for some
    $k_1 \geq k$ and $\ell''$. 
    After this transition we have that $\Gamma_{k_1+1}(\pq)$ is $\pp\andsign$-grafted by 
    some $\Tcx_i$. Hence we can apply (IH2)
    and say that $\Gamma_{k_2} \lts{\lblsync{p}{q}{\ell'}} \Gamma_{k_2+1}$ for
    some $\ell', k_2 \geq k_1+1 \geq k$, which gives the desired contradiction.    
  \end{enumerate}
\end{proof}
% raggedbottom means that the bottom of the page is a bit flexible, since the line spacing inside
% the code listing is very rigid and there can be some warnings when they don't properly line up
\raggedbottom