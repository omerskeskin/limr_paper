\section{Conclusion}
\subsection{Summary}
In this work, we have provided a typing system based on coinductive trees (\cref{sec:types}) 
for a simple synchoronous session calculus (\cref{sec-procs}).
We then defined local type contexts, provided a transition system for them,
and defined association between local type contexts and global types. 
In \cref{sec-props} we stated 
the safety and liveness properties for local type contexts based on this transition system,
and proved that these properties hold for associated types.
Then in \cref{sec:proc-props} we gave the typing rules for our session calculus,
formulated a liveness property for sessions, and showed that typeable sessions satisfy this property.
We have also mechanised many of the results in Rocq, with the soundness proof being the most involved
of them.  
\subsection{Future Work}
As far as the Rocq implementation is concerned, an obvious next step would be to actually finish 
the mechanisation. We believe that the definitions we provided are scalable and provide a good foundation
for future formalisation efforts. Beyond the verification of proofs, a full formalisation 
of the results of this paper would allow us to extract Rocq programs that are certifiably live,
which could have practical applications in error-sensitive settings. An example of this 
approach is the Zooid DSL \cite{zooid}.

One other possible future point of interest would be to extend our formalisation to use full merge instead of
plain merge. Time and again, proofs of properties such as subject reduction in settings
using full merge were shown to be flawed \cite{LessIsMore}, and formal verification of such proofs
would aid future research greatly. 

Another avenue for future efforts would be to investigate different flavours of fairness and 
liveness using the "association" framework we employed. There are dozens of fairness 
properties in the wider concurrency theory literature \cite{fairness}, with different fairness 
assumptions being useful for different liveness properties \cite{fairnesslock}. It could be interesting
to approach these notions of fairness from a MPST point of view and show how a specific typing discipline
relates to these properties. In particular, it would be nice to see variations on the balancedness (\cref{def-balance})
property we imposed on our type trees and see which of our results still hold, and what liveness
properties the new balancedness assumptions relate to.
