\section{Conclusion and Related Work}

\textbf{Liveness Properties.} Examinations of liveness, also called \textit{lock-freedom}, guarantees of multiparty session types abound in literature, e.g. 
\cite{padovani_typing_2014,kobayashi_type_2002,LessIsMoreRevisited,LessIsMore,barbanera_partially_2023}.
Most of these papers use the definition liveness proposed by Padovani \cite{padovani}, 
which doesn't make the fairness assumptions that characterize the property \cite{francez_fairness_1986} explicit. 
Contrastingly, van Glabbeek et. al. \cite{fairnesslock} examine several notions of fairness and the liveness properties
induced by them, and devise a type system with flexible choices \cite{castellani_reversible_2019} that captures
the strongest of these properties, the one induced by the \textit{justness} \cite{fairness} assumption.
In their terminology, \cref{def-live-sess}
corresponds to liveness under strong fairness of transitions (ST), 
which is the weakest of the properties considered in that paper. They also show that their 
type system is complete i.e. every live process can be typed. We haven't presented any completeness results 
in this paper. Indeed, our type system is not complete for \cref{def-live-sess}, even if we restrict our attention to safe and race-free sessions. 
For example, the session described in \cite[Example 9]{fairnesslock}
is live but not typable by a context associated with a balanced global type in our system.

Fairness assumptions are also made explicit in recent work by Ciccone et. al \cite{ciccone_fair_2024,ciccone_2022-binary} 
which use generalized inference systems with coaxioms \cite{ancona_generalizing_2017} to characterize 
\textit{fair termination}, which is stronger than \cref{def-live-sess}, but enjoys good composition properties.

\textbf{Mechanisation.} Mechanisation of session types in proof assistants is a relatively new effort. 
Our formalisation is built on recent work by Ekici et. al. \cite{srpaper} which uses a coinductive representation
of global and local types to prove subject reduction and progress. 
Their work uses a typing relation between global types and sessions while ours uses one 
between associated local type contexts and sessions. This necessiates the rewriting of
subject reduction and progress proofs in addition to the operational correspondence, safety and liveness
properties we have proved. Other recent results mechanised in Rocq include Ekici and Yoshida's \cite{ekici_completeness_2024}
work on the completeness of asynchronous subtyping, and Tirore's work \cite{tirore_thesis, tirore2025multiparty,tirore2023sound}
on projections and subject reduction for $\pi$-calculus. 

Castro-Perez et. al. \cite{castro2026synthetic} devise a multiparty session type system 
that dispenses with projections and local types by defining the typing relation directly on the LTS
specifying the global protocol, and formalise the results in Agda. Ciccone's PhD thesis \cite{ciccone2023concertogrossosessionsfair}
presents an Agda formalisation of fair termination for binary session types. 
Binary session types were also implemented in Agda by Thiemann \cite{thiemann2019} and in Idris by Brady\cite{brady_type-driven_2017}. Several implementations of
binary session types are also present for Haskell \cite{dardha2021,lindley2016embedding,pucella2008haskell}. 

Implementations of session types that are more geared towards practical verification include the 
Actris framework \cite{hinrichsen2019actris,jacobs2024deadlock} which enriches the 
seperation logic of Iris \cite{jung2018iris} with binary session types to certify deadlock-freedom.
In general, verification of liveness properties, with or without session types, in concurrent seperation logic is an active research area
that has produced tools such as TaDa \cite{gardner2021tada}, FOS \cite{lee2023fos} and LiLo \cite{lee2025lilo}
in the past few years. Further verification tools employing multiparty session types are 
Jacobs's Multiparty GV \cite{jacobs2024deadlock} based on the functional language of Wadler's GV \cite{wadler2012gv}, and Castro-Perez et. al's Zooid \cite{zooid},
which supports the extraction of certifiably safe and live protocols.

