Multiparty session types (MPST) offer a framework for the description of communication-based
protocols involving multiple participants. In the \textit{top-down} approach to MPST, 
the communication pattern of the session is described using a \textit{global type}. Then 
the global type is \textit{projected} on to a \textit{local type} for each participant,
and the individual processes making up the session are type-checked against these projections.
Typed sessions possess certain desirable properties such as \textit{safety}, \textit{deadlock-freedom} and 
\textit{liveness} (also called \textit{lock-freedom}).

In this work, we present the first mechanised proof of liveness
for synchronous multiparty session types in the Rocq Proof Assistant. 
Building on recent work, we represent global and local types
as coinductive trees using the paco library. We use a coinductively defined \textit{subtyping} relation 
on local types together with another coinductively defined \textit{plain-merge} projection
relation relating local and global types .
We then \textit{associate} collections of local types, or \textit{local type contexts}, with 
global types using this projection and subtyping relations, and prove an \textit{operational correspondence}
between a local type context and its associated global type. We then utilize this
association relation to prove the safety and liveness of associated local type contexts
and, consequently, the multiparty sessions typed by these contexts.  

Besides clarifying the often informal proofs of liveness found in the MPST literature,
our Rocq mechanisation also enables the certification of lock-freedom properties
of communication protocols. Our contribution amounts to around 12K lines of Rocq code.