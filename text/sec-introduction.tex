\section{Introduction}
\label{sec:introduction}
Multiparty session types \cite{honda2008} provide a type discipline for the correct-by-construction specification of message-passing protocols.
Desirable protocol properties guaranteed by session types include \textit{communication safety} 
(the labels and types of senders' payloads cohere with the capabilities of the receivers),
\textit{deadlock-freedom} (also called \textit{progress} or \textit{non-stuck property} \cite{srpaper}) 
(it is possible for the session to progress so long as it has at least one active participant), and 
\textit{liveness} (also called \textit{lock-freedom} \cite{fairnesslock} or \textit{starvation-freedom} \cite{castro2026synthetic})   
(if a process is waiting to send and receive then a communication involving it eventually happens).

There exists two common methodologies for multiparty session types. 
In the \textit{bottom-up} approach, the individual processes making up the session are typed using 
a collection of \textit{participants} and \textit{local types}, that is, a \textit{local type context}, and the properties of the session is examined by model-checking
this local type context. Contrastingly, in the \textit{top-down} approach sessions are typed by a \textit{global type}
that is related to the processes using endpoint \textit{projections} and \textit{subtyping}.
The structure of the global type ensures that the desired properties are satisfied by the session.
These two approaches have their advantages and disadvantages: the bottom-up approach is generally 
able to type more sessions, while type-checking and 
type-inferring in the top-down approach tend to be more efficient than model-checking the bottom-up system \cite{projsurvey}.  

In this work, we present the Rocq \cite{coq} formalisation of a 
synchronous MPST that that ensures the aforementioned properties
for typed sessions. Our type system uses an \textit{association} relation ($\assoc$) \cite{LessIsMoreRevisited, Pischke2026}
defined using (coinductive plain) projection \cite{tirore2023sound} and subtyping, in order to relate 
local type contexts and global types. This association relation ensures \textit{operational correspondence}
between the labelled transition system (LTS) semantics we define for local type contexts and global types. We then type ($\vdash_\mathcal{M}$) sessions using local type contexts that are associated with
global types, which ensure that the local type context, and hence the session, is well-behaved in some sense.
Whenever an associated local type context $\Gamma$ types a session $\M$, our type system guarantees 
safety (\cref{theo-type-safe}), deadlock-freedom \cref{theo-progress} and liveness \cref{theo-sess-live}.
To our knowledge, this work presents the first mechanisation of liveness for multiparty session types in
a proof assistant.



% https://tikzcd.yichuanshen.de/#N4Igdg9gJgpgziAXAbVABwnAlgFyxMJZABgBpiBdUkANwEMAbAVxiRAHEQBfU9TXfIRRkAjFVqMWbACrdeIDNjwEiZAEzj6zVohAAFOXyWCiI0mOpapugDo3gACjQB9LKWmuAlAAIAPt6xvOywwbwBJOy4AXjt2OgBbeLpDBX5lIWQ1c01JHRBYhKSAchTFARUUMw1LXLYAWVK0kxQs6oltepKeI3KMrMoajt12LvEYKABzeCJQADMAJwh4pAB2ahwIJAAWQesQOjg4CABjFIWl7fXNxAA2XbycAE80OHgEbpBz5cQyEA2kMztPZoRYAKzOi2+gP+iCyQIezxBJwhFx+VyQAGZ7mwDkdTh8vgD0YgMQTIUg4TCAKxk1FYv7XKnY3RPF5vFHfX4wtbwth2eZYCYACxwdHmiwA7hzMcSdrzbDYBcLReKIFLad8mQykHd5flFYKRWLJdwKFwgA
\begin{figure}
\begin{tikzcd}
\G \arrow[d, "\projf", dotted] \arrow[rd, "\assoc"] \arrow[rr, "\rightarrow",dotted] &   & \G' \arrow[d, "\assoc"]        \\
\T \arrow[d, "\vdash_{\mathsf{P}}", dotted] \arrow[r]                                  & {\{(p_i,T_i) \; |\; i \in I\}=\Gamma} \arrow[d, "\vdash_\M"] \arrow[r, "\rightarrow"] & \Gamma' \arrow[d, "\vdash_M"] \\
\mathsf{P} \arrow[r]                                                       & {\prod_{i}^{} \pp_i \triangleleft \mathsf{P}_i = \M} \arrow[r, "\rightarrow"]                                                     & \M'                          
\end{tikzcd}
\caption{Design overview. The dotted lines correspond to relations inherited from \cite{srpaper}
while the solid lines denote relations that are new, or substantially rewritten, in this paper.}
\end{figure}

Our Rocq implementation builds upon the recent formalisation of subject reduction for MPST by Ekici et. al. \cite{srpaper},
which itself is based on \cite{SynchronousSubtyping}.
The methodology in \cite{srpaper} takes an equirecursive approach where an inductive syntactic global or local type is identified 
with the coinductive tree obtained by fully unfolding the recursion. 
It then defines a coinductive projection relation between global and local type trees,
the LTS semantics for global type trees, and typing rules for the session calculus
outlined in \cite{SynchronousSubtyping}. 
We extensively use these definitions and the lemmas concerning them, but we still depart from and extend 
\cite{srpaper} in numerous ways by introducing local typing contexts, their correspondence with global types and 
a new typing relation. Our addition to the code amounts to around 12K lines of Rocq code.

As with \cite{srpaper}, our implementation heavily uses the parameterized coinduction technique 
of the paco \cite{paco} library. Namely, our liveness property is defined using 
possibly infinite \textit{execution traces} which we represent as coinductive streams.
The relevant predicates on these traces, such as fairness, are then defined as mixed inductive-coinductive 
predicates using linear temporal logic (LTL)\cite{pnueli1977temporal}.
This approach, together with the proof techniques provided by paco,
results in compositional and clear proofs.

\textbf{Outline.} In \cref{sec-procs} we define our session calculus and its LTS semantics. 
In \cref{sec:types} we recapitulate the definitions of local and global type trees, and the subtyping and projection relations on them, from \cite{srpaper}. 
In \cref{sec:lts} we give LTS
semantics to local type contexts and global types, and detail the association relation between them.
In \cref{sec-props} we define safety and liveness for local type contexts, and prove that they hold
for contexts associated with a global type tree. In \cref{sec-proc-props} we give the typing rules for 
our session calculus, and prove the desired properties of these typable sessions.