\section{Introduction}
\label{sec:introduction}
\todo{Session types introduction}

\todo{Liveness introduction}

% https://tikzcd.yichuanshen.de/#N4Igdg9gJgpgziAXAbVABwnAlgFyxMJZABgBpiBdUkANwEMAbAVxiRAHEQBfU9TXfIRRkAjFVqMWbACrdeIDNjwEiZAEzj6zVohAAFOXyWCiI0mOpapugDo3gACjQB9LKWmuAlAAIAPt6xvOywwbwBJOy4AXjt2OgBbeLpDBX5lIWQ1c01JHRBYhKSAchTFARUUMw1LXLYAWVK0kxQs6oltepKeI3KMrMoajt12LvEYKABzeCJQADMAJwh4pAB2ahwIJAAWQesQOjg4CABjFIWl7fXNxAA2XbycAE80OHgEbpBz5cQyEA2kMztPZoRYAKzOi2+gP+iCyQIezxBJwhFx+VyQAGZ7mwDkdTh8vgD0YgMQTIUg4TCAKxk1FYv7XKnY3RPF5vFHfX4wtbwth2eZYCYACxwdHmiwA7hzMcSdrzbDYBcLReKIFLad8mQykHd5flFYKRWLJdwKFwgA
\begin{figure}
\begin{tikzcd}
\G \arrow[d, "\projf", dotted] \arrow[rd, "\assoc"] \arrow[rr, "\rightarrow",dotted] &   & \G' \arrow[d, "\assoc"]        \\
\T \arrow[d, "\vdash_{\mathsf{P}}", dotted] \arrow[r]                                  & {\{(p_i,T_i) \; |\; i \in I\}=\Gamma} \arrow[d, "\vdash_\M"] \arrow[r, "\rightarrow"] & \Gamma' \arrow[d, "\vdash_M"] \\
\mathsf{P} \arrow[r]                                                       & {\prod_{i}^{} \pp_i \triangleleft \mathsf{P}_i = \M} \arrow[r, "\rightarrow"]                                                     & \M'                          
\end{tikzcd}
\caption{Design overview. The dotted lines correspond to relations inherited from \cite{srpaper}
while the solid lines denote relations that are new, or substantially rewritten, in this paper.}
\end{figure}

In this work we present the Rocq formalisation of a session type system for a
simple session calculus, and prove that sessions typable in this system are \textit{safe},
\textit{deadlock-free}, and \textit{live}.
The approach we take in our type system is very similar to the one followed by Hou and Yoshida in \cite{LessIsMoreRevisited}.
Namely, we proceed by defining local and global type trees, and relate them using projections. 
We then extend this projection relation to an \textit{association} relation between local type contexts i.e. 
collections of local types paired
with participants, and global type trees. 
Next we give LTS semantics to local type contexts and global type trees, and prove an operational correspondence between them.
We then proceed to formulate safety and liveness properties for local type contexts, and show that 
local type contexts associated with global type trees enjoy these properties. We
relate associated local type contexts to sessions via typing rules, and demonstrate an operational correspondence 
between contexts and sessions via \textit{subject reduction}, \textit{progress} and 
\textit{session fidelity} theorems. Finally we show, using the liveness properties we defined 
on local type contexts, that typable sessions are live.


Our Rocq implementation builds upon the recent formalisation of subject reduction for MPST by Ekici et. al. \cite{srpaper},
which itself is based on \cite{SynchronousSubtyping}.
The methodology in \cite{srpaper} takes an equirecursive approach where an inductive syntactic global or local type is identified 
with the coinductive tree obtained by fully unfolding the recursion. 
It then defines a coinductive projection relation between global and local type trees,
the LTS semantics for global type trees, and typing rules for the session calculus
outlined in \cite{SynchronousSubtyping}. 
We extensively make use of these definitions and the lemmas concerning them.

\todo{specifics of the project}

\textbf{Outline.} In \cref{sec-procs} we define our session calculus and its LTS semantics. 
In \cref{sec:types} we introduce local and global type trees. In \cref{sec:lts} we give LTS
semantics to local type contexts and global types, and detail the association relation between them.
In \cref{sec-props} we define safety and liveness for local type contexts, and prove that they hold
for contexts associated with a global type tree. In \cref{sec-proc-props} we give the typing rules for 
our session calculus, and prove \textit{non-stuck} and \textit{liveness} properties for typable sessions.