\section{Properties of Sessions}\label{sec-proc-props}

We give typing rules for the session calculus introduced in \ref{sec-procs}, and prove subject 
reduction and progress for them. Then we define 
a liveness property for sessions, and show that processes typable by a local type context 
that's associated with a global type tree are guaranteed to satisfy this liveness property.
\subsection{Typing rules}
We give typing rules for our session calculus based on \cite{SynchronousSubtyping} and \cite{srpaper}. 

We distinguish between two kinds of typing judgements and type contexts.
\newcommand{\vdashm}{\vdash_\M}
\newcommand{\vdashp}{\vdash_\PT}
\begin{enumerate}
    \item A local type context $\Gamma$ associates participants with local type trees, as defined in
    c{def-type-ctx}. Local type contexts are used to type sessions (\cref{def:sessions}) i.e. a set of 
    pairs of participants and single processes composed in parallel. We express 
    such judgements as $\Gamma \vdashm \M$, or as \lstin{typ_sess M gamma} or 
    \lstin{gamma $\;\vdash$ M} in Rocq.
    \item A process variable context $\Theta_\T$ associates process variables with local type trees,
    and an expression variable context $\Theta_\kf{e}$ 
    assigns sorts to expresion variables.
    Variable contexts are used to type single processes and expressions (\cref{def:processes}). 
    Such judgements are expressed 
    as $\Theta_\T, \Theta_\kf{e} \vdashp \PT : \T$, or in Rocq as 
    \lstin{typ_proc theta_T theta_e P T} or \lstin{theta_T, theta_e $\;\vdash$ P : T}.
\end{enumerate}

\begin{table}[h]
{\footnotesize
\[
\begin{array}{@{}l@{}}
  \inferrule%[\rulename{steq}]
  {}
  {\Theta \vdashp n\colon \texttt{nat}\qquad \Theta \vdashp i\colon \texttt{int}
  \qquad \Theta \vdashp \mathtt{true}\colon \texttt{bool}\qquad \Theta \vdashp \mathtt{false}\colon \texttt{bool}\qquad \Theta,x\colon\ST \vdashp x\colon \ST}
      \\[3mm]
     \inferrule%[\rulename{steq}]
  {\Theta \vdashp e\colon \texttt{nat}}
  {\Theta \vdashp \mathtt{succ}\ e\colon \texttt{nat}}
      \quad
     \inferrule%[\rulename{steq}]
  {\Theta \vdashp e\colon \texttt{int}}
  {\Theta \vdashp \mathtt{neg}\ e\colon \texttt{int}}
  \qquad
     \inferrule%[\rulename{steq}]
  {\Theta \vdashp e\colon \texttt{bool}}
  {\Theta \vdashp \neg\ e\colon \texttt{bool}}
      \\[3mm]
     \inferrule%[\rulename{steq}]
  {\Theta \vdashp e_1\colon \ST \quad \Theta \vdashp e_2\colon \ST}
  {\Theta \vdashp  e_1\oplus e_2\colon \ST}
  \qquad
     \inferrule%[\rulename{steq}
  {\Theta \vdashp e_1\colon \texttt{int} \quad \Theta \vdashp e_2\colon \texttt{int}}
  {\Theta \vdashp \ e_1 > e_2\colon \texttt{bool}}
  \qquad
     \inferrule%[\rulename{steq}]
  {\Theta \vdashp e\colon \ST \quad \ST \subso \ST'}
  {\Theta \vdashp e\colon \ST'}
\end{array}
\]}
\caption{Typing expressions}
\label{tbl:expr}
\end{table}
\begin{table}[h]
{\footnotesize
\[
\begin{array}{@{}l@{}}
  \inferrule[\rulename{t-end}]
  {}
  {\Theta \vdashp \inact \colon \tend}
\quad
  \inferrule[\rulename{t-var}]
  {}
  {\Theta,\Xv\colon\T \vdashp \Xv\colon\T}
 \quad
  \inferrule[\rulename{t-rec}]
  {\Theta,\Xv\colon\T \vdashp \PT\colon \T}
  {\Theta \vdashp \mu\Xv.\PT\colon\T}
 \quad
  \inferrule[\rulename{t-if}]
  {\Theta\vdashp e\colon \texttt{bool} \quad \Theta \vdashp \PT_1\colon \T  \quad \Theta \vdashp \PT_2\colon \T}
  {\Theta \vdashp \texttt{if}\ \kf{e} \ \mathtt{then}\ \PT_1  \ \mathtt{else} \ \PT_2\colon \T}
      \\[4mm]
  \inferrule[\rulename{t-sub}]
  {\Theta \vdashp \PT\colon \T \quad \T\leqslant \T'}
  {\Theta \vdashp \PT\colon \T'}
  \quad
   \inferrule[\rulename{t-in}]
  {\forall i\in I,\quad \Theta,x_i\colon\ST_i \vdashp \PT_i\colon \T_i}
  {\Theta \vdashp \sum_{i \in I} \prt{p} ?\ell_i(x_i).\PT_i\colon  
  \ltrec{\pp}{\ell_i(\ST_i).\T_i}_{i \in I}}
 \quad
   \inferrule[\rulename{t-out}]
  {\Theta \vdashp e\colon\ST \quad \Theta \vdashp \PT\colon \T}
  {\Theta \vdashp \prt{p}!\ell(\kf{e}).\PT \ \colon  \ {\pp}\sendsign\{{\ell(\ST).\T}\}}
\end{array}
\]}
\caption{Typing processes}
\label{tbl:proc}
\end{table}
\cref{tbl:expr} and \cref{tbl:proc} state the standard typing rules for expressions and processes which we don't elaborate on.
We have a single rule for typing sessions:
\[
\inferrule[\rulename{t-sess}]{\forall i \in I: \qquad \vdashp \PT_i \colon \Gamma(\pp_i) \qquad 
\Gamma \assoc \G}
{\Gamma \vdashm \prod_i \pp_i \triangleleft \PT_i}
\]
\rulename{t-sess} says that a session made of the parallel composition of processes $\prod_i \pp_i \triangleleft \PT_i$ 
can be typed by an associated local context $\Gamma$ if the local type of participant $\pp_i$ in $\Gamma$
types the process  
\subsection{Subject Reduction, Progress and Session Fidelity}
The subject reduction, progress and non-stuck theorems from \cite{srpaper} \todo{give theorem no}
also hold in this setting, with minor changes in their statements and proofs. 
We won't discuss these proofs in detail.
\begin{lemma}
  \label{lem-typ-unfold}
  If \lstin{gamma} $\vdashm$ \lstin{M}   and \lstin{M} $\Rrightarrow$ \lstin{M'} then \lstin{typ_sess M' gamma}.
\end{lemma}
\begin{proof}
  By induction on \lstin{unfoldP M M'}.
\end{proof}
\begin{theorem}[Subject Reduction]\label{theo-sub-red}
If \lstin{gamma} $\vdashm$ \lstin{M}   and 
\lstin{M} $\lts{\lblsync{p}{q}{\ell}}$ \lstin{M'}, then 
there exists a typing context \lstin{gamma'} such that 
\lstin{gamma} $\lts{\lblsync{p}{q}{\ell}}$ \lstin{gamma'}
and \lstin{gamma} $\vdashm$ \lstin{M}  .
\end{theorem}
\begin{theorem}[Progress] \label{theo-progress}
If \lstin{gamma} $\vdashm$ \lstin{M}  , one of the following hold :
\begin{enumerate}
  \item Either \lstin{M} $\Rrightarrow$ \lstin{M_inact} where every process making up \lstin{M_inact}
  is inactive, i.e. 
  \lstin{M_inact} $\equiv \prod_{i=1}^{n}\pp_i \triangleleft \bf{0}$ for some $n$. 
  \item Or there is a \lstin{M'} such that \lstin{M} $\rdc$ \lstin{M'}.
\end{enumerate}
\end{theorem}
\begin{remark}\label{remark-reactive-justif}
Note that in \cref{theo-sub-red} one transition between sessions corresponds to exactly one transition
between local type contexts with the same label. That is, every session transition is observed by the corresponding type.
This is the main reason for our choice of 
reactive semantics (\cref{def-sess-semantics}) as $\tau$ transitions are not observed by the type in
ordinary semantics. In other words, with $\tau$-semantics the typing relation is a \textit{weak simulation} \cite{weakbisim},
while it turns into a strong simulation with reactive semantics. For our Rocq implementation 
working with the strong simulation turns out be more convenient. 
\end{remark}
We can also prove the following correspondence result in the reverse direction to \cref{theo-sub-red}, analogus 
to \cref{theo-soundness}.
\begin{theorem}[Session Fidelity]\label{theo-sess-fid}
If \lstin{gamma} $\vdashm$ \lstin{M} and \lstin{gamma} $\lts{\lblsync{p}{q}{\ell}}$ \lstin{gamma'}, there exists a
message label
$\ell'$ and a session \lstin{M'} such that \lstin{M} $\lts{\lblsync{p}{q}{\ell'}}$ \lstin{M'} 
and \lstin{typ_sess M' gamma'}.  
\end{theorem}
\begin{proof}
By inverting the local type context transition and the typing.
\end{proof}
\begin{remark}
Again we note that by \cref{theo-sess-fid} a single-step context reduction induces 
a single-step session reduction on the type. 
With the $\tau$-semantics the session reduction induced by 
the context reduction would be multistep.
\end{remark}
Now the following type safety property follows from the above theorems:
\begin{theorem}[Type Safety]\label{theo-type-safe}
If \lstin{gamma} $\vdashm$ \lstin{M} and  \lstin{M} $\lts{}^*$ \lstin{M'} $\Rrightarrow$ 
\lstin{p <- p_send q ell P \|\|\| q <- p_recv p xs \|\|\| M''}, then \lstin{onth ell xs <> None}.
\end{theorem} 
\begin{proof}
  \todo{do the proof}
\end{proof}
\subsection{Session Liveness}
We state the liveness property we are interested in proving,
and show that typable sessions have this property.
\begin{definition}[Session Liveness]\label[definition]{def-live-sess}
  Session $\M$ is live iff
  \begin{enumerate}
    \item ${\M}\longrightarrow^* {\M'} \Rrightarrow \pq \triangleleft \tout{\pp}{\ell_i}{x_i}.Q \;|\; \N$ implies 
    ${\M'}\longrightarrow^*{\M''} \Rrightarrow \pq \triangleleft Q \;|\; \N'$ for some $\M'', \N'$
    
    \item ${\M}\longrightarrow^* {\M'} \Rrightarrow \pq \triangleleft \texternal \tin{\pp}{\ell_i}{x_i}.Q_i \;|\; \N$ implies 
    ${\M'}\longrightarrow^*{\M''} \Rrightarrow \pq \triangleleft Q_i[v/x_i] \;|\; \N'$ for some $\M'', \N', i, v.$ 
  \end{enumerate}
  In Rocq we express this with the following:
  \begin{tcb}{Rocq}
Definition live_sess Mp := forall M, betaRtc Mp M -> 
  (forall p q ell e P' M', p <>q -> unfoldP M ( (p <-- p_send q ell e P') \|\|\| M') -> exists M'',
  betaRtc M ((p <-- P')\|\|\|M''))
  /\
  (forall p  q llp M', p <>q -> unfoldP M ( (p <-- p_recv q llp) \|\|\| M') -> 
    exists M'' P' e k, onth k llp = Some P' /\ betaRtc M ((p <-- subst_expr_proc P' e 0 0)\|\|\|M'')). 
  \end{tcb}
\end{definition}
Session liveness, analogous to liveness for typing contexts (\cref{def-fair-live}), says that 
when $\M$ is live, if $\M$ reduces to a session $\M'$ containing a participant that's attempting to send 
or receive, then $\M'$ reduces to a session where that communication has happened. 
It's also called \textit{lock-freedom} in related work (\cite{fairnesslock,padovani}). 

We now prove that typed sessions are live. Our proof follows the following steps:
\begin{enumerate}
  \item Formulate a "fairness" property for typable sessions,
  with the property that any finite session reduction path can be extended to a fair
  session reduction path.  
  \item Lift the typing relation to reduction paths, and show that fair session reduction paths
  are typed by fair local type context reduction paths.
  \item Prove that a certain transition eventually happens in the local context reduction 
  path, and that this means the desired transition is enabled in the session reduction path.   
\end{enumerate}
We first state a "fairness" (the reason for the quotes is explained in \cref{remark-fairness-session})
property for session reduction paths, analogous to fairness 
for local type context reduction paths (\cref{def-fair-live}).  
\begin{definition}["Fairness" of Sessions]\label{def-fair-sess}
We say that a $(\pp,\pq)\ell$ transition is enabled at $\M$ if
$\M \lts{(\pp,\pq)\ell} \M'$ for some $\M'$. A session reduction path is fair if the following LTL property holds:
\begin{align*}
  \square(\mathtt{enabledComm}_{\prt{p},\prt{q},\ell} \implies 
  \lozenge(\mathtt{headComm}_{\prt{p},\prt{q}}))
  \end{align*}
\end{definition}
\begin{remark}\label{remark-fairness-session}
\cref{def-fair-sess} is not actually a sensible fairness property for our reactive semantics, 
mainly because it doesn't satisfy 
the \textit{feasibility} \cite{fairness} property stating that any finite execution can be extended 
to a fair execution. Consider the following session:
\begin{align*}
  \M = \pp \triangleleft \kf{if} (\kf{true} \oplus \kf{false}) \; \kf{then} \;  \pq!\ell_1(true) 
  \; \kf{else} \;
  \pr!\ell_2(true).\bf{0} \mid \pq \triangleleft \pp?\ell_1(x).0  \mid \pr \triangleleft \pp?\ell_2(x).0
\end{align*}
We have that $\M \lts{(\pp,\pq)\ell_1} \M'$ where $\M'=\pp \triangleleft \bf{0}
 \mid \pq \triangleleft \bf{0}  \mid \pr \triangleleft \pp?\ell_2(x).0
$, 
and  also $\M \lts{(\pp,\pr)\ell_2} \M''$ for another $\M''$. Now consider the reduction path 
$\rho = \M \lts{(\pp,\pq)\ell_1} \M'$. $(\pp,\pr)\ell_2$ is enabled at $\M$ so in a fair path it should eventually 
be executed, however no extension of $\rho$ can contain such a transition as $\M'$ has no remaining transitions.
%With $\tau$ semantics we wouldn't have $(\pp,\pq)\ell_1$ and $(\pp,\pr)\ell_2$ transitions at the same state, 
%as a $\pp$ would have to make a choice via a $\tau$ transition before a communication can be enabled.
Nevertheless, it turns out that there is a fair reduction path starting from every
typable session (\cref{lem-fair-scheduling}), and this will be enough to prove our desired liveness property.
\end{remark}
We can now lift the typing relation to reduction paths, just like we did in \cref{def-assoc-path}. 
\begin{definition}[Path Typing]
\label{def-typ-path}
Path typing is a relation between session reduction paths and local type context reduction paths,
defined coinductively by the following rules:
\begin{enumerate}[(i)]
  \item The empty session reductoin path is typed with the empty context reduction path.
  \item If $\M \lts{\lambda_0} \rho$ is typed by $\Gamma \lts{\lambda_1} \rho'$ 
  where ($\rho$ and $\rho'$ are session and local type context reduction paths, respectively), 
  then $\lambda_0=\lambda_1$ and $\rho$ is typed by $\rho'$.
\end{enumerate}
\end{definition}
Similar to \cref{lem-path-assoc-exists}, we can show that if the head of the path is typable
then so is the whole path.
\begin{lemma}\label{lem-typ-path-exists}
If \lstin{typ_sess M gamma}, then any session reduction path \lstin{xs} starting with \lstin{M}
is typed by a local context reduction path \lstin{ys} starting with \lstin{gamma}.
\end{lemma} 
\begin{proof}
We can construct a local context reduction path that types the session path. The construction
exactly like \cref{lem-path-assoc-exists} but elements of the output stream are generated by 
\cref{theo-sub-red} instead of \cref{theo-completeness}.
\end{proof}
We also have that typing path preserves fairness.
\begin{lemma}\label{lem-typ-path-fair}
If session path \lstin{xs} is typed by the local context path \lstin{ys}, and \lstin{xs} is fair, then so 
is \lstin{ys}.   
\end{lemma}
The final lemma we need in order to prove liveness is that there exists a fair reduction path from
every typable session.
\begin{lemma}[Fair Path Existence]\label{lem-fair-scheduling}
If \lstin{typ_sess M gamma}, then there is a fair session reduction path \lstin{xs} starting from
\lstin{M}.
\end{lemma}
\begin{proof}
We can construct a fair path starting from \lstin{M} by repeatedly cycling through all participants,
checking if there is a transition involving that participant, and executing that transition if there is.   
\end{proof}
\begin{remark}
The Rocq implementation of \cref{lem-fair-scheduling} computes a \lstin{CoFixpoint} corresponding to 
the fair path constructed above. As in \cref{lem-path-assoc-exists}, we use \lstin{constructive_indefinite_description}
to turn existence statements in \lstin{Prop} to dependent pairs. We also assume the informative law of excluded middle (\lstin{excluded_middle_informative})
in order to carry out the "check if there is a transition" step in the algorithm above. 
When proving that the constructed path is fair, we sometimes rely on the LTL constructs we outlined in \cref{subsec-ltl}
reminiscent of the techniques employed in \cite{coqfilter}.
\end{remark}
We can now prove that typed sessions are live.
\begin{theorem}[Liveness by Typing]\label{theo-sess-live}
For a session \lstin{Mp}, if \lstin{exists gamma gamma} $\vdashm$ \lstin{Mp} then \lstin{live_sess Mp}.
\end{theorem}
\begin{proof}
We detail the proof for the send case of \cref{def-live-sess}, the case for the receive is similar.
Suppose that \lstin{Mp} $\rdc^*$ \lstin{M} and 
\lstin{M} $\Rrightarrow$ \lstin{((p <- p_send q ell e P') \|\|\| M')}.
Our goal is to show that there exists a \lstin{M''} such that
\lstin{M} $\rdc^*$ \lstin{((p <- P')\|\|\|M'')}. 
First, observe that by \rulename{R-Unfold} it suffices to show that 
\lstin{((p <- p_send q ell e P') \|\|\| M')} $\rdc^*$ \lstin{M''}
for some \lstin{M''}. Also note that  \lstin{gamma} $\vdashm$ \lstin{M} for some \lstin{gamma}
by \cref{theo-sub-red}, therefore 
\lstin{gamma} $\vdashm$ \lstin{((p <- p_send q ell e P') \|\|\| M')}
by \cref{lem-typ-unfold}.

Now let \lstin{xs} be a fair reduction path starting from \lstin{((p <-- p_send q ell e P') \|\|\| M')}, 
which exists by \cref{lem-fair-scheduling}. Let \lstin{ys} be the
local context reduction path starting with \lstin{gamma} that types \lstin{xs}, which exists by 
\cref{lem-typ-path-exists}. Now \lstin{ys} is fair by \cref{lem-typ-path-fair}.
Therefore by \cref{theo-ctx-live} \lstin{ys} is live, 
so a \lstin{lcomm p q ell'} transition
eventually occurs in \lstin{ys} for some \lstin{ell'}.
Therefore  
\lstin{ys} = \lstin{gamma} $\rdc^*$ \lstin{gamma_0} $\lts{(\pp,\pq)\ell'}$ \lstin{gamma_1} $\rdc ..$
for some \lstin{gamma_0}, \lstin{gamma_1}. Now consider the session \lstin{M_0} typed by 
\lstin{gamma_0} in \lstin{xs}. 
We have \lstin{((p <- p_send q ell e P') \|\|\| M'')} $\rdc^*$ \lstin{M_0}
by \lstin{M_0} being on \lstin{xs}. 
We also have that 
\lstin{M_0} $\lts{(\pp,\pq)\ell''}$ \lstin{M_1} for some $\ell''$, \lstin{M_1} by \cref{theo-sess-fid}.
Now observe that \lstin{M_0} $\equiv$ \lstin{((p <- p_send q ell e P') \|\|\| M'')} for some 
\lstin{M''} as no transitions involving \lstin{p} have happened on the reduction path to \lstin{M_0}.
Therefore $\ell=\ell''$, so \lstin{M_1} $\equiv$ \lstin{((p <- P') \|\|\| M'')} for some \lstin{M''}, as needed.
\end{proof}