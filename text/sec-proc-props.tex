\section{Properties of Sessions}\label{sec-proc-props}

We give typing rules for the session calculus introduced in \ref{sec-procs}, and prove subject 
reduction and progress for them. Then we define 
a liveness property for sessions, and show that processes typable by a local type context 
that's associated with a global type tree are guaranteed to satisfy this liveness property.
\subsection{Typing rules}
We give typing rules for our session calculus based on \cite{SynchronousSubtyping} and \cite{srpaper}. 

We distinguish between two kinds of typing judgements and type contexts.
\newcommand{\vdashm}{\vdash_\M}
\newcommand{\vdashp}{\vdash_\PT}
\begin{enumerate}
    \item A local type context $\Gamma$ associates participants with local type trees, as defined in
    c{def-type-ctx}. Local type contexts are used to type sessions (\cref{def:sessions}) i.e. a set of 
    pairs of participants and single processes composed in parallel. We express 
    such judgements as $\Gamma \vdashm \M$, or as \lstin{typ_sess M gamma} in Rocq.
    \item A process variable context $\Theta_\T$ associates process variables with local type trees,
    and an expression variable context $\Theta_\kf{e}$ 
    assigns sorts to expresion variables.
    Variable contexts are used to type single processes and expressions (\cref{def:processes}). 
    Such judgements are expressed 
    as $\Theta_\T, \Theta_\kf{e} \vdashp \PT : \T$, or in as \lstin{typ_proc theta_T theta_e P T}.
\end{enumerate}

\begin{table}[h]
{\footnotesize
\[
\begin{array}{@{}l@{}}
  \inferrule%[\rulename{steq}]
  {}
  {\Theta \vdashp n\colon \texttt{nat}\qquad \Theta \vdashp i\colon \texttt{int}
  \qquad \Theta \vdashp \mathtt{true}\colon \texttt{bool}\qquad \Theta \vdashp \mathtt{false}\colon \texttt{bool}\qquad \Theta,x\colon\ST \vdashp x\colon \ST}
      \\[3mm]
     \inferrule%[\rulename{steq}]
  {\Theta \vdashp e\colon \texttt{nat}}
  {\Theta \vdashp \mathtt{succ}\ e\colon \texttt{nat}}
      \quad
     \inferrule%[\rulename{steq}]
  {\Theta \vdashp e\colon \texttt{int}}
  {\Theta \vdashp \mathtt{neg}\ e\colon \texttt{int}}
  \qquad
     \inferrule%[\rulename{steq}]
  {\Theta \vdashp e\colon \texttt{bool}}
  {\Theta \vdashp \neg\ e\colon \texttt{bool}}
      \\[3mm]
     \inferrule%[\rulename{steq}]
  {\Theta \vdashp e_1\colon \ST \quad \Theta \vdashp e_2\colon \ST}
  {\Theta \vdashp  e_1\oplus e_2\colon \ST}
  \qquad
     \inferrule%[\rulename{steq}]
  {\Theta \vdashp e_1\colon \texttt{int} \quad \Theta \vdashp e_2\colon \texttt{int}}
  {\Theta \vdashp \ e_1 > e_2\colon \texttt{bool}}
  \qquad
     \inferrule%[\rulename{steq}]
  {\Theta \vdashp e\colon \ST \quad \ST \subso \ST'}
  {\Theta \vdashp e\colon \ST'}
\end{array}
\]}
\caption{Typing expressions}
\label{tbl:expr}
\end{table}
\begin{table}[h]
{\footnotesize
\[
\begin{array}{@{}l@{}}
  \inferrule[\rulename{t-end}]
  {}
  {\Theta \vdashp \inact \colon \tend}
\quad
  \inferrule[\rulename{t-var}]
  {}
  {\Theta,\Xv\colon\T \vdashp \Xv\colon\T}
 \quad
  \inferrule[\rulename{t-rec}]
  {\Theta,\Xv\colon\T \vdashp \PT\colon \T}
  {\Theta \vdashp \mu\Xv.\PT\colon\T}
 \quad
  \inferrule[\rulename{t-if}]
  {\Theta\vdashp e\colon \texttt{bool} \quad \Theta \vdashp \PT_1\colon \T  \quad \Theta \vdashp \PT_2\colon \T}
  {\Theta \vdashp \texttt{if}\ \kf{e} \ \mathtt{then}\ \PT_1  \ \mathtt{else} \ \PT_2\colon \T}
      \\[4mm]
  \inferrule[\rulename{t-sub}]
  {\Theta \vdashp \PT\colon \T \quad \T\leqslant \T'}
  {\Theta \vdashp \PT\colon \T'}
  \quad
   \inferrule[\rulename{t-in}]
  {\forall i\in I,\quad \Theta,x_i\colon\ST_i \vdashp \PT_i\colon \T_i}
  {\Theta \vdashp \sum_{i \in I} \prt{p} ?\ell_i(x_i).\PT_i\colon  
  \ltrec{\pp}{\ell_i(\ST_i).\T_i}_{i \in I}}
 \quad
   \inferrule[\rulename{t-out}]
  {\Theta \vdashp e\colon\ST \quad \Theta \vdashp \PT\colon \T}
  {\Theta \vdashp \prt{p}!\ell(\kf{e}).\PT \ \colon  \ {\pp}\sendsign\{{\ell(\ST).\T}\}}
\end{array}
\]}
\caption{Typing processes}
\label{tbl:proc}
\end{table}
\cref{tbl:expr} and \cref{tbl:proc} state the standard typing rules for expressions and processes.
We have a single rule for typing sessions:
\[
\inferrule[\rulename{t-sess}]{\forall i \in I: \qquad \vdashp \PT_i \colon \Gamma(\pp_i) \qquad 
\Gamma \assoc \G}
{\Gamma \vdashm \prod_i \pp_i \triangleleft \PT_i}
\]

\subsection{Subject Reduction, Progress and Session Fidelity}
The subject reduction, progress and non-stuck theorems from \cite{srpaper} \todo{give theorem no}
also hold in this setting, with minor changes in their statements and proofs. 
We won't discuss these proofs in detail.
\begin{lemma}
  If \lstin{typ_sess M gamma} and \lstin{unfoldP M M'} then \lstin{typ_sess M' gamma}.
\end{lemma}
\begin{proof}
  By induction on \lstin{unfoldP M M'}.
\end{proof}
\begin{theorem}[Subject Reduction]\label{theo-sub-red}
If \lstin{typ_sess M gamma} and \lstin{betaP_lbl M (lcomm p q ell) M'}, then 
there exists a typing context \lstin{gamma'} such that 
\lstin{tctxR gamma (lcomm p q ell) gamma'}
and \lstin{typ_sess M' gamma'}.
\end{theorem}
\begin{theorem}[Progress]
If \lstin{typ_sess M gamma}, one of the following hold :
\begin{enumerate}
  \item Either \lstin{unfoldP M M_inact} where every process making up \lstin{M_inact}
  is inactive, i.e. 
  \lstin{M_inact}$= \prod_{i=1}^{n}\pp_i \triangleleft \bf{0}$ for some $n$. 
  \item Or there is a \lstin{M'} such that \lstin{betaP M M'}.
\end{enumerate}
\end{theorem}
\begin{remark}
Note that in \cref{theo-sub-red} one transition between sessions corresponds to exactly one transition
between local type contexts with the same label. That is, every session transition is observed by the corresponding type.
This is the main reason for our choice of 
reactive semantics (\cref{def-sess-semantics}) as $\tau$ transitions are not observed by the type in
ordinary semantics. In other words, with $\tau$-semantics the typing relation is a \textit{weak simulation} \cite{weakbisim},
while it turns into a strong simulation with reactive semantics. For our Rocq implementation 
working with the strong simulation turns out be more convenient. 
\end{remark}
We can also prove the following correspondence result in the reverse direction to \cref{theo-sub-red}, analogus 
to \cref{theo-soundness}.
\begin{theorem}[Session Fidelity]\label{theo-sess-fid}
If \lstin{typ_sess M gamma} and \lstin{tctxR gamma (lcomm p q ell) gamma'}, there exists a
message label
\lstin{ell'} and a session \lstin{M'} such that \lstin{betaP_lbl M (lcomm p q ell') M'} 
and \lstin{typ_sess M' gamma'}.  
\end{theorem}
\begin{proof}
By inverting the local type context transition and the typing.
\end{proof}
\begin{remark}
Again we note that by \cref{theo-sess-fid} a single-step context reduction induces 
a single-step session reduction on the type. 
With the $\tau$-semantics the session reduction induced by 
the context reduction would be multistep.
\end{remark}
\subsection{Session Liveness}
We state the liveness property we are interested in proving,
and show that typable sessions have this property.
\begin{definition}[Session Liveness]\label[definition]{def-live-sess}
  Session $\M$ is live iff
  \begin{enumerate}
    \item ${\M}\longrightarrow^* {\M'} \Rrightarrow \pq \triangleleft \tout{\pp}{\ell_i}{x_i}.Q \;|\; \N$ implies 
    ${\M'}\longrightarrow^*{\M''} \Rrightarrow \pq \triangleleft Q \;|\; \N'$ for some $\M'', \N'$
    
    \item ${\M}\longrightarrow^* {\M'} \Rrightarrow \pq \triangleleft \texternal \tin{\pp}{\ell_i}{x_i}.Q_i \;|\; \N$ implies 
    ${\M'}\longrightarrow^*{\M''} \Rrightarrow \pq \triangleleft Q_i[v/x_i] \;|\; \N'$ for some $\M'', \N', i, v.$ 
  \end{enumerate}
\end{definition}
Session liveness, analogous to liveness for typing contexts (\cref{def-fair-live}), says that 
when $\M$ is live, if $\M$ reduces to a session $\M'$ containing a participant that's attempting to send 
or receive, then $\M'$ reduces to a session where that communication has happened. 
It's also called \textit{lock-freedom} in related work (\cite{fairnesslock,padovani}). 

We now prove that typed sessions are live. Our proof follows the following steps:
\begin{enumerate}
  \item Formulate an analogue of fairness for typable sessions,
  with the property that any finite session reduction path can be extended to a fair
  session reduction path.  
  \item Lift the typing relation to reduction paths, and show that fair session reduction paths
  are typed by fair local type context reduction paths.
  \item Prove that a certain transition eventually happens in the local context reduction 
  path, and that this means the desired transition is enabled in the session reduction path.   
\end{enumerate}
\begin{theorem}[Liveness by Typing]
  If $\Gamma \vdashm \M$ then $\M$ is live.
\end{theorem}
\begin{proof}
  We proceed by assuming that Item 1 of \cref{def-live-sess}
  doesn't hold and showing a contradiction. The case for when Item 2 doesn't hold is similar.

  Suppose  $\transcls{\M}{\M'} \Rrightarrow \pq \triangleleft \tout{\pp}{\ell_i}{x_i}.Q \;|\; \N$,
  and there is no $\M''$ such that 
  $\M'' \Rrightarrow \pq \triangleleft Q \;|\; \N'$. By subject reduction we have 
  $\Gamma' \vdashm \M'$ for some $\Gamma'$. By inverting $\rulename{T-out}$ and subtyping
  we get that $\Gamma'(\pq)=\ltsend{p}{\ell_i(S_i).\T_i}_{i \in I}$ for some $I$.
  Therefore by \cref{def-ctx-red} we have that 
  $\Gamma'
  \lts{\lblsend{q}{p}{\ell_i}}$ for some $i$.
  Furthermore by our assumption we don't ever have $\pp \triangleleft \texternal \tin{\pq}{\ell_i(x_i).\PT_i}$
  in a session $\M'$ can reduce to, hence for all $\M''$ s.t. $\M' \longrightarrow^* \M''$
  and $\Gamma'' \vdashm \M''$ we don't have $\Gamma''(\pp)=\ltrec{q}{...}$.
  Thus we have that there is no reduction path from $\Gamma'$ that contains a $\lblsync{p}{q}{\ell_k}$
  transition. Now suppose there exists a fair path from $\Gamma'$, then on this path 
  we perpetually have $\Gamma'(\pq) \lts{\lblsend{q}{p}{\ell_i}}$ 
  without ever transitioning via $\lts{\lblsync{p}{q}{\ell_k}}$, therefore this path is not live.
  Therefore $\Gamma$ is not a live type context.
  However, we have by \ref{theo-assoc-live} that $\Gamma$ is live, which is a contradiction.
  
  Now it suffices to show that there exists a fair path starting from $\Gamma'$. This 
  path always exists as we can just "schedule" the available synchronous transitions so that
  all of them are eventually executed e.g. in a round-robin-like way.  
\end{proof}