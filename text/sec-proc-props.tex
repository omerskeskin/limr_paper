\section{Properties of Sessions}\label{sec-proc-props}

We give typing rules for the session calculus introduced in \ref{sec-procs}, and prove subject 
reduction and deadlock freedom for them. Then we define 
a liveness property for sessions, and show that processes typable by a local type context 
that's associated with a global type tree are guaranteed to satisfy this liveness property.
\subsection{Typing rules}
We give typing rules for our session calculus based on \cite{SynchronousSubtyping} and \cite{srpaper}. 

We distinguish between two kinds of typing judgements and type contexts.
\newcommand{\vdashm}{\vdash_\M}
\newcommand{\vdashp}{\vdash_\PT}
\begin{enumerate}
    \item A local type context $\Gamma$ associates participants with local type trees, as defined in
    c{def-type-ctx}. Local type contexts are used to type sessions (\cref{def:sessions}) i.e. a set of 
    pairs of participants and single processes composed in parallel. We express 
    such judgements as $\Gamma \vdashm \M$, or as \lstin{typ_sess M gamma} or 
    \lstin{gamma $\;\vdash$ M} in Rocq.
    \item A process variable context $\Theta_\T$ associates process variables with local type trees,
    and an expression variable context $\Theta_\kf{e}$ 
    assigns sorts to expresion variables.
    Variable contexts are used to type single processes and expressions (\cref{def:processes}). 
    Such judgements are expressed 
    as $\Theta_\T, \Theta_\kf{e} \vdashp \PT : \T$, or in Rocq as 
    \lstin{typ_proc theta_T theta_e P T} or \lstin{theta_T, theta_e $\;\vdash$ P : T}.
\end{enumerate}
\begin{table}[h]
{\footnotesize
\[
\begin{array}{@{}l@{}}
  \inferrule[\rulename{t-end}]
  {}
  {\Theta \vdashp \inact \colon \tend}
\quad
  \inferrule[\rulename{t-var}]
  {}
  {\Theta,\Xv\colon\T \vdashp \Xv\colon\T}
 \quad
  \inferrule[\rulename{t-rec}]
  {\Theta,\Xv\colon\T \vdashp \PT\colon \T}
  {\Theta \vdashp \mu\Xv.\PT\colon\T}
 \quad
  \inferrule[\rulename{t-if}]
  {\Theta\vdashp e\colon \texttt{bool} \quad \Theta \vdashp \PT_1\colon \T  \quad \Theta \vdashp \PT_2\colon \T}
  {\Theta \vdashp \texttt{if}\ \kf{e} \ \mathtt{then}\ \PT_1  \ \mathtt{else} \ \PT_2\colon \T}
      \\[4mm]
  \inferrule[\rulename{t-sub}]
  {\Theta \vdashp \PT\colon \T \quad \T\leqslant \T'}
  {\Theta \vdashp \PT\colon \T'}
  \quad
   \inferrule[\rulename{t-in}]
  {\forall i\in I,\quad \Theta,x_i\colon\ST_i \vdashp \PT_i\colon \T_i}
  {\Theta \vdashp \sum_{i \in I} \prt{p} ?\ell_i(x_i).\PT_i\colon  
  \ltrec{\pp}{\ell_i(\ST_i).\T_i}_{i \in I}}
 \quad
   \inferrule[\rulename{t-out}]
  {\Theta \vdashp e\colon\ST \quad \Theta \vdashp \PT\colon \T}
  {\Theta \vdashp \prt{p}!\ell(\kf{e}).\PT \ \colon  \ {\pp}\sendsign\{{\ell(\ST).\T}\}}
\end{array}
\]}
\caption{Typing processes}
\label{tbl:proc}
\end{table}
Typing rules for expressions are standard and can be found in e.g. \cite{SynchronousSubtyping}, and are therefore omitted.
\cref{tbl:proc} state the standard \cite{srpaper,SynchronousSubtyping} typing rules for processes, 
which we don't elaborate on.
We additionally have a single rule for typing sessions:
\[
\inferrule[\rulename{t-sess}]{\forall i \in I: \qquad \vdashp \PT_i \colon \Gamma(\pp_i) \qquad 
\Gamma \assoc \G}
{\Gamma \vdashm \prod_i \pp_i \triangleleft \PT_i}
\]
\rulename{t-sess} says that a session made of the parallel composition of processes $\prod_i \pp_i \triangleleft \PT_i$ 
can be typed by an associated local context $\Gamma$ if the local type of participant $\pp_i$ in $\Gamma$
types the process  
\subsection{Properties of Typed Sessions}
The subject reduction, progress and non-stuck theorems from \cite{srpaper} \todo{give theorem no}
also hold in this setting, with minor changes in their statements and proofs. 
We won't discuss these proofs in detail.
\begin{lemma}[Typing after Unfolding \rocqlink{todo}]
  \label{lem-typ-unfold}
  If \lstin{gamma} $\vdashm$ \lstin{M}   and \lstin{M} $\Rrightarrow$ \lstin{M'} then \lstin{typ_sess M' gamma}.
\end{lemma}
\begin{theorem}[Subject Reduction \rocqlink{todo}]\label{theo-sub-red}
If \lstin{gamma} $\vdashm$ \lstin{M}   and 
\lstin{M} $\lts{\lblsync{p}{q}{\ell}}$ \lstin{M'}, then 
there exists a typing context \lstin{gamma'} such that 
\lstin{gamma} $\lts{\lblsync{p}{q}{\ell}}$ \lstin{gamma'}
and \lstin{gamma} $\vdashm$ \lstin{M}  .
\end{theorem}
\begin{remark}\label{remark-reactive-justif}
Note that in \cref{theo-sub-red} one transition between sessions corresponds to exactly one transition
between local type contexts with the same label. That is, every session transition is observed by the corresponding type.
This is the main reason for our choice of 
reactive semantics (\cref{def-sess-semantics}) as $\tau$ transitions are not observed by the type in
ordinary semantics. In other words, with $\tau$-semantics the typing relation is a \textit{weak simulation} \cite{weakbisim},
while it turns into a strong simulation with reactive semantics. For our Rocq implementation 
working with the strong simulation turns out be more convenient. 
\end{remark}
\begin{theorem}[Deadlock Freedom \rocqlink{todo}] \label{theo-progress}
If \lstin{gamma} $\vdashm$ \lstin{M}  , one of the following hold :
\begin{enumerate}
  \item Either \lstin{M} $\Rrightarrow$ \lstin{M_inact} where every process making up \lstin{M_inact}
  is inactive, i.e. 
  \lstin{M_inact} $\equiv \prod_{i=1}^{n}\pp_i \triangleleft \bf{0}$ for some $n$. 
  \item Or there is a \lstin{M'} such that \lstin{M} $\rdc$ \lstin{M'}.
\end{enumerate}
\end{theorem}
We can also prove the following correspondence result in the reverse direction to \cref{theo-sub-red}, analogous 
to \cref{theo-soundness}.
\begin{theorem}[Session Fidelity \rocqlink{todo}]\label{theo-sess-fid}
If \lstin{gamma} $\vdashm$ \lstin{M} and \lstin{gamma} $\lts{\lblsync{p}{q}{\ell}}$ \lstin{gamma'}, there exists a
message label
$\ell'$, a context \lstin{gamma''} and a session \lstin{M'} such that \lstin{M} $\lts{\lblsync{p}{q}{\ell'}}$ \lstin{M'} 
, \lstin{gamma} $\lts{\lblsync{p}{q}{\ell'}}$ \lstin{gamma''} and \lstin{typ_sess M' gamma''}.  
\end{theorem}
\begin{remark}
Again we note that by \cref{theo-sess-fid} a single-step context reduction induces 
a single-step session reduction on the type. 
With the $\tau$-semantics the session reduction induced by 
the context reduction would be multistep.
\end{remark}
Now the following type safety property follows from the above theorems:
\begin{theorem}[Type Safety \rocqlink{todo}]\label{theo-type-safe}
If \lstin{gamma} $\vdashm$ \lstin{M} and  \lstin{M} $\lts{}^*$ \lstin{M'} $\Rrightarrow$ 
\lstin{(p <- p_send q ell P \|\|\| q <- p_recv p xs \|\|\| M'')}, 
then \lstin{onth ell xs <> None}.
\end{theorem} 
The final, and the most intricate, session property we prove is liveness.
\begin{definition}[Session Liveness \rocqlink{todo}]\label[definition]{def-live-sess}
  Session $\M$ is live iff
  \begin{enumerate}
    \item ${\M}\longrightarrow^* {\M'} \Rrightarrow \pq \triangleleft \tout{\pp}{\ell_i}{x_i}.Q \;|\; \N$ implies 
    ${\M'}\longrightarrow^*{\M''} \Rrightarrow \pq \triangleleft Q \;|\; \N'$ for some $\M'', \N'$
    
    \item ${\M}\longrightarrow^* {\M'} \Rrightarrow \pq \triangleleft \texternal \tin{\pp}{\ell_i}{x_i}.Q_i \;|\; \N$ implies 
    ${\M'}\longrightarrow^*{\M''} \Rrightarrow \pq \triangleleft Q_i[v/x_i] \;|\; \N'$ for some $\M'', \N', i, v.$ 
  \end{enumerate}
  In Rocq this is expressed with the predicate \lstin{live_sess} \rocqlink{todo}:
  \begin{tcb}{Rocq}
Definition live_sess Mp := forall M, betaRtc Mp M -> 
  (forall p q ell e P' M', p <>q -> unfoldP M ( (p <-- p_send q ell e P') ||| M') -> exists M'',
  betaRtc M ((p <-- P')\|\|\|M''))
  /\
  (forall p  q llp M', p <>q -> unfoldP M ( (p <-- p_recv q llp) ||| M') -> 
    exists M'' P' e k, onth k llp = Some P' /\ betaRtc M ((p <-- subst_expr_proc P' e 0 0) |||M'')). 
  \end{tcb}
\end{definition}
Session liveness, analogous to liveness for typing contexts (\cref{def-fair-live}), says that 
when $\M$ is live, if $\M$ reduces to a session $\M'$ containing a participant that's attempting to send 
or receive, then $\M'$ reduces to a session where that communication has happened. 
It's also called \textit{lock-freedom} in related work (\cite{fairnesslock,padovani}). 

We can now prove that typed sessions are live. First we prove the following lemma:
\begin{lemma}[Fair Extension of Typed Sessions \rocqlink{todo}]\label{lem-fair-scheduling}
If \lstin{typ_sess M gamma}, then there exists a session reduction path \lstin{xs} 
starting from \lstin{M} such that the following fairness property holds:
\begin{itemize}
  \item On \lstin{xs}, whenever a transition with label $\lblsync{p}{q}{\ell}$ is enabled,
  a transition with label $\lblsync{p}{q}{\ell'}$ eventually occurs for some $\ell'$.  
\end{itemize}
\end{lemma}
\begin{proof}
  The desired path can be constructed by repeatedly cycling through all participants,
checking if there is a transition involving that participant, and executing that transition if there is.
Correctness follows from \cref{theo-sub-red} and \cref{theo-sess-fid}.
\end{proof}
\cref{lem-fair-scheduling} defines a "fairness" property for sessions analogous to \cref{def-fair-live}.
It then shows that there exists a fair path from any typable session. 
This resembles the \textit{feasibility} property expected from sensible notions of fairness \cite{fairness},
which states that any partial path can be extended into a fair one 
\footnote{Note that this fairness property for sessions is not actually feasible as there are partial paths 
starting with an untypable session that can't be extended into a fair one. 
Nevertheless, \cref{lem-fair-scheduling} turns out to be enough to prove our liveness property.}.  
\begin{remark}
As in the proof of \cref{theo-ctx-live}, the construction in \cref{lem-fair-scheduling} uses 
the \lstin{constructive_indefinite_description} axiom to construct a \lstin{CoFixpoint}.
Additionally, we use the axiom \lstin{excluded_middle_informative}
for the "check if there is a transition involving a participant" part of the scheduling algorithm.
The use of this axiom is probably not necessary but it makes the proof easier. 
\end{remark}

\begin{theorem}[Liveness by Typing \rocqlink{todo}]\label{theo-sess-live}
For a session \lstin{Mp}, if \lstin{exists gamma gamma} $\vdashm$ \lstin{Mp} then \lstin{live_sess Mp}.
\end{theorem}
\begin{proof}
We detail the proof for the send case of \cref{def-live-sess}, the case for the receive is similar.
Suppose that \lstin{Mp} $\rdc^*$ \lstin{M} and 
\lstin{M} $\Rrightarrow$ \lstin{((p <- p_send q ell e P') \|\|\| M')}.
Our goal is to show that there exists a \lstin{M''} such that
\lstin{M} $\rdc^*$ \lstin{((p <- P')\|\|\|M'')}. 
First, observe that by \rulename{R-Unfold} it suffices to show that 
\lstin{((p <- p_send q ell e P') \|\|\| M')} $\rdc^*$ \lstin{M''}
for some \lstin{M''}. Also note that  \lstin{gamma} $\vdashm$ \lstin{M} for some \lstin{gamma}
by \cref{theo-sub-red}, therefore 
\lstin{gamma} $\vdashm$ \lstin{((p <- p_send q ell e P') \|\|\| M')}
by \cref{lem-typ-unfold}.

Now let \lstin{xs} be a fair session reduction path starting from \lstin{((p <- p_send q ell e P') \|\|\| M')}, 
which exists by \cref{lem-fair-scheduling}. 
By \cref{theo-sub-red}, let \lstin{ys} be a
local type context reduction path starting with \lstin{gamma} 
such that every session in \lstin{xs} is typed by the context at the corresponding index 
of \lstin{ys}, and the transitions of \lstin{xs} and \lstin{ys} at every step match. 
Now it can be shown that \lstin{ys} is fair \rocqlink{todo}.
Therefore by \cref{theo-ctx-live} \lstin{ys} is live, 
so a \lstin{lcomm p q ell'} transition
eventually occurs in \lstin{ys} for some \lstin{ell'}.
Therefore  
\lstin{ys} = \lstin{gamma} $\rdc^*$ \lstin{gamma_0} $\lts{(\pp,\pq)\ell'}$ \lstin{gamma_1} $\rdc ..$
for some \lstin{gamma_0}, \lstin{gamma_1}. Now consider the session \lstin{M_0} typed by 
\lstin{gamma_0} in \lstin{xs}. 
We have \lstin{((p <- p_send q ell e P') \|\|\| M'')} $\rdc^*$ \lstin{M_0}
by \lstin{M_0} being on \lstin{xs}. 
We also have that
\lstin{M_0} $\lts{(\pp,\pq)\ell''}$ \lstin{M_1} for some $\ell''$, \lstin{M_1} by \cref{theo-sess-fid}.
Now observe that \lstin{M_0} $\equiv$ \lstin{((p <- p_send q ell e P') \|\|\| M'')} for some 
\lstin{M''} as no transitions involving \lstin{p} have happened on the reduction path to \lstin{M_0}.
Therefore $\ell=\ell''$, so \lstin{M_1} $\equiv$ \lstin{((p <- P') \|\|\| M'')} for some \lstin{M''}, as needed.
\end{proof}
