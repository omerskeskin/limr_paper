\section{Notes on the Rocq Implementation}\label{appx:coq}
\subsection{Structure of the Code}
Our code extends Tadayoshi Kamegai's library \cite{srpaper} for subject reduction,
which formalises most of the definitions in \cref{sec:types}.
Most of the code submitted is from that library and our contribution is to be found mainly in the following files:
\begin{itemize}
    \item \path{src/lcontext.v}: Contains definitions of local typing contexts (\cref{def-type-ctx}),
    typing context reductions (\cref{def-ctx-red}) and theorems about them (\cref{appx:proof}, Lemmas A.1 to A.5)
    \item \path{src/assoc.v}: Contains the definition of association (\cref{def-assoc}) and  
    lemmas on them (\cref{appx:proof}, Lemmas A.6 to A.8)
    \item \path{lemma/correspond.v}: Contains the proof for the soundness of association
    (\cref{appx:proof}, Lemmas A.10 to A.11)
   \item \path{src/path_props.v}: Defines safety (\cref{def:safety}) and fairness and liveness (\cref{def-fair-live})
   for reduction paths
   \item \path{lemma/example_3.v}: Illustrates the definitions from \path{src/path_props.v} and 
   \path{src/lcontext.v} with examples. In particular, we formalise the reductions in 
   Example \ref{exam:reductions}, and the safety property of $\Gamma$ in Example \ref{exam:safe}. We also 
   show the fairness of the infinite path in Example \ref{exam:live}.  
\end{itemize}
Additionally, the submitted code contains Adam Chlipala's CpdtTactics library \cite{cpdt}.
We mainly use that library for its powerful \lstin{crush} tactic which automates a lot of the 
tedium out of the proofs. Note that we didn't rely on automation as much as we could have 
due to performance concerns. Many proofs, especially the ones using rewrites on finite maps, could be made
shorter by using more automation features.
\subsection{Axioms and Admitted Lemmas}
The axioms we use are the dependent equality axiom \cite[Eqdep]{coqstl}
employed by the CpdtTactics library, constructive indefinite description
\cite[ChoiceFacts]{coqstl} for some constructive proofs, particularly \cref{theo-soundness},
and coinductive extensionality as described in \cref{sec:types}. 
In \path{lemma/correspond.v} we admit two lemmas, \lstin{projection_implies_slist_send}
and \lstin{projection_implies_slist_recv}, which state that well-formed global types 
project onto local types with non-empty continuations. The proof is tedious and divorced from the rest 
of the formalised proofs, so we have left it as admitted. 
Additionally, in  \path{lemma/example_3.v} we have admitted three lemmas which state that 
certain types satisfy \rulesafesync (represented in Rocq with \lstin{weak_safety}). Again, the 
proofs are long and non-informative, our purpose in proving \lstin{gamma_safe} is to showcase 
the coinductive definition of safety and show how it corresponds to checking that all reducts 
satisfy \rulesafesync.
