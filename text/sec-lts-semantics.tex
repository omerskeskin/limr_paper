\section{Semantics of Types}
\label{sec:lts}
In this section we introduce local type contexts, and define Labelled Transition System 
semantics on these constructs.
\subsection{Local Type Contexts and Reductions} 
We start by defining typing contexts as finite mappings of participants to local type trees.
\begin{minipage}{0.45\textwidth}
\begin{definition}[Typing Contexts]\label[definition]{def-type-ctx}
  \[
  \Gamma \;::=\;%
  \emptyset \SEP \Gamma, {\prt{p}}:\T
\]
\end{definition}
\end{minipage}
\begin{minipage}{0.5\textwidth}
  \begin{tcb}{Rocq}
Module M := MMaps.RBT.Make(Nat).
Module MF := MMaps.Facts.Properties Nat M.
Definition tctx: Type := M.t ltt.
\end{tcb}
\end{minipage}

Intuitively, p : {\T} means that participant p is associated with a process that has the type tree T. 
We write $\dom{\Gamma}$ to denote the set of participants occuring in $\Gamma$. We write $\Gamma(\pp)$ for the type of $\pp$ in $\Gamma$. 
We define the composition $\Gamma_1,\Gamma_2$ iff $\dom{\Gamma_1} \cap \dom{\Gamma_2}=\emptyset$.

In the Rocq implementation we implement local typing contexts as finite maps of 
participants, which are represented as natural numbers, and local type trees.
We use the red-black tree based finite map implementation of the MMaps library \cite{mmaps}.

\begin{remark}
From now on, we assume the all the types in the local type contexts always have non-empty continuations.
In Rocq terms, if \lstin{T} is in context \lstin{gamma} then \lstin{wfltt T} holds. This is expressed by the predicate
\lstin{tctx_wf: tctx -> Prop}.
\end{remark}

We now give LTS semantics to local typing contexts, for which we first define the transition labels.
\begin{definition}[Transition labels] A transition label $\alpha$ has the following form:
  \begin{align*}
    \alpha ::&= \lblrec{p}{q}{\ell(S)}  && 
    \text{(${\prt{p}}$ receives a value of sort ${S}$ from ${\prt{q}}$ with message label ${\ell}$)}\\
    & \SEP \lblsend{p}{q}{\ell(S)} && \text{(${\prt{p}}$ sends a value of sort $S$ to ${\prt{q}}$ with message label $\ell$ )}\\
    & \SEP \lblsync{p}{q}{\ell} && \text{(A synchronised communication from $\pp$ to $\pq$ occurs via label $\ell$)}\\
  \end{align*}
\end{definition}
Next we define labelled transitions for local type contexts.
\begin{definition}[Typing context reductions] \label[definition]{def-ctx-red}
The typing context transition $\lts{\alpha}$ is defined inductively by the following rules:
  \[
\begin{array}[t]{@{}c@{}}
\inferrule[]{
     k \in I }{
      \lbltrans
    {\prt{p} : \procinset{\prt{q}}{\ell_i(\S_i)}{\T_i}{i \in I}} 
    {\lblrec{p}{q}{\ell_k(S_k)}}
    {\prt{p} : \T_k}
    }
    \; \rulename{ $\Gamma$-\andsign}
    \;
  \inferrule[]{
     k \in I }{
      \lbltrans
      {\prt{p} : \procoutset{\prt{q}}{\ell_i(\S_i)}{\T_i}{i \in I}} 
      {\lblsend{p}{q}{\ell_k(S_k)}}
      {\prt{p} : \T_k}
    }
    \; \rulename{ $\Gamma$-$\oplus$}
  \\\\
    \inferrule[]{
  \lbltrans{\Gamma}{\alpha}{\Gamma'}}
  {
    \lbltrans{\Gamma, \prt{p} : \T}
     {\alpha} {\Gamma', {\prt{p}} : \T} 
  }
  \enspace \rulename{$\Gamma$-,}
  
  \qquad
    \inferrule[]{
    \lbltrans{\Gamma_1}{\lblsend{\prt{p}}{\prt{q}}{\ell(S)}}{\Gamma'_1}
    \qquad  
    \lbltrans{\Gamma_2}{\lblrec{\prt{q}}{\prt{p}}{\ell(S')}}{\Gamma'_2}
    \qquad S \subso \S'
    }{
    \lbltrans{\Gamma_1, \Gamma_2 }
    {\lblsync{p}{q}{\ell}}
    {\Gamma'_1, \Gamma'_2}
    }
    \enspace \rulename{$\Gamma$-$\oplus$\andsign}
\end{array}
\]
We write $\Gamma \lts{\alpha}$ if there exists $\Gamma'$ such that $\Gamma\lts{a}\Gamma'$.  
We define a reduction $\Gamma\lts{}\Gamma'$ that holds iff \; $\lbltrans{\Gamma}{\lblsync{\prt{p}}{\prt{q}}{\ell}}{\Gamma'}$ for some ${\prt{p}}$, ${\prt{q}}$, $\ell$. We write $\Gamma\lts{}$ iff \; $\Gamma\lts{}\Gamma'$ for some $\Gamma'$. 
We write $\lts{}^*$ for the reflexive transitive closure of $\lts{}$.
\end{definition}
$\ruleredsend$ and $\ruleredrec$, express a single participant sending or receiving.
$\ruleredsync$ expresses a synchronised communication where one participant sends while another receives,
and they both progress with their continuation. $\ruleredvar$ shows how to extend a context.  
In Rocq typing context reductions are defined with the predicate \lstin{tctxR}.

\begin{minipage}{0.45\textwidth}
  \begin{tcb}{Rocq}
Notation opt_lbl := nat.
Inductive label: Type :=
  | lrecv: part -> part -> option sort -> opt_lbl -> label
  | lsend: part -> part -> option sort -> opt_lbl -> label
  | lcomm: part -> part -> opt_lbl -> label.
\end{tcb}
\end{minipage}
\begin{minipage}{0.45\textwidth}
\begin{tcb}{Rocq}
Inductive tctxR: tctx -> label -> tctx -> Prop :=
  | Rsend: ...
  | Rrecv: ...  
  | Rcomm: ...
  | RvarI: ...
  | Rstruct: forall g1 g1' g2 g2' l, tctxR g1' l g2' ->
    M.Equal g1 g1' -> M.Equal g2 g2' -> tctxR g1 l g2.
\end{tcb}
\end{minipage}

The first four constructors in the definition of \lstin{tctxR} corresponds to the 
rules in \cref{def-ctx-red}, and \lstin{Rstruct} expresses the indistinguishability
of local contexts under the \lstin{M.Equal} predicate from the MMaps library. 

We illustrate typing context reductions with an example.
\begin{example}\label{exam:reductions}
  Let $\Gamma = \{{\prt{p}}:\T_{\prt{p}}, \; {\prt{q}} : \T_{\prt{q}},\; {\prt{r}}: \T_{\prt{r}}\}$ where
  $\T_{\prt{p}} = \procoutmult{\prt{q}}{\ell_0(\tint).\T_{\prt{p}},\ell_1(\tint).\tend}$
  $\T_{\prt{q}} = \procinmult{\prt{p}}{\ell_0(\tint).\T_{\prt{q}},  \ell_1(\tint).\procoutsingle{\prt{r}}{\ell_2(\tint)}{\tend}}$
  and 
  $\T_{\prt{r}} = \procinmult{\prt{q}}{\ell_2(\tint).\tend}$.
We have the reductions $\Gamma\lts{\lblsend{p}{q}{\ell_0(\tint)}} \Gamma$ and
$\Gamma \lts{\lblrec{\prt{q}}{\prt{p}}{\ell_0(\tint)}} \Gamma$, which synchronise to give the reduction
and $\Gamma \lts{\lblsync{\prt{p}}{\prt{q}}{\ell_0}}  \Gamma$. Similarly via synchronised communication of $\pp$ and $\pq$
via message label $\ell_1$ we get $\Gamma \lts{\lblsync{\prt{p}}{\prt{q}}{\ell_1}} \Gamma'$
where $\Gamma'$ is defined as  $\{{\prt{p}} : \tend, \; {\prt{q}}: 
\procoutsingle{\prt{r}}{\ell_2(\tint)}{\tend}, {\prt{r}} : \T_{\prt{r}}\}$.
We further have that $\Gamma' \lts{\lblsync{q}{r}{\ell_2}} \Gamma_\mathtt{end}$ where
$\Gamma_{\mathtt{end}}$ is defined as 
$\{{\prt{p}}:\tend, \; {\prt{q}} : \tend,\; {\prt{r}}: \tend \}$.

In Rocq, $\Gamma$ is defined the following way \rocqlink{todo}:
\begin{tcb}{Rocq}
Definition prt_p:=0.
Definition prt_q:=1.
Definition prt_r:=2.
CoFixpoint T_p := ltt_send prt_q [Some (sint,T_p); Some (sint,ltt_end); None].
CoFixpoint T_q := ltt_recv prt_p [Some (sint,T_q); Some (sint, ltt_send prt_r [None;None;Some (sint,ltt_end)]); None].
Definition T_r := ltt_recv prt_q [None;None; Some (sint,ltt_end)].
Definition gamma := M.add prt_p T_p (M.add prt_q T_q (M.add prt_r T_r M.empty)).
\end{tcb}
Now $\Gamma \lts{\lblsync{\prt{p}}{\prt{q}}{\ell_0}}  \Gamma$  can be expressed as 
\lstin{tctxR gamma (lsend prt_p prt_q (Some sint) 0) gamma}.
\end{example}

\subsection{Global Type Reductions}
As with local typing contexts, we can also define reductions for global types.
\begin{definition}[Global type reductions]
  The global type transition $\lts{\alpha}$ is defined coinductively as follows.
  \[
\begin{array}[t]{@{}c@{}}
\cinferrule[]{
     k \in I }{
      \lbltrans
    {\GvtPair{p}{q}{\ell_i(S_i).\G_i}_{i \in I}} 
    {\lblsync{p}{q}{\ell_k}}
    {\G_k}
    }
    \enspace \rulename{GR-$ \sendsign \andsign$}
    \\\\
    \cinferrule[]{
     \forall i \in I \enspace \lbltrans{\G_i}{\alpha}{\G'_i} \qquad
     \subject{\alpha} \cap \{\prt{p},\prt{q}\} = \emptyset
     \qquad \forall i \in I \enspace \{\prt{p},\prt{q}\} \subseteq \participant{\G_i}
     }{
      \lbltrans
    {{\GvtPair{p}{q}{\ell_i(S_i).\G_i}_{i \in I}}} 
    {\alpha}
    {{\GvtPair{p}{q}{\ell_i(S_i).\G'_i}_{i \in I}}}
    }
    \enspace \rulename{GR-Ctx}
\end{array}
\]
\end{definition}
\rulegredsendrec says that a global type tree with root $\pp \rightarrow \pq$ can
transition to any of its children corresponding to the message label choosen by $\pp$. 
\rulegredctx says that if the subjects of $\alpha$ are disjoint from the root and all its children
can transition via $\alpha$, then the whole tree can also transition via $\alpha$, with the root remaining 
the same and just the subtrees of its children transitioning.
In Rocq global type reductions are expressed using the coinductively defined predicate \lstin{gttstepC}. 
For example, $\lbltrans{\G}{\lblsync{p}{q}{\ell_k}}{\G'}$ translates to \lstin{gttstepC G G' p q k}. 
We refer to \cite{srpaper} for details.

\subsection{Association Between Local Type Contexts and Global Types}
We have defined local type contexts which specifies protocols bottom-up 
by directly describing the roles of every participant,
and global types, which give a top-down view of the whole protocol, and the transition relations on them.
We now relate these local and global definitions by defining \textit{association} between local type
context and global types.
\begin{definition}[Association \rocqlink{todo}]\label[definition]{def-assoc}
A local typing context $\Gamma$ is associated with a global type tree $\G$, written $\Gamma \assoc \G$,
if the following hold:
\begin{itemize}
  \item For all $\pp \in \funprt(\G)$, $\pp \in \dom{\Gamma}$ and $\Gamma(\pp) \subtp \G \projf{\pp}$.
  \item For all $\pp \notin \funprt(\G)$, either $\pp \notin \dom{\Gamma}$ or $\Gamma(\pp)=\tend$. 
\end{itemize}
In Rocq this is defined with the following:
\begin{tcb}{Rocq}
Definition assoc (g: tctx) (gt:gtt) := 
    forall p, (isgPartsC p gt -> exists Tp, M.find p g=Some Tp /\  
        issubProj Tp gt p) /\
         (~ isgPartsC p gt -> forall Tpx, M.find p g = Some Tpx -> Tpx=ltt_end).
\end{tcb}
\end{definition} 
Informally, $\Gamma \assoc \G$ says that the local type trees in $\Gamma$ 
obey the specification described by the global type tree $\G$. 
\begin{example}\label{exam:assoc}
  In Example \ref{exam:reductions},
  we have that $\Gamma \assoc \G$ where
  $\G := \GvtPair{\prt{p}}{\prt{q}}{\ell_0(\tint).\G,\ell_1(\tint).\GvtPair{\prt{q}}{\prt{r}}{\ell_2(\tint).\tend}}$.
  In fact, we have $\Gamma(\ps) = \projfn{\G}{\ps}$ for $\ps \in \{\prt{p},\pq,\pr\}$. 
  Similarly, we have $\Gamma' \assoc \G'$ where
    $\G' := \GvtPair{\prt{q}}{\prt{r}}{\ell_2(\tint).\tend}$
\end{example}

It is desirable to have the association
be preserved under local type context and global type reductions, that is, 
when one of the associated constructs "takes a step" so should the other. We formalise this property with 
soundness and completeness theorems.

\begin{theorem}[Soundness of Association \rocqlink{todo}]\label{theo-soundness} 
  If \lstin{assoc gamma G} and \lstin{gttstepC G G' p q ell}, then there is a local type context \lstin{gamma'},
  a global type tree \lstin{G''} and a message label \lstin{ell'} such that 
  \lstin{gttStepC G G'' p q ell'}, \lstin{assoc gamma' G''} and 
  \lstin{tctxR gamma (lcomm p q ell') gamma'}.
\end{theorem}
\begin{theorem}[Completeness of Association \rocqlink{todo}] \label{theo-completeness}
  If \lstin{assoc gamma G} and \lstin{tctxR gamma (lcomm p q ell) gamma'}, 
  then there exists a global type tree \lstin{G'} such that 
  \lstin{assoc gamma' G'} and \lstin{gttstepC G G' p q ell}. 
\end{theorem}
\begin{remark}
  Note that in the statement of soundness we allow the message label for the local type context reduction 
  to be different to the message label for the global type reduction. 
  This is because our use of subtyping in association causes the entries in the local type context
  to be less expressive than the types obtained by projecting the global type. For example consider
    $\Gamma = \pp : \ltsend{q}{\ell_0(\tint).\tend}, \; \pq : \ltrec{p}{\ell_0(\tint).\tend, \ell_1(\tint).\tend}$
    and $\G= \GvtPair{p}{q}{\ell_0(\tint).\tend, \ell_1(\tint).\tend}$.
  We have $\Gamma \assoc \G$ and $\G \lts{\lblsync{p}{q}{\ell_1}}$.
  However $\Gamma \lts{\lblsync{p}{q}{\ell_1}}$ is not a valid transition.
\end{remark}
